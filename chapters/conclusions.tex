\chapter{Summary and conclusion}

This dissertation presented two searches of new scalars leading to single-lepton final states with high $b$-jet multiplicity, using the data collected by the \acrshort{ATLASlabel} experiment at the \acrshort{LHClabel} between 2015 and 2018, at a center-of-mass energy $\sqrt{s}=13$~TeV for a total of 139~fb$^{-1}$ integrated luminosity. The first analysis searches for a heavy charged Higgs boson decaying to a top and a bottom quark, produced in association with top and bottom quarks; while the second searches for the FCNC decay of a top to an $up$-type quark and a new neutral scalar, decaying into a pair of $b$-quarks. The targetted processes are very different, but they share the final state and the dominant \ttjets\ background. Hence, similar approaches can be used for both analyses. Data-based corrections to the \ttbar\ background are applied to improve its modelling, and the signal sensitivity is enhanced with the implementation of parameterised NNs in regions where the signal rate is expected to be largest. The neural network output depends on the target scalar mass, and a fit to the data is performed simultaneously in the different analysis regions, separately for each signal type and mass. No significant excess above the expected Standard Model background is found and 95\% confidence level upper limits on the production of the scalars are set.

For the first analysis, the observed (expected) upper limits are set on the production cross-section $\sigma(pp\to tb H^+)$ times the branching fraction $\text{B}(H^+\to tb)$, which range from $\sigma\times\text{B}=3.6(2.6)$~pb at $m_{H^+}=200$~GeV to $0.036(0.019)$~pb at $m_{H^+}=2$~TeV. Compared to the previous \acrshort{ATLASlabel} search with 36~fb$^{-1}$, the observed $\sigma\times\text{B}$ limits improved by 5\% to 70\%, depending on the $H^+$ mass, except the lowest one. The measurement at low $H^+$ mass region is dominated by systematic uncertainties, while for higher values, the improvement is achieved with the increase in statistics, tighter lepton triggers, improved $b$-tagging, along with the parameterised NN approach. In the context of the hMSSM and several $M_{h}^{125}$ scenarios, values of $\tan\beta\in[0.5,2.1]$ are excluded for $m_{H^+}\in[200,1200]$~GeV. In addition, $\tan\beta>34$ values are excluded for $m_{H^+}\in[200,750]$~GeV. Compared to previous results, this analysis extends the exclusion at low and high $\tan\beta$ values, specially for high $m_{H^+}$. The analysis was interpreted for the first time in literature in the context of 2HDM+a and, although the $\MET+Z(\ell^+\ell^-)/h(\bbar)$ searches are the most sensitive to the model, this analysis contributes to the exclusion of high $m_a$ values for up to intermediate values of $m_{H^+}$ and low values of $\tan\beta$.

The $t\to qX$ analysis sets the observed (expected) upper limits between 0.019\% (0.017\%) and 0.062\% (0.056\%) are derived for the branching fraction $\text{B}(t\to uX)\times\text{B}(X\to \bbar)$ and between 0.018\% (0.015\%) and 0.078\% (0.056\%) for the branching fraction $\text{B}(t\to cX)\times\text{B}(X\to\bbar)$ in the explored mass range. The same workflow is  used to derive limits for the branching fraction of a top quark into the Standard Model Higgs boson and an $up$-type quark, resulting in 95\% confidence level upper limits of 0.077\% (0.088\%) for the observed (expected) $\text{B}(t\to uH)$ and 0.12\% (0.076\%) for the observed (expected) $\text{B}(t\to cH)$.

\todo{Outlook here, maybe?}


