\chapter{Summary and conclusions}
\label{chapter:summary}

This dissertation presented two searches of new scalars leading to single-lepton final states with high $b$-jet multiplicity, using the proton-proton collisions data collected by the \acrshort{ATLASlabel} experiment at the \acrshort{LHClabel} between 2015 and 2018 at a center-of-mass energy $\sqrt{s}=13$~TeV for a total of 139~fb$^{-1}$ integrated luminosity. The first analysis searches for a heavy charged Higgs boson decaying to a top and a bottom quarks, produced in association with top and bottom quarks. The second analysis searches for a FCNC decay process of a top to an $up$-type quark and a new neutral scalar decaying into a pair of bottom quarks. The two searched processes are very different but share the same final state and \ttjets\ is a common dominant background. Hence, similar approaches are used for both analyses. Data-based corrections to the \ttbar\ background are applied to improve its modelling and the signal sensitivity is enhanced with the implementation of parameterised neural networks in regions where the signal is expected to be largest. The neural network output depends on the neutral or charged scalar mass hypothesis. A fit to the data is performed simultaneously in the analysis regions, separately for each signal type and mass. No significant excess above the expected Standard Model background is found and 95\% confidence level upper limits on the production of the scalars are extracted.\\

For the charged Higgs boson search, observed (expected) upper limits are set for the production cross-section $\sigma(pp\to tb H^+)$ times the branching fraction $\text{B}(H^+\to tb)$, and range from $\sigma\times\text{B}=3.6(2.6)$~pb at $m_{H^+}=200$~GeV to $0.036(0.019)$~pb at $m_{H^+}=2$~TeV. The observed $\sigma\times\text{B}$ limits improve by 5\% to 70\%, depending on the $H^+$ mass, when compared to the previous \acrshort{ATLASlabel} search with 36~fb$^{-1}$, except for the lowest mass. The improvements are achieved through increased statistics, tighter lepton triggers, improved $b$-tagging, and the use of parameterised neural networks. The limits at the low $H^+$ mass region are dominated by systematic uncertainties. In the context of the hMSSM model and several $\text{M}_{h}^{125}$ scenarios, values of $\tan\beta\in[0.5,2.1]$ are excluded for $m_{H^+}\in[200,1200]$~GeV. In addition, values of $\tan\beta>34$ are excluded for $m_{H^+}\in[200,750]$~GeV. Compared to previous results, this analysis extends the exclusion at low and high $\tan\beta$ values, especially for high $m_{H^+}$. For the first time in literature, the analysis is interpreted in the context of a 2HDM+a model. Although the $\MET+Z(\ell^+\ell^-)/h(\bbar)$ searches are the most sensitive ones to the model, this analysis contributes to the exclusion of high $m_a$ values for up to intermediate values of $m_{H^+}$ and low values of $\tan\beta$.\\

The $t\to qX$ analysis sets observed (expected) upper limits between 0.019\% (0.017\%) and 0.062\% (0.056\%) for the branching fraction $\text{B}(t\to uX)\times\text{B}(X\to \bbar)$, and between 0.018\% (0.015\%) and 0.078\% (0.056\%) for the branching fraction $\text{B}(t\to cX)\times\text{B}(X\to\bbar)$ in the explored mass range. The same analysis is used to derive limits for the branching fraction involving the SM Higgs boson, resulting in 95\% confidence level upper limits of 0.077\% (0.088\%) for the observed (expected) $\text{B}(t\to uH)$ and 0.12\% (0.076\%) for the observed (expected) $\text{B}(t\to cH)$.

