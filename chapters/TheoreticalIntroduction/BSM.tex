\section{Beyond the Standard Model}
Intro from Shota 

Introduce 2HDM -> potential -> massterms -> H+
FCNC depending on the type
Can be part of other theories like SUSY
Brief SUSY for hmmsm 
(what about mh125 models? what are those?)

Charged Higgs phenomenlogy (production and decay)

Flavon -> types of mechanisms -> U(1) (check paper and internal note)
Flavon, prediction and decay aswell?
assuming f / lambda sm to be cabibo angle 0.23???


---
CONF %https://cds.cern.ch/record/2777863/files/ATLAS-CONF-2021-036.pdf
The ATLAS experiment [6] at the Large Hadron Collider (LHC) [7] is a general-purpose detector that allows
for a wide range of DM searches. In the following, the focus will be entirely on the hypothesis that DM is a
weakly interacting massive particle (WIMP) [8] and, more specifically, a Dirac fermion. WIMPs could in
principle interact with the SM sector in different ways. A particular strength of collider searches lies in the
fact that the high-energy collisions of SM particles could not only produce DM directly under controlled
experimental conditions but also provide access to particles mediating the interactions between DM and
the SM sector. A mediator produced in a collision could decay to DM particles, which themselves could
not be detected and would lead to missing transverse momentum (with magnitude \MET). Alternatively, a
mediator could decay back into SM particles, from which its properties could be reconstructed.
Dark matter searches at the LHC explore both avenues in the quest to solve the puzzle of DM. Invisible
mediator decays can be detected only if the mediator is produced in association with another particle,
for example from initial-state radiation that results in a hadronic jet, leading to a characteristic \MET+jet
signature [9, 10]. Visible mediator decays allow for the reconstruction of the mediator particle from
its decay products, for example in the context of resonance searches, if the mediator is produced in the
s-channel [11-15].

The above-mentioned searches are traditionally interpreted in the context of so-called simplified models of
DM, which rely on a minimal set of new particles and interactions. The most commonly used among these
simplified models postulate the existence of a single fermionic DM particle and a single mediator, which,
depending on the model, could be a vector, axial-vector, scalar, or pseudo-scalar particle [16-18]. The
models are characterised by a minimal set of free parameters, namely the masses of the DM and mediator
particles and the couplings of the mediator to the SM and dark sectors.

In this note, a more complete benchmark model is used. It provides an ultra-violet complete and
renormalisable framework for the interpretation of DM searches. It is built upon the assumption that the
SM Higgs boson is part of an extended Higgs sector with two complex Higgs doublets. This is a key
assumption in many theories extending the SM, such as supersymmetry [19]. The interaction between the
SM sector and DM in this model is mediated by a pseudoscalar particle a. Pseudoscalar mediators are not
strongly constrained by direct-detection experiments because the tree-level amplitude for the DM-nucleon
elastic scattering is suppressed by the momentum transfer in the non-relativistic limit [20]. This renders
collider searches for these processes particularly useful.

The model, referred to as Two Higgs Doublet Model (2HDM) + pseudoscalar mediator (a), 2HDM+a, is
the simplest and currently only gauge-invariant and UV-complete extension of the simplified model with a
pseudoscalar mediator, and has been first introduced in Ref. [21]. The model is adopted as a common LHC
benchmark model by the LHC Dark Matter Working Group [22], which is a joint forum of theory groups
and the ATLAS, CMS, and LHCb collaborations. This model, unlike the simplified models described above,
predicts a wide variety of detector signatures. Among the most prominent signatures in the parameter space
explored in this note are the production of DM in association with a Higgs boson (\MET+h signatures) or
with a Z boson (\MET+Z signatures). Further signatures are related to DM production in association with
a top quark and a W boson (\MET+Wt), visible decays of the additional heavy Higgs bosons, and invisible
decays of the SM Higgs boson to DM.
A summary paper including constraints on the 2HDM+a benchmark based on a variety of dark matter
searches using 36 fb$^{-1}$ data of sqrts = 13 TeV proton-proton collisions has been previously published by the
ATLAS Collaboration [23]. Constraints on the model have also been derived by the CMS Collaboration
using the results from searches in the \MET+ h(\bbar) [24] and \MET.
-------------------------