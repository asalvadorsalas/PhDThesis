\chapter{The \acrlong{SMlabel} of Particle Physics and beyond}

Be great to say that the lagrangian we want has to describe reality, what we want is something Lorentz invariant, Pointcaré group?

The \acrfull{SMlabel} of particle physics~\cite{PhysRevLett.19.1264,GLASHOW1961579,RevModPhys.52.525} is the theoretical framework
that so far better describes subatomic particles and their interactions.
It is a \acrlongpl{QFT} (\acrshort{QFT}) and since its initial development in the 1960's,
the model has been overwhelmingly successful and guided many experimental achievements
including the discovery of the top quark~\cite{topsearch1995,PhysRevLett.74.2626} in 1995
and the Higgs boson at the LHC in 2012~\cite{ATLASHiggs2012,CMShiggs2012}.
Regardless of the success of the model, there are known phenomena not covered in the model
and other questions which clearly point to the need of a new theories.\\
This chapter presents an overview of the \acrshort{SMlabel} with a brief summary of the particle content as of yet without
[going] into underlying theoretical details. \textcolor{red}{Next, the description of\ldots}\\

Throughout this dissertation, natural units are used: the speed of light and the reduced Plank constant are set to unity ($c=\hbar=1$),
electric charges are expressed in units of the electron electric charge ($-e$) and masses are expressed in terms of energy (eV).
Within the theoretical developments in this chapter, the Einstein's summation convention is used by default. 

\section{The \acrlong{SMlabel} of Particle Physics}

From the mathematical point of view, the \acrshort{SMlabel} is a renormalisable non-abelian gauge \acrshort{QFT} based on the
symmetry group, 

\begin{equation}
    \label{Theory_eq:SMgroup}
    SU(3)_C\otimes SU(2)_L\otimes U(1)_Y
\end{equation}

where $SU(3)_C$ is the group described by \acrlongpl{QCD}~(\acrshort{QCD})~\cite{QCD} that
represents the strong interactions of colored quarks and gluons (strong force),
while $SU(2)_L\times U(1)_Y$ is the inclusive representation of both electromagnetic (EM) and weak interactions
described by the \acrlongpl{EW}~(\acrshort{EW}) theory~\cite{PhysRevLett.19.1264,Salam:1968rm,GLASHOW1961579}. The \acrshort{SMlabel} describes all the interactions between elementary
particles except gravity, for which no renormalisable \acrshort{QFT} has been formulated so far.
The following sections, introduce the particles of the \acrshort{SMlabel} and the theories that describe their interactions.  

\subsection{Particle content of the \acrlong{SMlabel}}

In the \acrshort{SMlabel}, elementary particles are described as excitations of quantum fields.
There are two main classes of particles within the theory: \textit{fermions} and \textit{bosons}.
The main difference between the two is the spin: fermions have half-integer spin and therefore obey the Pauli exclusion
principle~\cite{Pauli1925}, while bosons have integer spin.

\subsubsection{Fermions}

Fermions can be divided further into two categories: quarks and leptons, based on their interactions, or their charges.
Both types manifest in \acrshort{EW} interactions, having a weak isospin $T_3=\pm1/2$ although only the quarks experience
the strong interaction in addition,
so have strong charge refereed to as \textit{colour}. Quarks have a fractional electric charge $Q=2/3$ or $1/3$, and the colour,
which values are usually denoted as \textit{red}, \textit{green} and \textit{blue}. \textcolor{red}{Table~[] presents a summary of the fundamental fermions and their characteristics}.\\

There is a total of six quark types, named \textit{flavours} and are split into three generations.
The first generation consists in the \textit{up} and the \textit{down} quark, the former with $Q=+2/3$ and $T_3=+1/2$,
while the latter $Q=-1/3$, $T_3=-1/2$ and a slightly lower mass.
The next two generations are copies of the first one with increasing mass, with a pair of a
\textit{up}-type quark and a \textit{down}-type quark.
The second family consists in \textit{charm} and \textit{strange} quarks, and the third of \textit{top} and \textit{bottom} quarks.
In addition, all of the six quark flavours have antimatter states with the same mass, but opposite quantum numbers, as an example,
an anti-\textit{up}-type quark has $Q=-2/3$, $T_3=-1/2$ and can carry anti-\textit{red} colour.

Leptons are also similarly divided in six different types and in three separate generations named
\textit{electron} ($e$), \textit{muon} ($\mu$) and \textit{tau} ($\tau$), also with increasing mass.
Each generation contains a lepton with $Q=-1$ and $T_3=+1/2$ named after its generation,
and an associated electrically neutral lepton with $T_3=-1/2$ named neutrino ($\nu$).
The neutrino is assumed to be massless in the formulation of the \acrshort{SMlabel},
however the phenomena of neutrino oscillations is experimental proof of them actually having very small, but non-zero, mass values.
This apparent failure of the theory is discussed in \textcolor{red}{Section[]}. As before, the associated antimatter states have the
same mass but opposite quantum numbers.

All the stable \acrshort{SMlabel} matter in the universe is constituted by the massive particles of the first generations of quarks
and leptons, as the heavier versions eventually decay to lighter ones through their disclosed interactions.
While it is possible to observe free leptons, quarks exist only in bound states, or hadrons, like the neutron or the proton.
This is a feature of the strong interaction under the name of confinement, discussed in \textcolor{red}{Section[]}.
Only colour-less bounded states are observable then, and can be built from three quarks with overall half-spin, named baryons,
or by two quarks with integer spin, named mesons.

In the context of particle physics, the formulation of the classical Lagrangian, $\mathcal{L}$, is used to describe physics systems.
A generic free fermion field $\psi$ with mass $m$, can be described by the Dirac Lagrangian, 

\begin{equation}
\label{Theory_eq:diraceq}
    \mathcal{L} = \bar{\psi}(i\gamma^\mu\partial_\mu-m)\psi,
\end{equation}
where $\gamma^\mu$ are Dirac matrices and $\partial_\mu$ is the four-momentum derivative.

\subsubsection{Bosons}

Particles with integer spin are referred to as bosons. The bosonic sector with spin-1 gauge fields are force carriers that
naturally follow from imposing the requirement of local gauge invariance on Eq.~\ref{Theory_eq:diraceq} under symmetry groups,
in this case Eq.~\ref{Theory_eq:SMgroup}. \textcolor{red}{In Section[]} the nature and origin of the gauge bosons will be detailed.
In summary, the photon ($\gamma$) is the carrier of the electromagnetic force, being a massless and electrically neutral particle.
The weak force carriers are the $W^+$, $W^-$ and $Z$ bosons, all massive with the $Z$ boson being electrically neutral and the $W^\pm$ with either $Q=\pm1$.
Gluons ($g$) are the strong force carriers which are massless and with no electric charge.
Instead, there are eight different gluons representing each possible colour exchange. \textcolor{red}{Table~[]} presents a summary of the gauge bosons that mediate the different interactions.\\

The \acrshort{SMlabel} also includes a neutral spin-0 particle, or \textit{scalar}, the Higgs boson, with a mass of 125.25 $\pm$ 0.17 GeV~\cite{pdg}.
The Higgs field is responsible for all SM particles acquiring mass through the Higgs mechanism, as described in \textcolor{red}{Section[]}.
The kinematics of a generic scalar, $\phi$ with mass $m$, is described by the Klein-Gordon Lagrangian,

\begin{equation}
    \label{Theory_eq:KGeq}
    \mathcal{L}=\frac{1}{2}\partial^\mu\phi\partial_\mu\phi - m^2\phi^2
\end{equation}

Charged scalars can be described instead through a complex field and the expression of the Lagrangian is slightly modified,

\begin{equation}
    \label{Theory_eq:KGeqcharged}
    \mathcal{L}=\partial^\mu\phi\partial_\mu\phi^* - m^2\phi\phi^*
\end{equation}

Vector fields $A^\mu$, which represent spin-1 bosons, are described by the Proca Lagrangian,

\begin{equation}
    \label{Theory_eq:Proca}
    \mathcal{L}=-\frac{1}{4}F^{\mu\nu}F_{\mu\nu}+\frac{1}{2}m^2A^\mu A_\mu
\end{equation}
with $F^{\mu\nu}=\partial^\mu A_\nu -\partial^\nu A_\mu$ the field strength tensor.
In the case of massless particles, the previous expression with $m=0$ is known as the Maxwell Lagrangian.

\subsection{Interactions of the Standard Model}

The Lagrangian of the \acrshort{SMlabel} is defined to be locally invariant to the Eq.~\ref{Theory_eq:SMgroup} symmetry group,
condition that generates and defines the interactions of the corresponding particles as representations of the symmetry transformations.
For a generic Lagrangian, the physical system can have symmetries, so its Lagrangian is invariant under different kind of transformations.
These transformations can be time-space independent, called global transformations, or dependent, called gauge or local transformations.
Any invariant transformation of a Lagrangian describes a physical system which conserves a physical quantity,
as described by the Noether theorem~\cite{Noether}.
Then, the interactions are introduced in the Lagrangian as additional terms by promoting an already existing global symmetry, $\phi$,
of the Lagrangian to a local gauge symmetry, $\phi(x)$. The physical motivation behind introducing gauge symmetries
is to be able to describe vector bosons in \acrshort{QFT}, as seen in \textcolor{red}{Section~[]}.
The procedure expands the theory with additional fields that mediate the resulting interactions,
which properties depend on the characteristics of the symmetry group.

An example of the process is shown in the following, to afterwards derive the \acrshort{SMlabel} interactions of the strong and electroweak sectors. 

\subsubsection{Gauging a symmetry to interaction}

A general global transformation $\theta$ which acts on the field $\psi$ is described as,

\begin{equation}
    \psi\rightarrow e^{ig\theta^aT^a}\psi
\end{equation}

with $g$ the coupling constant and $T^a$ the generators of the Lie group associated to the transformation
(like $SU(n)$ or $U(n)$), with $a$ ranging from 1 to $n^2-1$, for the corresponding number of the Lie algebra, $n>1$.
The generators can be characterised by their commutation relation, 

\begin{equation}
\label{Theory_eq:nonabcomutator}
    [T^a,T^b]=if^{abc}T^c
\end{equation}

where $f^{abc}$ are the structure constants of the group. Following Noether's theorem, there are as many conserved quantities as
generators of the Lagrangian's symmetries. As an example, it is straightforward to see that a Lagrangian like
Equation~\ref{Theory_eq:diraceq} is invariant to a $U(1)$ transformation where $\theta$ is just a constant and hence, a constant phase change.
One can obtain the current, $j^\mu$, that is conserved, $\partial_\mu j^\mu = 0$,

\begin{equation}
    j^\mu = \bar{\psi}\gamma^\mu\psi
\end{equation}

and the conserved charge,
\begin{equation}
Q=\int \mathrm{d}^3x j^0 = \int \mathrm{d}^3x\psi^{\dag}\psi %= \int \frac{\mathrm{d}^3p}{(2\pi)^3}(a^\dag a - b^\dag b)
\end{equation}

%where the field has been expressed in an expansion over plane waves,

%\begin{equation}
%\psi(x)=\int\frac{\mathrm{d}^3k}{2\sqrt{8\pi^3(\textbf{k}^2+m^2)}}[a(\textbf{k})e^{-ikx}+b^\dag(k)e^{ikx}]
%\end{equation}
%with $a,b$ the Fock's space anihilation operators to the vacuum of two different particles. 

With some algebra and introducing solutions in momentum space, $\psi$ can be interpreted as annihilating a fermion and creating
an anti-fermion ($\psi^\dag$ the other way around) in the Fock space and then, this product becomes the difference
of the number of fermion and anti-fermion leading to the conservation of the fermion number.

Promoting the global symmetry to a local symmetry is done by introducing locality in the $\theta$ transformation,
$\theta \rightarrow\theta(x)$, which introduces new $\partial_\mu\theta$ terms in the Lagrangian.
A way to counter the new terms and, hence, keep the Lagrangian invariant, is to introduce gauge vector fields $A_\mu^a$,
following Yang-Mills theory~\cite{YangMills}. In the most generalised approach, there have to be as many $A_\mu^a$
as generators of the symmetry, that transform as,

\begin{equation}
    A_\mu^a \rightarrow A_\mu^a + \partial_\mu\theta^a + gf^{abc}A_\mu^b\theta^c
\end{equation}

Note that the last term proportional to the structure constant is relating the gauge field to the conserved symmetry charge.
The next step is to replace the standard derivative in the Lagrangian by the covariant derivative,

\begin{equation}
    D_\mu\equiv\partial_\mu - igT^aA_\mu^a
\end{equation}

The final ingredient is to complete the Lagrangian with the the kinematic Lagrangian for the massless vector fields,
the Maxwell Lagrangian from Equation~\ref{Theory_eq:Proca} with a slightly different field strength tensor,

\begin{equation}
    F_{\mu\nu}^a=\partial_\mu A_\nu^a - \partial_\nu A_\mu^a + gf^{abc}A_\mu^bA_\nu^c
\end{equation}

The last term is present only for non-abelian symmetry groups, since it is proportional to the structure constants,
and has huge consequences in the resulting interactions as discussed in \textcolor{red}{Section~[]}.
Another remark is that the gauge fields have to be massless, as a mass term proportional to $A_\mu^cA^{\mu c}$ is not gauge invariant.

As an example, the promotion of the global $U(1)$ symmetry seen in Equation~\ref{Theory_eq:diraceq} results in the upgraded Lagrangian,
\begin{equation}
\label{Theory_eq:diraceq20}
\begin{split}
    \mathcal{L}_{\ \ } &= \bar{\psi}(i\gamma^\mu D_\mu-m)\psi - \frac{1}{4}F_{\mu\nu}F^{\mu\nu} \\
    D_{\mu \ } &\equiv \partial_\mu -igA_\mu \\
    F_{\mu\nu} &\equiv \partial_\mu A_\nu - \partial_\nu A_\mu
\end{split}
\end{equation}
introducing just one massless gauge field that interacts with the field $\psi$.
The interaction term between the two fields is $g\bar{\psi}\gamma^\mu A_\mu\psi$,
hidden in the covariant derivative definition and proportional to the coupling constant $g$.

The Lagrangian of the \acrshort{SMlabel} is built from imposing local invariance under $SU(3)_C$ transformations,
which leads to strong interactions, and $SU(2)_L\times U(1)_Y$ transformations, which brings EW interactions,
\begin{equation}
    \mathcal{L}_{SM} = \mathcal{L}_{QCD}+\mathcal{L}_{EW}
\end{equation}

After this introduction on field theory, the theories of the two orthogonal sectors can now be described and then,
the mechanism to introduce mass terms in the Lagrangian, the spontaneous symmetry breaking.

\subsection{\acrlongpl{QCD}}

The quantum field theory that describes quarks and gluons interactions is named \textit{\acrlong{QCD}},
based on the $SU(3)_C$ symmetry group. Each quark has an internal degree of freedom, known as the colour charge, and it is defined by a triplet of fields,

\begin{equation}
\label{Theory_eq:colortriplet}
    q=\begin{pmatrix}
    q_{\mathrm{red}}\\
    q_{\mathrm{blue}}\\
    q_{\mathrm{green}}
    \end{pmatrix}
\end{equation}

where each of the components is a Dirac spinor associated to the corresponding colour state (red, blue and green). The colour
In addition, there are a total of six quarks, so the fields are labelled as $q_{f\alpha}$ with $f$ indicating the quark flavour
($f=u,d,c,s,t,b$) and $\alpha$ the colour. Note that there is an anti-quark of each flavour carrying an anti-colour charge.\\

The algebra of the $SU(3)$ group is characterised by the non-abelian commutation relation from Equation~\ref{Theory_eq:nonabcomutator}
with a total of eight generators, $T^a$. The generators can be written as $T^a=\lambda^a/2$ where $\lambda^a$
denote the Gell-Mann matrices~\cite{GellMann}.
Because of the eight generators, the interaction is mediated by a total of eight gauge bosons, called gluons $G_\mu^a$.
There are different matrix representation for the colour states of the gluons, following with the Gell-Mann matrices, taking,
%\renewcommand*{\arraystretch}{1.5}
\begin{equation}
\lambda^1 = \begin{pmatrix} 0 & 1 & 0 \\
1 & 0 & 0 \\
0 & 0 & 0 \end{pmatrix}
\end{equation}

and applying it to a general quark triplet like Equation~\ref{Theory_eq:colortriplet},
it can be seen that the transformation switches the red and blue charges.
To do so, the gluon has to carry a colour/anti-colour pair, to be able to "remove" the red charge ($r$) and "add" the blue charge ($b$),
and the other way around. There are nine possible combinations of colour/anti-colour pairs, which can be used to re-write the $\lambda^1$
transformation as,

\begin{equation}
\frac{r\bar{b}+b\bar{r}}{\sqrt{2}}
\end{equation}

known as the first state of the gluon colour octet. The rest of the states are equivalent to the other Gell-Mann matrices and all conserve
the three different colour flows.

The QCD Lagrangian can be obtained from modifying the the Dirac Lagrangian (Equation~\ref{Theory_eq:diraceq})
to achieve gauge invariance under $SU(3)_C$ transformations, following the definitions from \textcolor{red}{Section[]}.
The resulting Lagrangian is,

\begin{equation}
\label{Theory_eq:diraceq30}
\begin{split}
    \mathcal{L}_{QCD} &= i\sum_f \bar{q}_f\gamma^\mu D_\mu q_f - \frac{1}{4}G_{\mu\nu}^aG^{a\ \mu\nu} \\
    D_{\mu \ } &\equiv \partial_\mu -ig_sT^aG^a_\mu \\
    G_{\mu\nu}^a &\equiv \partial_\mu G_\nu^a - \partial_\nu G_\mu^a + g_s f^{abc}G_\mu^b G_\nu^c
\end{split}
\end{equation}

with $g_s$ being the strong force coupling constant and where the covariant derivative has been introduced
with the $G_\mu^a$ gluons fields, together with the kinematic term for the gluons, introducing the gluon tensor, $G_{\mu\nu}^a$.
As described in \textcolor{red}{Section[]}, gluons are massless because the term in the Lagrangian is not gauge invariant.
Notice that the masses of the quarks are also not present, not because it would break the symmetry, but for convention.
The masses in the \acrshort{SMlabel} come from the electro-weak sector.
Another remark is that the addition of a charge conjugation and parity symmetry (CP)
violating interaction term is allowed under local gauge invariance, but such an interaction has been experimentally observed
to be effectively zero\textcolor{red}{referencethestrongCPexperiment}.

The resulting interactions in the Lagrangian are shown in \textcolor{red}{Figure[]}, consisting of couplings between quarks and gluons\sidenote{Equivalent to the interaction obtained from the gauge $U(1)$ symmetry.}, and three- and four-point gluon self-interactions. As foreshadowed in \textcolor{red}{Section[]}, for non-abelian groups the gauge bosons have the self-interacting terms in the tensor.

There are two more important characteristics of this theory, that also arise from the non-abelian nature of the symmetry:
asymptotic freedom and confinement~\cite{PhysRevLett.30.1346,PhysRevLett.30.1343}. Asymptotic freedom refers to the fact that at very
high energies (in momentum transfer), or short distances, quarks and gluons interact weakly with each other allowing predictions
to be obtained using perturbation theory. Confinement is the name given to the impossibility of directly observing quarks,
only confined in hadrons, which are colorless composite states\sidenote{Color singlets are quantum states that are invariant under all eight generators of $SU(3)$, and therefore carry vanishing values of all colour conserved charges.}.
The idea is that for high distances, the strong coupling becomes larger, so when the distance between two quarks is increased, the energy of the gluon field is larger, up to the point to create from the vacuum a quark/anti-quark pair and thus forming a new hadron.

\subsubsection{Running coupling}
Say somewhere that we abuse group theory
?????????
introduce the beta function? just the coupling from griffits?

\subsection{\acrfullpl{EW} theory}

The quantum field theory that describes both the electromagnetic and weak interactions is named \textit{\acrlong{EW}} theory,
based on the $SU(2)_L\otimes U(1)_Y$ symmetry group\sidenote{$L$ refers to the left-handed chirality and $Y$ to the weak hypercharge}.
The product is non-abelian, like the $SU(3)_C$ group, and chiral. It will spawn four mediators, as the number of generators.
The symmetry spontaneously breaks down through \textit{\acrlongpl{EWSB}}, giving rise to the electromagnetic interaction,
mediated by the photon, and to the weak interaction, mediated by the $Z$ and $W^\pm$ bosons.
This process occurs at $\sim$100~GeV, defined as the \acrshort{EW} scale, and after which only the $U(1)_Q$ symmetry is unbroken,
described by \textit{\acrlongpl{EWSB}}~(\acrshort{QED}). The process of the \acrshort{EWSB}, and the resulting effects are described in
more detail in \textcolor{red}{Section}.\\

The interactions for the \acrshort{EW} sector can be obtained following the procedure described in general in \textcolor{red}{Section},
already used in \textcolor{red}{Section} for \acrshort{QCD}. First, only left-handed fermion fields interact via the weak
interaction\sidenote{As a consequence, parity can be violated in weak interactions~\cite{Lee,Wu}.},
transforming as doublets under $SU(2)_L$, whereas right-handed fermion fields do not interact weakly and thus transform as singlets,

\begin{equation}
\begin{split}
    \psi_L^i &= \begin{pmatrix}\ell^i_L\\ \nu^i_L \end{pmatrix}, \begin{pmatrix} u^i_L \\ d^i_L \end{pmatrix}\\
    \psi_R^i &= \ell^i_R, u^i_R, d^i_R
\end{split}
\end{equation}

with $i$ corresponding to the number of the generation. Fields with subscripts $L/R$ are left- and right-handed fields that can be defined
through the chirality operators $P_L$ and $P_R$, projecting a generic field into only its left- and right-handed components, respectively,

\begin{equation}
    \begin{split}
        \psi_L = P_L\psi = \frac{1}{2}(1-\gamma_5)\psi\\
        \psi_R = P_R\psi = \frac{1}{2}(1+\gamma_5)\psi
    \end{split}
\end{equation}

with $\gamma_5$ defined from the Dirac matrices $\gamma_5\equiv i\gamma^0\gamma^1\gamma^2\gamma^3$.\\

The $SU(2)_L$ group consists of three generators, $\hat{T}_i$, which can be written as $\hat{T}_i=\sigma_i/2$ where $\sigma_i$
denotes the Pauli matrices. Also, the quantum number associated is the weak isospin, $T$.
On the other side, the $U(1)_Y$ group introduces the weak hypercharge quantum number, $Y$. After \acrshort{EWSB},
the Gell-Mann-Nishijima equation relates $Y$ to the third component of the weak isospin operator, $T_3$ and the electric charge $Q$,

\begin{equation}
Q = Y+T_3
\end{equation}

Regarding the \acrshort{EW} Lagrangian, four gauge fields need to be introduced to achieve invariance under the
$SU(2)_L\otimes U(1)_Y$, $W_{\mu\nu}^i$ ($i$=1,2,3) from $SU(2)_L$, and $B_\mu$ from $U(1)_Y$. The resulting Lagrangian is,

\begin{equation}
\label{Theory_eq:EWlagrangian}
\begin{split}
    \mathcal{L}_{EW}&=i\sum_{f=l,q}\bar{f}(\gamma^\mu D_\mu)f - \frac{1}{4}W_{\mu\nu}^iW^{i\ \mu\nu} - \frac{1}{4}B_{\mu\nu}B^{\mu\nu}\\
    D_{\mu \ } &\equiv \partial_\mu - ig\frac{\sigma}{2}W_\mu^i-ig'YB_\mu \\
    W_{\mu\nu}^i &\equiv \partial_\mu W_\nu^i - \partial_\nu W_\mu^i +g\epsilon^{ijk}W_\mu^j W_\nu^k\\
    B_{\mu\nu}&\equiv\partial_\mu B_\nu - \partial_\nu B_\mu
    \end{split}
\end{equation}

with $\epsilon^{ijk}$ the Levi-Civita symbol, an antisymmetric tensor defined as $\epsilon^{ijk}\epsilon_{imn}=\delta^j_m\delta^k_n-\delta^j_n\delta^m_k$ with $i,j,k,l,m,n\in[1,2,3]$. Also, the $W_{\mu\nu}^i$ and $B_{\mu\nu}$ field tensors are defined to introduce the additional kinetic terms to the Lagrangian. The former contains a quadratic piece, due to the non-abelian nature of $SU(2)_L$, hence the full Lagrangian contains cubic and quartic self-interactions, as seen for the gluons in QCD. In contrast, the coupling constant $g$ increases rapidly with the energy scale.... As encountered before, mass terms for the gauge boson would break the gauge invariance. In this case, terms for the fermion masses would also break the symmetry as they would mix left- and right-handed fields, which transforms distinctively under $SU(2)_L$. Instead, the mass terms appear from the EWSB, described in Section[].

Summing all the interactions described, the SM Lagrangian for all the fermions before \acrshort{EWSB} becomes,

\begin{equation}
    \label{Theory_eq:SMbeforeEWSB}
    \begin{split}
    \mathcal{L}_{SM} &= \sum_f\sum_{\psi=L,e_R,Q_L,u_R,d_R} i\bar{\psi}^f\gamma^\mu D_\mu \psi^f\\
    &- \frac{1}{4}G^a_{\mu\nu}G^{a\_\mu\nu} - \frac{1}{4}W^i_{\mu\nu}W^{i\_\mu\nu} - \frac{1}{4}B_{\mu\nu}B^{\mu\nu}\\
    D_\mu &= \partial_\mu - i g_s T^a G^a_\mu - i g \frac{\sigma^i}{2}W_\mu^i - ig'YB_\mu 
    \end{split}
\end{equation}

no right handed neutrino, masses mix in EW

\subsection{Spontaneous symmetry breaking and the Higgs mechanism}

The model described so far cannot reproduce measured results, first of all the different fermions and the weak force mediators
have mass and second, the $SU(2)_L\times U(1)_Y$ symmetry is not preserved in nature.
Even if somehow the \acrshort{EW} gauge bosons are allowed to have mass, it leads to the lack of renormalisability and the violation of
unitarity. Renormalisation is a collection of techniques that allows the computation of measurable observables in \acrshort{QFT},
managing the different sources of infinities within the theory like those from self-interactions.
Unitarity is needed more in general in quantum mechanics, to ensure proper time-evolution predictions of
a quantum state. The longitudinal component of the massive boson is the cause of the problem,
as in a boosted frame in which $p^\mu=(p^0,0,0,|\textbf{p}|)$, the parallel polarisation component of a massive boson is
$\epsilon_\mu=(|\textbf{p}|/m,0,0,p^0)$, growing indefinitely with the energy of the system.
When computing the cross-section of the corresponding boson scattering, the value will indefinitely grow breaking the mentioned unitarity.
If computed explicitly for the $W^\pm$ bosons, the energy scale where this happens is around the TeV scale,
pointing to a fundamental problem in the theory to describe that scale.

The solution is provided by the \acrshort{EWSB} and the Higgs-Englert-Brout mechanism, discussed next, after showing the spontaneous
symmetry breaking process for a simple gauge theory.

\subsubsection{How to break a symmetry}

Spontaneous symmetry breaking is a phenomenon where a symmetry of the theory is unstable and the vacuum, or fundamental state, is degenerate.
In the process, new interactions appear and a field obtains a non-zero vacuum expectation value.

The topic is broad as there are many symmetries and representations to potentially break, to illustrate the mechanism for the \acrshort{SMlabel},
lets consider a system with a scalar field $\phi$, a gauge field $A_\mu$, and the following Lagrangian with a gauge symmetry,

\begin{equation}
    \begin{split}
    \mathcal{L}_{\ \ }&=(D^\mu\phi)^\dag D_\mu\phi - V(\phi) - \frac{1}{4}F_{\mu\nu}F^{\mu\nu}\\
    D_{\mu \ } &\equiv \partial_\mu - igA_\mu\\
    F_{\mu\nu}&\equiv\partial_\mu A_\nu - \partial_\nu A_\mu
    \end{split}
\end{equation}

with a general potential $V(\phi)$ given by,

\begin{equation}
    V(\phi) = \frac{1}{2}\mu^2\phi^\dag\phi + \frac{1}{4}\lambda(\phi^\dag\phi)^2
\end{equation}

with the real parameters $\mu^2$ and $\lambda$ relating respectively to the mass term and the strength of the self-interaction.
\textcolor{red}{[INSERT FIGURE]}
There are two sensible ranges for these parameters, the first one is the case $\lambda,\mu^2>0$,
similar to the previous seen theories and only one solution in the minimisation.
The second one is for $\lambda>0$ and $\mu^2<0$, where the $\mu^2\phi^\dag\phi$ term cannot be understood as a mass term and
the solution $\phi=0$ is a local maximum, physically unstable. The minimum of the potential is degenerate and identified by the complex
plane circle, $\phi^\dag\phi=v^2/2$ with $v^2\equiv-\mu^2/\lambda$ and

\begin{equation}
    \phi = v e^{-i\theta}
\end{equation}

The symmetry is broken spontaneously when the system choses the fundamental state. Suppose $\phi=0$, then the \textit{\acrlongpl{VEV}}~(\acrshort{VEV})
of $\phi$ is set to,

\begin{equation}
    \expval{\phi}{0} = \frac{v}{\sqrt{2}}
\end{equation}

Next, lets suppose the following change of variables to center the new fundamental state,

\begin{equation}
    \phi(x)=\left( \frac{v+\eta(x)}{\sqrt{2}} \right) e^{i\zeta(x)/v}
\end{equation}

the Lagrangian can be expressed as,

\begin{equation}
\begin{split}
\mathcal{L} &= \frac{1}{2}(\partial_\mu\eta)^2 + \frac{1}{2}(\partial_\mu\zeta)^2 - \frac{1}{4}F_{\mu\nu}F^{\mu\nu}\\
&+\mu^2\eta^2 + \frac{1}{2}g^2v^2A_\mu A^\mu - gv A\mu \partial^\mu\zeta +\ \text{interactions}
\end{split}
\end{equation}

which now contains the $\eta$ and $\zeta$ fields, additional to the gauge $A_\mu$. Also, square terms appear for $\eta$ and $A_\mu$,
which can be identified as mass terms, $\frac{m_\eta}{2}\eta^2$ and $\frac{m_A}{2}A_\mu A^\mu$,
resulting in $m_\eta=\sqrt{-2\mu^2}$ and $m_A=gv$. $\zeta(x)$ is massless and a particular resulting type of field named
\textit{Goldstone boson}, which the \textit{Goldstone theorem} predicts.
The theorem states that a massless boson appears for every symmetry that the \acrshort{VEV} spontaneously breaks.
In this abelian case, the \acrshort{VEV} is not invariant under the $U(1)$ transformation.
$\zeta(x)$ does not appear explicitly in the potential, therefore can take any value without affecting the energy of the system,
which is not very physical. In addition, it appears in an estrange mixing term with $A_\mu$, $-gvA_\mu\partial^\mu\zeta$.
A way to remove this annoyance is to choose the gauge,

\begin{equation}
\begin{split}
    &\phi\rightarrow\phi'=e^{-i\zeta/v}\phi \\
    &A_\mu\rightarrow A'_\mu = A_\mu-\frac{1}{gv}\partial_\mu\zeta
\end{split}
\end{equation}
together with the previous change of variable for $\phi$. Essentially the gauge freedom of the Lagrangian is being used to remove $\zeta$,
which becomes the longitudinal component of the transformed gauge boson $A_\mu$.
The gauge chosen is the so-called \textit{unitary gauge}, which makes the physical content of the Lagrangian explicit.

In summary, this process of acquiring mass by means of absorbing a Goldstone boson is known as the \textit{Higgs mechanism}.

\subsubsection{The Higgs-Englert-Brout Mechanism in the Electroweak Sector}

The Higgs-Englert-Brout mechanism~\cite{Higgs1,Higgs2,Englert} solved the contradictions found between massive particles and
the requirement of gauge invariance. The mechanism is based in a spontaneous symmetry breaking of the $SU(2)_L\otimes U(1)_Y$ to $U(1)_{EM}$,
giving mass to the different particles involved in the \acrshort{EW} interactions except the photon.
A similar procedure can be applied to the EW Lagrangian derived in Equation~\ref{Theory_eq:SMbeforeEWSB},
first introducing an isospin doublet ($Y$=+1/2) of complex scalar fields $\Phi$, the Higgs field,

\begin{equation}
    \Phi\equiv
    \begin{pmatrix} \phi^+ \\ \phi^0 \end{pmatrix}
    =\frac{1}{\sqrt{2}}
    \begin{pmatrix} \phi_1 + i\phi_2 \\ \phi_3 + i\phi_4 \end{pmatrix}
\end{equation}

where $\phi^+$ corresponds to an electrically charged field ($T_3$=+1/2) and $\phi^0$ to a neutral one ($T_3$=-1/2).
This field transforms under $SU(2)_L$ and its Lagrangian, the Higgs Lagrangian,

\begin{equation}
    \mathcal{L}_\Phi = (D_\mu\Phi)^\dag(D^\mu\Phi)-V(\Phi)
\end{equation}

with the same covariant derivative as in Equation~\ref{Theory_eq:SMbeforeEWSB} and the Higgs potential given by,

\begin{equation}
    V(\Phi) = \mu^2\Phi^\dag\Phi+\lambda(\Phi^\dag\Phi)^2
\end{equation}

which shape depends on the parameters $\mu^2$ and $\lambda$. As seen before, choosing the case where $\lambda>0$ and $\mu^2<0$,
the potential at $\Phi=0$ is unstable and a continuous collection of possible minimum values appear, defined by the circle,

\begin{equation}
    \Phi^\dag\Phi=\frac{1}{2}\frac{-\mu^2}{\lambda}\equiv\frac{1}{2}v^2
\end{equation}

Following, the spontaneous symmetry breaking with the choice of the new vacuum state,

\begin{equation}
    \expval{\Phi}{0} =\frac{1}{\sqrt{2}}\begin{pmatrix}
    0\\v
    \end{pmatrix}
\end{equation}

This vacuum is not invariant to any of the $SU(2)_L$ and the $U(1)$ transformations, however, the $Q=T_3+Y$ transformation is not affected,

\begin{equation}
    Q\expval{\Phi}{0} = \frac{1}{2\sqrt{2}}\sigma_3\begin{pmatrix}0\\v\end{pmatrix}+\frac{1}{2\sqrt{2}}Y\begin{pmatrix}0\\v\end{pmatrix} = \frac{1}{2\sqrt{2}}\left[ 
    \begin{pmatrix} 0 \\ -v  \end{pmatrix} +
    \begin{pmatrix} 0 \\ v  \end{pmatrix}\right] = \begin{pmatrix}
    0\\0
    \end{pmatrix}
\end{equation}

The field is rewritten in the unitary gauge, which automatically removes the extra nonphysical Goldstone bosons,

\begin{equation}
    \Phi(x) = \frac{1}{\sqrt{2}}\begin{pmatrix}
    0 \\ v+H(x)
    \end{pmatrix}
\end{equation}

where $H(x)$ is centered around the vacuum state. With this change the Higgs potential becomes,

\begin{equation}
    V(\Phi) =\frac{1}{4}\lambda v^2 H^2 + \frac{1}{4} \lambda v H^3 + \frac{1}{16} \lambda H^4
\end{equation}

spawning the Higgs boson mass $m_H^2=\lambda v^2/2 = -\mu^2/2$, in the quadratic $H$ term.
The cubic and quartic terms constitute the three- and four-point Higgs boson self-interactions.

The \acrshort{EWSB} generates new interactions and mass terms for the different particles involved in the \acrshort{EW} interactions.
Gluons are not affected as the scalar field is a doublet and does not transform under $SU(3)$.
The effects on the boson and fermion sectors of the \acrshort{SMlabel} are discussed in the following, individually. 

\subsubsection{Boson sector}

The gauge boson masses spawn from the covariant derivative, $(D_\mu\Phi)^\dag(D^\mu\Phi)$, which includes the gauge fields.
Expanding,

\begin{equation}
\label{Theory_eq:Lgaugemass}
\mathcal{L}_{mass} = \frac{v^2}{8}
V_\mu
\begin{pmatrix} \begin{matrix} g^2 & 0 \\ 0 & g^2 \end{matrix} & 0_{2\times2} \\ 0_{2\times2} & \begin{matrix} g^2 & -gg' \\ -gg' & g'^2 \end{matrix}
\end{pmatrix} V^\mu 
\end{equation}

with $V_\mu = \begin{pmatrix} W_\mu^1 & W_\mu^2 & W_\mu^3 & B_\mu
\end{pmatrix}$. Diagonalising the matrix, the next eigenvectors are found,

\begin{equation}
\begin{split}
    A_\mu &\equiv \sin\theta_W W_\mu^3 + \cos\theta_WB_\mu \\
    Z_\mu &\equiv \cos\theta_W W_\mu^3 - \sin\theta_WB_\mu  
\end{split}
\end{equation}

where the Weinberg angle, or weak mixing angle, is defined by $\tan\theta_W\equiv g'/g$. The corresponding eigenvalues, the square masses, for the $A_\mu$ and $Z_\mu$ fields are zero and $v^2(g^2+g'^2)/8$. On the other side, $W_\mu^1$ and $W_\mu^2$ are well defined mass states but not charge states. This is due $T_1$ and $T_2$ being not diagonal, connecting the different states of $T_3$ (hence of $Q$). The operator $T_\pm=T_1\mp iT_2$ can be defined, which increases or decreases one unit of $T_3$ (hence of $Q$). In addition, the fields can be redefined,

\begin{equation}
    W_\mu^\pm = \frac{1}{\sqrt{2}}(W_\mu^1\mp i W_\mu^2)
\end{equation}

In summary the Lagrangian in Equation~\ref{Theory_eq:Lgaugemass} can now be written as

\begin{equation}
    \mathcal{L}_{mass} = \frac{g^2v^2}{4}W_\mu^+W^{- \mu} - \frac{v^2}{8}(g^2+g'^2)Z_\mu Z^\mu
\end{equation}

where the mass terms of the different bosons can be identified,

\begin{equation}
\begin{split}
    &m_A = 0\\
    &m_Z = \frac{v}{2}\sqrt{g^2+g'^2}\\
    &m_W = \frac{vg}{2} = m_Z \cos\theta_W
\end{split}
\end{equation}

Note that the remaining symmetry after breaking $SU(2)_L\otimes U(1)_L$ is $U(1)_{EM}$. The associated $A_\mu$ field is massless, the photon, which is a combination of the $W_\mu^3$ and $B_\mu$ fields. The associated the quantum number, the electric charge, has been defined previously in the chapter, $Q = T_3-Y$.\\

Regarding interactions, the covariant derivative can be expressed in terms of the new bosons,

\begin{equation}
    \partial_\mu - igW_\mu^3 = \partial_\mu - ig\sin\theta_W A_\mu - ig\cos\theta_W Z_\mu
\end{equation}

where the electromagnetic coupling constant $e$ can be defined as $e=g\sin\theta_W$. In addition, the field tensors can be rewritten as,

\begin{equation}
\begin{split}
    W_{\mu\nu}^3 &= \partial_\mu W_\nu^3 - \partial_\nu W_\mu^3 - ig(W_\mu^+W_\nu^- - W_\nu^+ W_\mu^-)\\
    &= \sin\theta_W F_{\mu\nu} + \cos\theta_W Z_{\mu\nu} - ig(W_\mu^+W_\nu^- - W_\nu^+ W_\mu^-)\\
    B_{\mu\nu} &= \cos\theta_W F_{\mu\nu} - \sin\theta_W Z_{\mu\nu}
\end{split}
\end{equation}

where the field strength tensors for the photons and the Z boson, $F_{\mu\nu}$ and $Z_{\mu\nu}$ are defined.

\subsubsection{Fermion sector}

The procedure required to acquire the fermion masses is more complicated than for the gauge bosons. Instead of just expanding the kinematic term with the new Higgs field, Yukawa~\cite{yukawa} interactions that couple left- and right-handed fermions with the Higgs need to be introduced.\\

As seen in this chapter, only $q_{\alpha L}^i$ and $l^i_L$ fields are $SU(2)_L$ doublets,

\begin{equation}
    \label{Theory_eq:SUdoublets}
    q_{\alpha L}^i=\begin{pmatrix} u^i_{\alpha L} \\ d^i_{\alpha L} \end{pmatrix},\ l_L^i = \begin{pmatrix} \nu^i_L \\ \ell^i_L \end{pmatrix}
\end{equation}

where the $i$ refers to the generation and $\alpha$ to the colour. It has been already pointed out that is not possible to construct a well defined $mf^\dag f$ term that transforms under the SM group, necessary for gauge invariance.\\

The solution is provided by introducing Yukawa interactions between the fermion fields and the Higgs field $\Phi$, also a doublet under $SU(2)$,

\begin{equation}
\begin{split}
    &\mathcal{L}_{Yukawa} = -y^{ab}\bar{q}^a_{\alpha\ L}\Phi d^b_{\alpha\ R} - y'^{ab}\bar{q}^a_{\alpha\ L}\tilde{\Phi} u^b_{\alpha\ R}-y''^{ab}\bar{l}^a_{L}\Phi \ell^b_{R}+\ \text{h.c}\\
    %&-\epsilon^{ij}\Phi_i (q_{\alpha\ L})^a_j y^{ab} (\bar{d}_{\alpha\ L})^b_{\alpha}
    %-\epsilon^{ij}\Phi_i (q_{\alpha\ L})^a_j y^{ab} (\bar{d}_{\alpha\ L})^b_{\alpha}
\end{split}
\end{equation}

where y, y' and y'' are the Yukawa matrices, $3\times3$ matrices with one dimension for each generation. Also, $\tilde{\Phi}\equiv i\sigma_2\Phi^*$. Note that there is no second term for the leptons, as the SM does not contemplate the right handed neutrino, $\nu_R$. Also, this Lagrangian breaks explicitly the chiral symmetry but yields a singlet representation, safe for gauge invariance. Next, writing the field $\Phi$ in terms of the unitary gauge as in the EWSB, $\phi^0(x)=v+H(x)$,

\begin{equation}
\begin{split}
    \mathcal{L}_{Yukawa} &= -\frac{1}{\sqrt{2}}(v+H)y^{ab}\bar{q}^a_{\alpha\ L} d^b_{\alpha\ R} - \frac{1}{\sqrt{2}}(v+H)y'^{ab}\bar{q}^a_{\alpha\ L}u^b_{\alpha\ R}\\
    &-\frac{1}{\sqrt{2}}(v+H)y''^{ab}\bar{l}^a_{L}\ell^b_{R}+\ \text{h.c}\\
    &=-\frac{1}{\sqrt{2}}(v+H)y^{ab} \bar{D}^a_\alpha D^b_\alpha - \frac{1}{\sqrt{2}}(v+H)y'^{ab}\bar{U}^a_\alpha U^b_\alpha\\
    &-\frac{1}{\sqrt{2}}(v+H)y''^{ab}\bar{L}^a L^b+\ \text{h.c}
\end{split}
\end{equation}

where the expression has been rearranged to define Dirac fields in spinor notation,

\begin{equation}
\label{Theory_eq:Diracmassspace}
    D_\alpha^a = \begin{pmatrix} d_\alpha^a \\ \bar{d}^{\dag a}_\alpha \end{pmatrix},\ 
    U_\alpha^a = \begin{pmatrix} u_\alpha^a \\ \bar{u}^{\dag a}_\alpha \end{pmatrix},\ 
    L_\alpha^a = \begin{pmatrix} \ell_\alpha^a \\ \bar{\ell}^{\dag a}_\alpha \end{pmatrix}
\end{equation}

After diagonalising the three Yukawa matrices, the eigenvalues terms are related to the masses, which can be identified for each generation as,

\begin{equation}
\begin{split}
m_{d^i} = y^{ii}v/\sqrt{2} \\ 
m_{u^i} = y'^{ii}v/\sqrt{2} \\
m_{\ell^i} = y''^{ii}v/\sqrt{2} \\
m_{\nu^i} = 0
\end{split}
\end{equation}

There is a major consequence from the differences between the representation in generation space (Equation~\ref{Theory_eq:SUdoublets}, $SU(2)_L$ doublets), and in mass space, after diagonalising the Yukawa matrices. $D^a_{\alpha}$ and $U^a_{\alpha}$ rotated to diagonalise their corresponding Yukawa matrix are affected by different transformations, however the individual $d^a_{\alpha\ L}$ and $u^a_{\alpha\ L}$ fields are part of the same $SU(2)_L$ doublet. The effect can be seen writing the $W^\pm$ interactions in the mass state representation of the fields which become off-diagonal,

\begin{equation}
    \frac{-g}{\sqrt{2}} \begin{pmatrix} \bar{u}_L & \bar{c}_L & \bar{t}_L \end{pmatrix} \gamma^\mu W_\mu^+ V_{CKM} \begin{pmatrix} d_L \\ s_L \\ b_L \end{pmatrix} +\ \text{h.c}
\end{equation}
\begin{equation}
\begin{pmatrix} d'_L \\ s'_L \\ b'_L \end{pmatrix} = V_{CKM}\begin{pmatrix} d_L \\ s_L \\ b_L \end{pmatrix} =\begin{pmatrix} V_{ud} & V_{us} & V_{ub} \\ V_{cd} & V_{cs} & V_{cb} \\ V_{td} & V_{ts} & V_{tb} \end{pmatrix} \begin{pmatrix} d_L \\ s_L \\ b_L \end{pmatrix}
\end{equation}

where the superscript ' denotes the mass representation and $V_{CKM}$ is the Cabibbo-Kobayashi-Maskawa matrix~\cite{Cabibbo,KobayaMaska}. This unitary matrix is the product of the transformations that diagonalise the y and y' Yukawa matrices, which encodes the mixing of the different generations of fields in charged-mediated weak interactions. This is known as flavour violation, where a weak interaction of a quark can result on changing its flavour\sidenote{Maybe add a feynman diagram Rafel notes}. The neutral current interactions, mediated by the $Z$ boson, relate the fields with the same charge, affected by the same transformation, hence not spawning a mixing matrix. The reason for not having \textit{Flavour Changing Neutral Currents} (FCNC) explicitly in the SM Lagrangian. On the other side, the leptons are represented with the same $SU(2)_L$ doublet, so any mixing of lepton generations is not present in the theory.

There is still another interesting feature that arises from the CKM matrix. The standard representation~\cite{Ling-Lie} of the matrix takes into account invariant phase rotations of the fields, leaving as free parameters three angles $\theta_{12}$, $\theta_{23}$ and $\theta_{13}$ (chosen to lie in the first quadrant so $\sin\theta,\cos\theta\geq0$), and a single complex phase $\delta$ that cannot be rotated to zero. The matrix reads,

\begin{equation}
\begin{split}
    V_{CKM} &= \begin{pmatrix} 1 & 0 & 0 \\ 0 & c_{23} & s_{23} \\ 0 & -s_{23} & c_{23} \end{pmatrix} \begin{pmatrix} c_{13} & 0 & s_{13}e^{-i\delta} \\ 0 & 1 & 0 \\ -s_{13}e^{i\delta} & 0 & c_{13} \end{pmatrix}
    \begin{pmatrix} c_{12} & s_{12} & 0 \\ -s_{12} & c_{12} & 0 \\ 0 & 0 & 1 \end{pmatrix} \\ 
    &= \begin{pmatrix} c_{12}c_{13} & s_{12}c_{13} & s_{13}e^{-i\delta} \\ -s_{12}c_{23}-c_{12}s_{23}s_{13}e^{i\delta} & c_{12}c_{23}-s_{12}s_{23}s_{13}e^{i\delta} & s_{23}c_{13} \\ s_{12}s_{23}-c_{12}c_{23}s_{13}e^{i\delta} & -c_{12}s_{23}-s_{12}c_{23}s_{13}e^{i\delta} & c_{23}c_{13} \end{pmatrix}
\end{split}
\end{equation}
where $s_{ij}=\sin\theta_{ij}$ and $c_{ij}=\cos\theta_{ij}$. The presence of the complex phase leads to different couplings for anti-matter, as the complex phase will switch sign, leading to matter/anti-matter asymmetry. This asymmetry in flavour-changing processes is the only source in the SM of \textit{CP} violation, or \textit{T} violation (from the time-reversal symmetry\sidenote{The three symmetries are related as the combination, $CPT$ symmetry, must always be respected in theory. }) however, as discussed in Section[], fails to describe the current matter/anti-matter content of the universe. The CKM matrix is predicted and measured to be almost diagonal, with very small sources of CP violation, or $V_{ub}$ and $V_{td}$. The current matrix as in 2022~\cite{pdg} reads, 
\begin{equation}
        V_{CKM}= \begin{pmatrix} 0.97401 \pm 0.00011 & 0.22650 \pm 0.00048 & 0.00361^{+0.00011}_{-0.00009} \\ 0.22636 \pm 0.00048 & 0.97320 \pm 0.00011 & 0.04053^{+0.00083}_{-0.00061} \\ 0.00854^{+0.00023}_{-0.00016} & 0.03978^{+0.00082}_{-0.00060} & 0.999172^{+0.000024}_{-0.000035} \end{pmatrix}
\end{equation}

\section{Successes and shortcomings of the Standard Model}

Since the formulation of the SM, most experimental observations and measurements have been described successfully by the model. Throughout the years, predicted particles have been found and multiple precision measurements have testes its validity. However, there are theoretical and experimental issues not solved by the theory, leading to the conclusion that the SM is an effective theory and there is a more complete theory that can explain the whole range of observations. In this section, a brief summary of the measurements of the SM parameters is presented, followed by an overview of the main open questions. 

\subsection{Experimental measurements}

Decades of experiments have performed measurements into parameters that define the SM. The SM can be summarised with nineteen parameters, which have been described in this chapter: nine fermion masses (six for quarksm three for leptons), the three gauge couplings ($g_S$, $g$ and $g'$), the Higgs vacuum expectation value ($v$), the Higgs mass, four parameters of the CKM matrix (three angles and the complex phase), the QCD CP violating phase\sidenote{This has not been described}. There is no underlying relation between these parameters, only being set from experimental observations. With these parameters measured, theoretical predictions of observables can be tested with experimental data in order to explore new physics.\\

One typical observable in particle physics is the cross-section $\sigma$, the expected interaction rate between two interacting particles in terms of the effective surface area measured in \textit{pb} (picobarn, 1pb = 10$^{-40}$ m$^2$). The cross-section of a process depends on the interacting forces involved, as well as the energy and momentum of the interacting particles, which can be calculated from the S-matrix (scattering matrix) using relativistic mechanics. Feynman diagrams are a tool to translate a visual description of a process to a mathematical expression, the matrix amplitude, which is proportional to the probability of the specific process happening and needed for the computation. The decay width, $\Gamma$, can be computed in similar fashion to obtain another common observable, the Branching Ratio (BR). The BR of an unstable particle is the probability for it to decay into specific particles among all possible states. It is computed dividing the $\Gamma$ of the specific process with respect to the sum of all the possible process. Both $\sigma$ and $\Gamma$ are calculated from perturbation approximations, as the actual process is not the product of just one Feynman diagram, but all the possible interactions that lead to the same final state including loops, interferences and radiative corrections, refereed to as high order corrections. However, each particle interaction is proportional to the probability making higher order corrections become less important. Typically, \textit{leading-order} (LO) calculations use only the leading order terms from the perturbation expansion, while if complemented by higher order corrections are referred to next-to-leading-order (NLO) or next-to-NLO (NNLO) calculations.\\

Figure[] shows a summary of a wide range of cross-section measurements by the ATLAS Collaboration, compared to the theoretical predictions, showing an excellent agreement between data and theory. In addition, the Higgs boson has been scrutinised since its discovery in 2012 [refs], both to characterise all its properties and because the uniqueness of its interaction with massive particles. Figure[] shows a summary of Higgs boson production cross-sections and measurements, including the coupling strengths to other SM particles, showing that the coupling is proportional with the mass of the resulting particle as expected from the Higgs mechanism. As the Higgs couples with any particle that acquires mass thought its field, it is an excellent candidate to study any other particle still to be discovered.\\

On another note, the top quark is the most massive known particle known to date since its discovery in 1995. Such characteristic makes the top quark the only one that decays before hadronisation and an excellent candidate to study new particles, from much more massive ones that might decay to the top quark to lighter exotic particles that might decay into. Figure[]crossection?. The top quark is also used to measure possible FCNC processes, for which there is no direct coupling in the SM as explained in Section[]. Nevertheless, higher order processes involving $W^\pm$ are possible making FCNC processes very unlikely. As the top quark can decay to any other SM particle, observation of FCNC processes would imply the need to expand the SM.Table with measuremetns\\

Interest in top physics

\subsection{Open questions}

Observed neutrino oscillations~\cite{neutrinoosc} are only possible if there are mass differences between the three neutrino generations, which implies non-zero masses for some neutrinos altought not measured directly. A mass term for neutrinos could be added in the SM in different ways, through adding right-handed neutrinos or as describing neutrinos as Majorana particles [ref]. Nevertheless, the description of the SM particles needs at least seven additional parameters: three for the neutrino masses, three for their mixing angles and one CP violating phase for the neutrino mixing Pontecorvo-Maki-Nakagawa-Sakata (PMNS) matrix~\cite{}, similar to the CKM quark flavor matrix.

The SM also fails to describe the other known fundamental force in nature, gravity. There is no renormalisable quantum field theory for gravity successfully described as general relativity only describes macroscopic systems, with the observation of gravitational waves as the latest achievement~\cite{}. There are theories like string theory that provide alternatives although difficult to test experimentally. The SM is understood as an effective theory of a more complete unified theory, hence only valid at low energies as it breaks in the most extreme scenario around the Plank scale ($M_P=\sqrt{\bar{h}/(8\pi G_N)}\sim 2.4\ 10^{18}$~GeV), where gravitational effects are supposed to become as important as the other forces in the SM. 

The SM describes what is known as baryonic matter, which accounts for about 5\% of the energy density of the universe. Cosmology, which studies the composition of the universe, estimates huge amounts of \textit{dark matter} (DM) and \textit{dark energy}, phenomena not contemplated by the SM. The existence of DM was postulated as extra non-luminous matter needed to explain observed rotation curves of galaxies not matching the gravitational pull of observed stars~\cite{}. In addition, gravitational lensing effects observed in some galaxy collisions~\cite{} also provide the need of huge invisible mass concretations. More recently, the WMAP and Planck collaborations have studied anisotropies in the cosmic microwave background (CMB)~\cite{}, postulating cold DM. On the other side, the universe is observed to be expanding at an accelerated rate compatible with dark energy, understood to be the product of an intrinsic space-time energy density, or cosmological constant, that causes the expansion. Observation of red-shift of light from supernovae, used as standard candles, points that cosmological objects are moving away at faster rate with the distance~\cite{}. Further, studies of the CMB provide another measurements of the accelerated expansion~\cite{}. The most updated results~\cite{} points that baryonic matter accounts for a mere \% of the total energy density of the universe, dark matter for \%, while dark energy for \%.\\

The universe seems to be completely made up by matter. To explain the imbalance in abundance of matter and anti-matter, often referred as matter/anti-matter assymetry, the SM only provides one not nearly sufficient source of CP violation in the quark weak interactions, as mentioned in Section[]. Additional sources are added such as the complex phase in the PMNS matrix, however more phenomena is needed to have generated the current net balance of matter, like possible baryon number violating effects at high energy scales.\\

Besides the natural phenomena uncovered by the SM, there are also what are known as naturalness problems. Those are aesthetic concerns about the precise different values of some of the SM parameters, which seem "unnatural" if there is no hidden mechanism behind. The general consensus is that the fewer fine tuning is needed in a theory, the more natural it is. Although these matters are completely subjective, these unexplained features could be a hint for the existence of a new underlying mechanism that could complement the SM. The first problem is commonly named as the hierarchy problem, where the cutoff energy of the SM ($\Lambda_{\text{SM}}$) is commonly set to the Planck scale, $\sim10^{18}$~GeV, but in contrast the EW scale ($v\sim246$~GeV) is very small. The problem can be read as there is no apparent reason for the EWSB to happen at its scale, orders of magnitude smaller than the Plank scale. When calculating high-order corrections from the SM like the computation of loops, $\Lambda_{\text{SM}}$ has to be introduced to manage ill-defined divergent integrals in a process called regularisation~\cite{}. As particles are endlessly interacting with virtual particles, represented by self-energy diagrams like Figure[], their mass are effectively diverging. This is solved by interpreting the physical mass measured in experiments as the resulting after dealing with the divergent terms. Leading radiation corrections for the fermion masses are of the order of $\log\Lambda_{\text{SM}}$, sensitive to the scale but the fine-tuning is considered small. On the other side, the physical Higgs mass including radiation corrections reads,

\begin{equation}
    m^2_H = m_0^2 + \frac{3}{8\pi^2v^2}\Lambda_{\text{SM}}^2 [ m_0^2 + 2m_W^2 + M_Z^2 - 4m_t^2] + \mathcal{O}(\ln\frac{\Lambda_{\text{SM}}}{m_0})
\end{equation}

with $m_0$ the bare Higgs mass. The nature of the hierarchy problem is evident in the correction as the Higgs mass is more sensitive to the cutoff scale and requires huge tuning to counter the $\Lambda_{\text{SM}}$ term and achieve such a low measured physical mass. It can also be observed that the most important correction is given by the top quark, and it is often questioned whether the reason for the huge mass of this quark could hide a solution. Although the Higgs mass and the EW scale are difficult to justify, it can be argued that the appearance of the $\Lambda_{\text{SM}}$ is related to the chosen regularisation scheme and cut-offs play no physical role. 
Some Feyman diagrmas?

Another related problem is the fermion mass hierarchy, as the fact that the masses of the SM particles range from $\sim1$~MeV to $\sim$173~GeV (of the top quark~\cite{pdg}), it is not understood. Additionally, there is also not an apparent reason for the three mass families of quarks and leptons. It might be related again to renormalisation, since fermion masses also have correction terms with the logarithm of the cut-off scale as stated before.\\

There is also the problem known as the strong CP problem. The most general QCD Lagrangian could include a CP-violating angle without breaking any symmetry or the renormalisability of the theory. This would lead to the prediction of axion particles and the neutron having non-zero electric dipole moment. Measures of the former in ultracold neutrons and mercury, constrain the CP-violating angle to $\abs{\theta}\lesssim10^{-10}$~\cite{}, and supposes an incredibly low value for a parameter that can have any value in the theory.

\section{Beyond the Standard Model}


Where to look? SM right?

Statistical tools?

\chapter{The ATLAS experiment at the LHC}

Introduction


\section{CERN and the Large Hadron Collider}

The Large Hadron Collider (LHC) is a circular particle accelerator, consisting of 27 km of circumference tunnel built by the European Organisation for Nuclear Research (CERN), situated underground in the France-Swiss boarder. This tunnel contain two parallel pipes, with more than 1000 superconducting magnets and accelerating structures, where two beams of protons travel at the same time in opposite directions to collide in four different particle detectors built around the ring to record and study the interactions, as shown in the (nom de la figura).\\

A Large Ion Collider Experiment (ALICE)  (explicacio d'alice)

A Toroidal LHC ApparatuS (ATLAS)

Compact Muon Solenoid (CMS)

LHC-beauty (LHCb)



\chapter{Physics simulation of proton collisions}

Proton collisions are complex processes and their understanding is essential to interpret the experimental data from the \acrshort{LHClabel}. Normally, physics analyses rely on the ability to accurately simulate the various processes of proton-proton collisions and the interactions with the detector in order to perform comparisons with the recorded data and quantify its the level of agreement of the SM. The simulation is usually performed with Monte Carlo \acrlong{MClabel}~(\acrshort{MClabel}) generators, which are stochastic tools that incorporate both theoretical predictions and empirical results to describe the statistical processes.
Sections bla and bla 

\section{Event simulation}

The typical proton-proton collision at the LHC is depicted in \textcolor{red}{Figure}. The inelastic scattering is the main interesting process, where the energy of the system is large enough so a constituent of each proton (partons) interact and allow the production of additional particles. The interaction that involves any of the other partons, normally at lower energies, is referred to as underlying event. A key phenomenon is the parton shower, a processes where due to the strong interaction, particles loose energy due the radiation of gluons which further generate quark-antiquark pairs, which in turn radiate gluons again in a chain reaction. These generated particles loose energy progressively down to the point where \acrshort{QCD} leaves the perturbative regime ($\sim$1~GeV) and the hadronisation occurs, when quarks and gluons form hadrons, colorless bound states. To complete the simulation of the collision, the pile-up is included which adds the effects from the other proton collisions that originate from the same or previous bunch-crossing.  

\subsection{Factorisation theorem}

The cross-section to produce a final stat $X$ from the hard scattering of two protons, $\sigma_{pp\to X}$ can be factorised into two components in perturbation theory, as the strong coupling constant, $\alpha_s$, is small at high energy kinematic regimes. Using the factorisation theorem \textcolor{red}{cite},

\begin{equation}
    \sigma_{pp\to X}=\sum_{a,b}\int \text{d}x_a\text{d}x_b f_a(x_a,\mu_F^2)f_b(x_b,\mu_F^2)\cdot\hat{\sigma}_{ab\to X}(x_a p_a,x_b p_b,\mu_F^2,\mu_R^2),
\end{equation}

where $f_i(x_i,\mu_F^2)$ are the \acrlong{PDFlabel}~(\acrshort{PDFlabel}) for partons $i=a,b\in\left\{g, u, \bar{u}, d, ...\right\}$ and encodes the probability of finding a parton of type $i$ within the proton carrying a fraction of the proton's momentum $x_i$, at the factorisation scale $\mu_F$. The dependence of the scale appears from performing only fixed-order calculations and the value is typically set comparable to the energy of the process, for example, to the total transverse mass of the final-state particles. The partonic cross-section, $\hat{\sigma}_{ab\to X}(x_a p_a,x_b p_b,\mu_F^2,\mu_R^2)$, is calculated at finite perturbative order, hence the additional dependence on the renormalisation scale, $\mu_R$, at which to evaluate $\alpha_s$.

\subsection{Parton density function}

The \acrshort{PDFlabel}s are crucial for the accurate description of the partons that form the protons. The first type of partons are the valence quarks which determine the quantum numbers of the proton. In addition, gluons and virtual quark-antiquark pairs (sea-quarks) are also part of the proton and come from the vaccum fluctuations. A \acrshort{PDFlabel}, $f_i^A(x_i,Q^2)$ describres the probability density of a parton of a certain type, $i$, inside a given hadron, $A$ to carry a certain momentum fraction, $x=p_i/p_A$ evaluated at a specific momentun transfer $Q^2$. In general, \acrshort{PDFlabel} are excracted from empirical measurements performed at a specific scale. Then, the \acrlong{DGLAPlabel}~(\acrshort{DGLAPlabel}) equations are used to extrapolate the \acrshort{PDFlabel} to different scales. Other alternatives to extract the functions like latice \acrshort{QCD} are possible, but very computationally challenging. %https://arxiv.org/abs/2005.02102 
There are dedicated collaborations such as the \textit{NNPDF}, \textit{CTEQ} and \textit{MSTW} that provide %https://arxiv.org/abs/1410.8849 https://arxiv.org/abs/hep-ph/0508110 https://arxiv.org/abs/0901.0002
 \acrshort{PDFlabel}s for physics analyses. \textcolor{red}{Figure} shows the \textit{NNPDF3.0NLO} \acrshort{PDFlabel} set for the different proton partons and two different factorisation scales.

 There are two main factorisation schemes to describe processes involving $b$-quarks: the \acrlong{4FSlabel}~(\acrshort{4FSlabel}) and the \acrlong{5FSlabel}~(\acrshort{5FSlabel}). The \acrshort{4FSlabel} treats the $b$-quarks massive ($m_b>\mu_R$) and since $m_b>m_p$, they are not included in the sea of quarks and do not have an associated \acrshort{PDFlabel}. In the context of \acrshort{QCD} perturbative evolution, one of the consequences is that calculations at lower scales $\mu_R<m_b$ are especially impacted as the $\alpha_s$ running depends on the number of quark flavours in the initial state, $n_f=4$ \todo{reference euqation of the running}. On the other hand, at high scales the mass effects are negligible and usually described by the \acrshort{5FSlabel}, in which the $b$-quark is considered massless, included in the initial state and treated as the other quarks, $n_f=5$. 

\subsection{Matrix element}

The computation of the partonic cross-section of partons $i,j$ into an arbritrary final state $X$, is related to the \acrshort{ME} amplitude as,

\begin{equation}
    \hat{\sigma}_{ij\to X} \sim \sum_{k=0}^\inf \int \text{d}\Phi_{X+k}\left|\sum_{l=0}^\inf M^l_{X+k}\right|^2(\Phi_F,\mu_F,\mu_R)
\end{equation}

were \acrshort{PDFlabel}s and other normalisation factors are removed for compactness. $M^l_{X+k}$ is the \acrshort{ME} amplitude for the production of $X$ in association with $k$ additional final-state partons, or legs, and with $l$ additional loop corrections. In a perturbative regime, the \acrshort{MElabel} amplitudes for increasingly complex processes (diagrams with additional legs and loops) tend to decrease. As a result, the cross-section is generally computed at a perturvative order, without the sum computed to infinity and for a choice of $\mu_F$ and $\mu_R$. The \acrlong{LOlabel}~(\acrshort{LOlabel}) is the lowest possible order for the calculation, with $k=l=0$. Next, $l=0,k=n$ provides the \acrshort{LO} computation for the production of $X+n$ jets. Finally, $k+l\leq n$ corresponds to a N$^n$LO prediction for the production of $X$, while also to a N$^{n-k}$LO prediction for the production of $X$ in association of $k$ jets.

\subsection{Parton shower}

One problem that arises in the fixed order computations of the differential cross-section is the appearance of logarithmic divergences from collinear splitting that originate from the integration of the phase space, $\Phi$, of the additional $k$ partons. For an inclusive cross-section computation, these divergencies cancel out with virtual corrections order by order, following the KLN theorem. %https://aip.scitation.org/doi/10.1063/1.1724268 %https://journals.aps.org/pr/abstract/10.1103/PhysRev.133.B1549
In this case, the basic event is simulated at fixed order while the \acrshort{QCD} emission process (splitting) is computed with the \acrshort{PSlabel} algorithm, %https://www.sciencedirect.com/science/article/abs/pii/055032138090111X?via%3Dihub
which generates a sequence of emissions with decreasing angle or energy. The algorithm recursively produces the typical splitting processes ($g\to q\bar{q}$, $g\to gg$ and $q\to qg$) for each parton until the energy of the shower reaches $\sim$1~GeV, the hadronisation scale. This showering process that is applied to the final products after the hard-scattering is refered to as \acrlong{FSRlabel}~(\acrshort{FSRlabel}), while the simulation of the \acrlong{ISRlabel}~(\acrshort{ISRlabel}) is performed to the incoming partons. In the case of \acrshort{ISRlabel}, the subsequent emissions grow on energy and are modelled instead with a backwards-evolution algorithm. % https://www.sciencedirect.com/science/article/abs/pii/0370269385906744?via%3Dihub

There is an incompatibility with the full cross-section computation at order $n>1$ as there is a possible overlap in the phase space of the extra partons that are considered for the \acrshort{ME} computation at order $n$ with the ones considered in the \acrshort{PSlabel} at order $n-1$. There are different approaches to solve the double counting, known as ME-PS matching. The most common strategy is known as slicing, which defines a matching scale where the higher energy region is described by the \acrshort{MElabel} while any additional parton below the scale is vetoed and only allowed to be prompted by the \acrshort{PS} algorithm.

\subsection{Hadronisation}



\subsection{Monte Carlo simulation and generators}

The simulation of the hadron collisions is performed through dedicated software tools called
\acrshort{MClabel} generators, which use pseudo-random numbers to generate the distributions predicted by the process. They are generally classified according to which of the steps of the simulation can perform, with general purpose generators being capable of simulating the whole event process, while dedicated generators target specific parts of the chain, such as the matrix element or the parton shower.

Typically, the event generation is divided into two steps: the \acrlong{MElabel}~(\acrshort{MElabel}) generation and the \acrlong{PSlabel}~(\acrshort{PSlabel}) evolution and hadronisation modelling. 
The hard scattering describing the hard scattering and secondly the parton shower (PS) evolution and hadronisation modelling including initial state radiation (ISR) and final state radiation (FSR). While the ME and most
parts of the PS can be calculated perturbatively, the other processes are non-perturbative. A simplified
illustration of this full simulation process is shown in Figure 4.2. For the modelling of the hadronisation, there are different models, the most widely used models are: the Lund string model [81] and
the cluster model [82]. In the Lund string model, the colour connection of a quark-antiquark pair is
described as a string and the potential between them is assumed to be linearly increasing with their
distance. The strings then split according to a fragmentation function forming new quark-antiquark
pairs which continues until only hadrons with on-shell mass remain. The cluster model is based on
QCD pre-confinement, where neighbouring partons build colour-singlet clusters, these clusters then
decay into two hadrons and they then decay further until the final state hadrons are formed.

The full process involving matrix element generation, parton shower, underlying event, hadronisation and fragmentation can be simulated by MC generators like PYTHIA8 [84], HERWIG7 [85, 86]
or SHERPA [87]. However, PYTHIA8 provides mainly leading order calculations which are often not
sufficient since the next-to-leading order (NLO) corrections can be fairly large. HERWIG7 provides
many MEs also at NLO. Since the fraction of negative event weights can be quite large (up to
∼ 40% for certain generator setups), the generator is only used as parton shower in this thesis. In fact,
there are other generators like POWHEGBOX [88–92] or MADGRAPH5_aMC@NLO [93] providing
higher-order calculations which can be interfaced with PYTHIA8 or HERWIG7 for the simulation of
PS and hadronisation.
Furthermore, the models used to describe the non-perturbative processes have parameters that can
be tuned using collision data. The most common tunes used by the ATLAS experiment are the A14
parameters [94] for PYTHIA8 or the H7UE set of tuned parameters [86] for HERWIG7

Throughout this thesis the physics processes for proton-proton collisions at a centre-of-mass energy
√
s = 13 TeV are modelled using various combinations of MC generators and settings. The specific
details are stated in the dedicated chapters. Nevertheless, all MC samples using PYTHIA8 or HERWIG7 to model the multi-parton interaction (MPI), hadronisation and PS use the same settings if not
differently stated. The mass of the top quark is set to mt = 172.5 GeV, the Higgs boson mass to
mH = 125 GeV and the mass of the b-quark to mb = 4.8 GeV for PYTHIA8, to mb = 4.5 GeV for
HERWIG7 and to mb = 4.75 GeV for SHERPA. The simulation of b- and c-hadron decays is performed via the EVTGEN v1.6.0 program [95] with the exception of SHERPA. As mentioned above
the two tunes A14 combined with the NNPDF2.3LO PDF set [96] and H7UE together with the set
of MMHT2014LO PDFs [97] are used for PYTHIA8 and HERWIG7, respectivel


\section{Detector simulation}

The last step in the simulation chain is the detector simulation. The MC generators, as described
in Section 4.1, provide information about stable particles in the final state, not taking into account
the detector response. The full ATLAS detector simulation [98] is performed in two steps. The
first step is based on GEANT4 [99] incorporating the geometry of the detector and providing highly
precise modelling of the particle interactions with the detector matter. However, it comes with the
shortcoming of using a large fraction of the available computing power of ATLAS. As an alternative,
fast calorimeter simulation algorithms [100–102] are developed and already used in practice. They
mimic the GEANT4 results, based on thousands of individual parametrisations of the calorimeter
response, using significantly less computing resources with a trade-off in precision. A comparison
of the necessary CPU time for the different detector simulations are shown in Figure 4.3. In practice,

he fast simulation algorithms are widely used in ATLAS and are called AtlFast-II. In the second
step, the readout electronics and digitisation is simulated which is adjusted for the different detector
systems.
Taking advantage of the latest machine learning developments in the last years, deep generative
algorithms such as Generative Adversarial Networks (GANs) and Variational Auto-Encoders (VAEs)
are studied to improve the fast calorimeter simulation [103] showing already promising results.

\chapter{Object reconstruction}

The concept of \textit{reconstruction} refers to the use of algorithms for the identification of physics objects from the signals recorded in the different sub-systems of the detector. The physics processes described in this thesis produce electrons, muons, taus, photons, neutrinos and quarks in the final state. However, not all of these listed particles can be directly observed, as quarks form cascades of hadronic particles, neutrinos leave without interacting with \acrshort{ATLASlabel} and tau leptons decay before reaching the detector. \textcolor{red}{Figure} illustrates the interaction of different particles with the \acrshort{ATLASlabel} detector. Charged particles produce a track in the \acrshort{IDlabel}, electrons and photons shower in the \acrshort{EMlabel} calorimeter, hadrons shower in the hadronic calorimeter and muons leave signals in the muon spectrometer.

The reconstruction of the different physics objects used in this thesis analyses is described in the following chapter.

\section{Base objects}

The fundamental blocks used in the reconstruction algorithms are tracks, vertices and topo-clusters (or calorimeter energy clusters). All physics objects are composed by these blocks and introduced in the following section.

\subsection{Tracks and vertices}

Tracks are objects produced by charged particles interacting in the \acrshort{IDlabel} and used to identify their trajectory. The reconstruction consists in grouping hits in the different tracking sub-systems and requiring different criteria to ensure the quality of the tracks. The tracks that originate from the hard-scattering are referred to as primary tracks, and the origin of the track (vertex) is referred to as the  \acrfull{PVlabel}~(\acrshort{PVlabel})

As a first step, hits are built from groups of pixels and strips that reach a threshold energy deposit starting from the inner layers \acrshort{IDlabel}. The seed to reconstruct a track consists in three hits in the silicon detector, and then hits from the outer layers of the tracker compatible with the trajectory are added iteratively. When adding points, a score is assigned to the track to quantify the correctness of the track trajectory and suppresses the contribution of random collections of hits (or fake tracks). Then, a dedicated algorithm evaluates the different seeds to limit shared hits, which typically indicate wrong assignments. In addition, quality criteria are applied where tracks are required to have $\pT>500$~MeV, $\abs{\eta}<2.5$, minimum of seven pixel and \acrshort{SCTlabel} clusters, a maximum of either one shared pixel cluster or two \acrshort{SCTlabel} on the same layer, no more than one missing expected hit (or hole) in the pixel detector and a maximum of two holes in both pixel and \acrshort{SCTlabel}. Also, requirements in the transverse impact parameter calculated with respect to the beam position, $\abs{d_0}<2$~mm, and related to $z_0$, the longitudinal difference between the \acrshort{PVlabel} and $d_0$ along the beam, $\abs{z_0 \sin\theta} <3$~mm. As a last step, \acrshort{TRTlabel} hits are added to the tracks after extrapolation.

Vertices are of particular interest as they are the origin of the charged particles or interactions. The \acrshort{PVlabel} is the most important, as denotes the origin of the hard-scattering interaction, but secondary vertices are also characteristic of long-lived particles or for heavy-flavour tagging.\todo{not talked yet}

For a given event, the \acrshort{PVlabel}s are reconstructed iteratively from tracks using a dedicated vertex finding algorithm. From a set of quality tracks, a candidate position is defined and the compatibility with the set of tracks in terms of weights is evaluated in order to recompute the vertex position. In each step then, the tracks that are less compatible are given smaller weights and, after the convergence of the optimal vertex position, are left unassigned and remain as input for the following vertex. The \acrshort{PVlabel} is defined as the vertex with the largest $\pT^2$ sum. 

\subsection{Topological clusters}

Topological cell clusters, or topo-clusters, are objects reconstructed iteratively from calorimeter information and are the first step in the reconstruction of electrons, photons and hadrons. The seed consists of calorimeter cells which readout signal is four times higher than the background noise, and neighbour cells are added if the ratio is higher than two. As a last step, an extra layer is added regardless of the signal-to-background ratio. \todo{a little bit weak?}

\section{Jets}

Jets are the cone-shaped collimated showers formed by the hadronic cascades that originate from the complex interactions of quarks and gluons when travelling through the detector. These objects are essential for physics analyses with partons in the final state, especially $b$-quarks, which jets have particular properties that can be used to characterise them with great efficiency. Nevertheless, the kinematic properties of the cascades are challenging to define, as they can contain information from one or multiple final state partons and from the hard-scattering or other radiation processes. There are different possible definitions that depend of dedicated algorithms which group calorimeter information and do not depend on common \acrshort{QCD} effects. Jet algorithms are collinear safe, referred to the jet not changing if two constituents are merged forming one with double the momentum (or vice-versa), and infrared safe, meaning that the reconstruction is not affected by adding low \pT\ particles.

\subsection{Reconstruction}

The jet reconstruction is typically performed using the anti-k$_t$ algorithm. %https://iopscience.iop.org/article/10.1088/1126-6708/2008/04/063
. This family of algorithms merges clusters based on a relative distance defined as,

\begin{equation}
    d_{i,j} = \min (p_{\text{T},i}^{2n},p_{\text{T},j}^{2n}) \frac{\Delta R_{i,j}}{R^2}
\end{equation}

with $p_{\text{T},i/j}$ the \pT\ of the cluster $i$ and $j$, $\Delta R_{i,j}$ the angle separation between them, $R$ the chosen radius parameter that sets the size of the jet and $n$ chosen integer that defines the \pT\ dependance of $d_{i,j}$. The decision to combine clusters or to define a cluster as a jet comes from comparing the $d_{i,j}$ value with the beamspot distance, $d_{i,B} = p_{\text{T},i}^{2n}$. Clusters are grouped if $d_{i,j} < d_{i,B}$, otherwise the cluster $i$ is defined as a jet, in an iterative process until all input clusters are used. The anti-k$_t$ algorithm is defined by setting $n=-1$, which groups with higher priority the high energy clusters, and leads to a cone-shape around the highest object. This feature can be observed in \textcolor{red}{Figure}.

Various jet collections based on the anti-k$_t$ algorithm are used in \acrshort{ATLASlabel}, two of them are used in this thesis: EMTopo jets and Pflow jets.

\subsubsection{EMTopo jets}

The so-called EMTopo jets were the primary jet collection used in physics analyses in \acrshort{ATLASlabel} before the end of Run~2. The reconstruction is performed at the EM energy scale only using topo clusters %https://doi.org/10.1103/PhysRevD.96.072002
with the anti-k$_t$ algorithm implemented in the \textit{FASTJET} software package %https://link.springer.com/article/10.1140/epjc/s10052-012-1896-2
. The jets used in this thesis are reconstructed with the radius parameter $R = 0.4$ with requirements in $\pT > 25$~GeV and $\abs{\eta} < 2.5$. The EMtopo jets are calibrated in several steps summarised in \textcolor{red}{Figure} %https://arxiv.org/pdf/2007.02645.pdf
. After the jet reconstruction, the jet direction is modified such that the jet originates from the primary vertex. Then, energy corrections based on pile-up are applied subtracting the average energy due to in-time pile-up and other residual corrections that depend on the number of \acrshort{PVlabel} and bunch crossings. After, absolute calibrations are applied to the \acrlong{JESlabel}~\acrshort{JESlabel} and $\eta$ derived from dedicated dijet \acrshort{MClabel} events. Then, a global sequential calibration is set to improve the \pT\ resolution and the associated
uncertainties from the jet fluctuations that can arise from various initial factors like the flavour or energy of the original parton. The final step is the in-situ calibration which is only applied to data, extracted from \pT and $\eta$ comparisons to known well-modelled \acrshort{MClabel} that include central jets in dijet events, $\gamma/Z+jets$ or multijet events. 

\subsubsection{PFlow jets}

Particle Flow jets, known as PFlow jets, were introduced during Run~2 and combine tracking and calorimeter information. This collection of jets has improved energy and angular resolution compared to EMTopo jets and enhanced reconstruction and stability against pile-up. The reconstruction %https://link.springer.com/article/10.1140/epjc/s10052-017-5031-2
is also based on the anti-k$_t$ algorithm with $R=0.4$, and the first step consists in matching the tracks (from the \acrshort{IDlabel}) from charged particles to the topo-clusters. The energy deposits of the matched topo-clusters are replaced by the corresponding track momentum. Then, the resulting topo-clusters and the tracks matched to the \acrshort{PVlabel} are used as input of the anti-k$_t$ algorithm. The jets are calibrated like the EMTopo jets in the range 20~GeV $<\pT<1500$~GeV. 

\subsection{Jet tagging}

Jet or flavour tagging consists in identifying the parton flavour that generated the signal reconstructed as the jet. Efficient tagging is essential for analyses studying processes with $b$- or $c$-quarks in their final state (knows as heavy flavour quarks), as it is additional information which can be used to select events based on the flavour of their jets and improve the selection of the signal.

Jets originating from the hadronisation of $b$-quarks, or $b$-tagged jets, leave a distinct signal due to the properties of $b$-hadrons: lifetime of $\sim 10^-12$~s (decay after 2.5~mm with a momentum of 30~GeV), mass of $\sim 5$~GeV and high decay multiplicity (including semi-leptonic decays). \textcolor{red}{Figure} shows a scheme of a typical signal, that includes displaced tracks from the \acrshort{PVlabel} with large $d_0$.
The signal of the $c$-hadrons is similar but not identical as the lifetime, mass and decay multiplicity are lower, which makes the distinction between these two kinds of jets difficult. The last type of jet is referred to light-flavour jets, which signal originates directly from quark fragmentation and can be easily separated from $b$-jets. However, other phenomena like long-lived particles, photon conversion or low quality tracks can also prompt displaced vertices and tracks.

\subsubsection{Algorithms}

Flavour tagging algorithms use the properties of a given jet to return a score, referred to as output discriminant, which indicates how likely the input jet is considered to be a $b$-, $c$- or $light$-jet. Two main taggers are used in \acrshort{ATLASlabel}: the MV2c10 tagger which was the default option for EMTopo jets, and the DL1r tagger that is the recommendation for PFlow jets.

The MV2c10 tagger  %http://cdsweb.cern.ch/record/2037697 
is based on the MV2 algorithm, which relies on boosted decision trees~(BDTs) trained with several kinematic and other trigger taggers as inputs. This particular tagger was trained with \ttbar\ and $Z'$ events, to cover a large \pT\ spectrum, and $b$-jets defined as signal while the background was defined to consist of 7\% $c$-jets and 93\% light-jets.

The DL1r tagger is a multi-class Deep Neural Network (DNN) model, with three output nodes corresponding to the classification of the input jet to be a $b$-, $c$- or light jet. The final discriminant is given as a function of the three probabilities. The input consists in the MV2c10 tagger input, additional variables for $c$-jet identification used in a jet vertex finder algorithm and flavour probabilities provided by a recursive NN \todo{NN defined at this point?}. The training set consists of the same \ttbar\ + $Z'$ events, weighted to have an equal mix of quark flavour jets.

Comparing the two algorithms used in this thesis, the DL1r tagger is the most recent recommendation and relies on more advanced machine learning techniques than the MV2c10 tagger. The multi-class output together with the possibility of tunning the final discriminant computation, makes the DL1r tagger more flexible than the binary classification of the MV2c10. Regarding performance, the efficiency of both algorithms to tag true $b$-jets is comparable, while the rejection rates of $c$- and light jets is greater for the DL1r tagger. The improvement in rejection for the 60\% WP, detailed below, is by up to 70\% for $c$-jets and 120\% for light jets.


\subsubsection{Working points}

The full spectrum of the final $b$-tagging discriminant is not directly used in physics analyses due to the complexity of the calibration. Instead, four different $b$-tagging \acrlong{WPlabel}~(\acrshort{WPlabel}) are defined based on the $b$-jet acceptance efficiency evaluated on a \ttbar\ sample: 60\%, 70\%, 77\% and 85\%, which are often referred to as \textit{Very Tight}, \textit{Tight}, \textit{Medium} and \textit{Loose} operating points, respectively. Most of the $c$- and light-jets do not pass the 85\% \acrshort{WPlabel}, ending up in the $b$-tagging efficiency between 85\% and 100\%. Meanwhile, the jets that pass the 60\% \acrshort{WPlabel} mainly consists in $b$-jets. This criteria is important when defining the $b$-jets for event selection, as the $b-$jets misidentification, so $c$- and $light$-jet acceptance inefficiency, improves for lower $b$-jet efficiency working points, therefore rejecting more background but with lower signal statistics. On the other hand, the pseudo-continuous b-tagging \acrshort{WPlabel}, so the \acrshort{WPlabel} that a jet passes, is additional information that can be used to further refine the selection or in multivariate methods.

\section{Leptons}

\subsection{Electrons}

Electrons interact with the \acrshort{IDlabel} and the \acrshort{EMlabel} calorimeter system. The typical signature is a track in the \acrshort{IDlabel} and electromagnetic shower in the \acrshort{EMlabel} calorimeter. Overall, the performance in terms of identification and reconstruction of electrons is high.

First, topo-clusters are selected and matched to \acrshort{IDlabel} tracks in the region $\abs{\eta}$ < 2.47 excluding the transition region of the barrel and end-cap (1.37 < $\abs{\eta}$ < 1.52). Next, the matched clusters are grouped to form superclusters, which are variable-size clusters, using a dynamic clustering algorithm. After a first energy and position calibration, tracks are matched to the electron superclusters. The calibration of the energy scale and resolution of electrons is computed from $Z\rightarrow ee$ decays and validated in $Z\rightarrow \ell\ell\gamma$ %https://doi.org/10.1088/1748-0221/14/12/P12006
. In addition, the energy resolution of the reconstructed electron is optimised using a multivariate algorithm based on the properties of showers in the \acrshort{EMlabel} calorimeter.

Further identification criteria are required for an electron candidate, passing a selection to increase the purity. The prompt electrons are evaluated with a likelihood discriminant to define three operating points with different purities: \textit{Tight}, \textit{Medium} and \textit{Loose}. The discriminant is computed using variables measured in the \acrshort{IDlabel} and the \acrshort{EMlabel} calorimeter, chosen such that they discriminate prompt isolated electrons from other signals deposits (jets, converted photons or other electrons from heavy-flavoured hadron decays). The most important quantities are based on the track quality, the lateral and longitudinal development of the electromagnetic shower as well as the particle identification in the \acrshort{TRTlabel}. The probability density function to build the likelihood are derived from $Z\rightarrow ee$ ($E^\text{T}$ > 15~GeV) and $J/\psi \rightarrow ee$ ($E^\text{T}$ < 15~GeV) events.
%Figure 6.4 shows the data efficiency as a function of ET and as a function of the average number of bunch crossings for all three operating points. They are all optimised in 9 |eta| and 12 ET bins. 

Another requirement is the isolation criteria, to require the electron signal to be separated from other particles. Electrons are typically required to be spatially separated from other particles, based on two quantities: a maximum value for the sum of transverse energy of topo-clusters in a $\Delta R=0.2$ cone surrounding the electron and of the sum of transverse momentum of tracks around the electron, with a $\Delta R$ cone that decreases with \pT. Effects of leakage and pile-up are taken into account and also tracks are required to satisfy \pT\ > 1~GeV, $\abs{\eta}<2.5$ and quality criteria. In this thesis, the criteria used is the Gradient isolation which has an efficiency of 90\% at $\pT = 25$~GeV and 99\% at $\pT = 60$~GeV.

\subsection{Muons}

Muons leave the \acrshort{ATLASlabel} detector without significant energy loss. The typical signal consists on a track in the \acrshort{IDlabel} and \acrshort{MSlabel} sub-detectors. There are different types of muons depending on which \acrshort{IDlabel}, \acrshort{MSlabel} or calorimeter information is available %https://link.springer.com/article/10.1140/epjc/s10052-016-4120-y
.

As a summary, the muon reconstruction has two stages: tracks are reconstructed independently in the \acrshort{IDlabel} and \acrshort{MSlabel}, and then are combined to form the muon tracks. The muon track candidates are built from track segments found in the different \acrshort{MSlabel} sub-systems. In the muon trigger chambers and \acrshort{MDTlabel}s, segments are reconstructed with a straight line to fit the hits of each detector layer after an alignment to the trajectory in the bending plane of the detector. The \acrshort{RPClabel}, \acrshort{TGClabel} and \acrshort{CSClabel} hits provide measurements in the orthogonal direction and the forward region of the detector to build additional track segments. The muon track candidates are then built from the track segments fit together using a global $\chi^2$ fit. With that information, different types of muons can be defined.

The combined (CB) muons are the muon candidates obtained from using combined information from \acrshort{MSlabel} tracks that are extrapolated to the tracks of \acrshort{IDlabel} (an inside-out approach is also used). The segment-tagged (ST) muons are reconstructed from tracks in the \acrshort{IDlabel} extrapolated to typically one track segment in the \acrshort{MDTlabel}s and \acrshort{CSClabel}. ST muons are normally low in \pT\ and in regions with low acceptance. Calorimeter-tagged (CT) muons are built from an \acrshort{IDlabel} track that is instead matched to an energy deposit in the calorimeter compatible with a minimal ionising particle. The CT muon strategy outputs the lowest purity, although proves useful for detector regions not fully covered by \acrshort{MSlabel}, optimised for 15~GeV < \pT\ < 100~GeV and $\abs{\eta}<0.1$ is optimised. The fourth type, extrapolated (ME) muons, are only reconstructed using the \acrshort{MS} with an acceptance of $2.5 < \abs{\eta} < 2.7$.

The muon identification criteria (similar to the electron identification) is performed applying quality criteria to increase the purity of the selection. In order to identify prompt muons with high efficiency and a
good momentum resolution, a requirement is done for amount of hits in the \acrshort{IDlabel} and the \acrshort{MSlabel} systems. Four different muon operating points are defined: \textit{Tight}, \textit{Medium}, \textit{Loose}, \textit{high \pT} and \texit{low \pT}. The Medium and Loose working points are used in this thesis. The first one is widely used in physics analyses and is designed to minimise muon reconstruction and calibration systematic uncertainties. It consists of combined and extrapolated muons with three or more hits in at least two \acrshort{MDTlabel} layers, or just one hit for $\abs{\eta}<0.1$ with no more than one hole in the \acrshort{MSlabel}. On the other hand, the Loose working point maximises the reconstruction efficiency and accounts all types of muons, adding the segmented- and calorimeter-tagged muons for $\abs{\eta}<0.1$. The reconstruction efficiency for muons with \pT > 20~GeV at the Medium and Loose working points is 96.1\% and 98.1\%, respectively. beginning 

The isolation criteria is based on track and calorimeter variables, similar to the electron case. The criteria improves the efficiency removing non-prompt muons, the ones not generated in the hard-scattering but in other parton shower processes for example, which are usually close to jets and other objects. The track related variable, \pT^{\text{varcone30}} is the scalar \pT\ sum of the additional tracks in a cone $\Delta R=10$~GeV$/\pT^\mu$ (maximum of 0.3), that depends on the muon transverse momentum \pT$^\mu$. The calorimeter related variable is the same as for electrons, build from the sum of energies around the muon track. In this thesis, the \textit{FixedCutTightTrackOnly} working point is used, which is defined only with track isolation: $\pT^\text{varcone30}/\pT^\mu<0.06$.       

\section{Taus}

The $\tau$ leptons typically decay before reaching active electronics of the \acrshort{ATLASlabel} detector and have to be identified via their decay products. The decay can be either leptonically (into electrons or muons) or hadronically. The leptonic decay represents the 35\% of the cases and is covered by the reconstruction of the produced electron or muon. The hadronic decays represent 65\%, which contain one or three charged pions in 72\% and 22\% of the cases, respectively. In addition, at least one associated neutral pion is also produced in 68\% of the hadronic decays. The dedicated $\tau$ reconstruction and identification algorithms in \acrshort{ATLASlabel} target the hadronic decay, with the main background being jets from energetic hadrons produced in the fragmentation of quarks and gluons, known as the \acrshort{QCD} background. Therefore, the $\tau$ objects in \acrshort{ATLASlabel} mentioned in this thesis refer to hadronically decaying $\tau$ leptons. 

The candidates are seeded by jets which are required to have $\pT > 10$~GeV and $\abs{\eta} < 2.5$ excluding the barrel-end-cap transition region %https://cds.cern.ch/record/2261772
. The tau identification is based on a machine learning classifier which is trained using the calorimeter information and the tracks associated to the jet candidate. A trained BDT is used for EMTopo jets while a recursive neural network (RNN) is used for PFlow jets. Three different efficiency working points are defined: \textit{Loose
}, \textit{Medium} and \textit{Tight}. The $\tau$ leptons used in this thesis are defined with the medium working point, required to have $\pT>25$~GeV and an isolation criteria of $\Delta R<0.2$ between the $\tau$ and any selected electron or muon.

\section{Missing transverse energy}

The missing transverse momentum, also denoted as \MET\ is the transverse component of the negative vector sum of the fully calibrated objects (electrons, muons, photons, $\tau$ leptons and jets) as well
as soft objects associated to the \acrshort{PVlabel}. In an ideal detector, the the sum of four-momenta of all particles produced is equal to the net momentum of the initial collision, implying that the net momentum in the transverse plane of the collision has to be zero, $\MET=0$. Nevertheless, the net momentum is not null as particles like neutrinos leave the detector without depositing energy or others can interact with the detector in regions not covered by electronics. For analyses with neutrinos in the final state, it typical to consider that the transverse energy carried by the neutrinos is the $\MET$, which allows their reconstruction.

\chapter{Machine learning}

\chapter{Machine learning and statistical methods}
\label{chapter:MLStat}

This chapter introduces the Machine Learning (ML) methods used in the two analysis described in this thesis to enhance the separation between the signal and the background, and also the statistical methods used to extract the signal.\\

\acrlong{MLlabel}~(\acrshort{MClabel}) is one of the core developing fields in computer science allowing the analysis of large and complex datasets, offering sophisticated techniques with a broad range of possible applications. Regarding high energy physics, the large amount of \acrshort{MClabel} simulations or data that is being recorded makes it a field well suited for the application of \acrshort{MLlabel} techniques. In this chapter, different multi-variate techniques used in this thesis are introduced, focusing on the classification methods used to improve signal and background separation.\\

In order to test the predictions of a given model, experimental data and \acrshort{MClabel} simulations are compared using statistical methods. This chapter describes the tools used to extract the production cross-section of a given signal, in the absence of it, their upper limits.

\section{Machine Learning}

The deployment of \acrshort{MLlabel} methods is already reaching crucial tasks in \acrshort{ATLASlabel} such as online data recording, the implementation of neural networks in calorimetry FPGAs~\cite{Laatu:2022fni} or for particle reconstruction in trigger algorithms~\cite{ATLAS:2019uhp}, which result in more efficient triggers than previous ones.\\

For these cases, a neural network is trained to reduce the signal to background ratio, offering a high-level discriminating variable for a classification problem or providing a prediction of a certain quantity. These methods can outperform conventional algorithms as machine learning algorithms perform the classification or inference from multi-dimensional inputs, allowing the extraction of more complex correlations and functions, given enough data. In addition, \acrshort{MLlabel} methods scale easier than non-\acrshort{MLlabel} algorithms in terms of the size of the dataset and the amount of input variables, or features.\\

On the other hand, detector simulation is one of the most computational intensive tasks within \acrshort{ATLASlabel}, especially the calorimeter simulation, and solutions involving adversarial networks and auto-encoders are being studied to output faster and reliable output.\\

Regarding particle reconstruction and identification, examples of \acrshort{MLlabel} implementations can be found within the $\tau$ identification~\cite{ATLAS:2019uhp} or $b$-tagging algorithms~\cite{ATL-PHYS-PUB-2020-014}. In physics analyses, the use of \acrshort{MLlabel} is already standardised typically to reconstruct signal processes or to discriminate them from the background. In this case, the output is a high discriminating variable that can be used to define high-purity signal analysis regions, for example.\\

\acrshort{MLlabel} algorithms are not designed for a specific task, they consist in general models with hundreds of free parameters that are tuned using data. The process of optimising the internal parameters to correclty solve a specific problem is known as training the model. Depending on the available data, model and its application, several steps are needed to adequately perform and evaluate the training.\\

Generally, two types of \acrshort{MLlabel} algorithms can be distinguished: Supervised learning and Unsupervised learning models. The main difference is that the first requires fully labelled training data, namely that each instance in the data is associated with a known output or target value (like the true mass of a particle). The provided labels allow the model to map the input features and output labels, to generalise the relationship to unseen data. On the other hand, unsupervised learning models do not rely on labelled data and instead are used to uncover hidden patterns or structures within the data itself (like detection of anomalies). In the context of this thesis, supervised approaches are used based on Neural Networks (NNs) and Boosted Decision Trees (BDTs).\\

The relationship between a machine learning model, its parameters, and the input data can be formally expressed using statistical notation.The statistical model of a general \acrshort{MLlabel} algorithm is denoted as $\text{P}_\text{model}(\mathbf{x_i}; \boldsymbol{\theta})$ and is parameterised with a set of parameters $\boldsymbol{\theta}$ while $\mathbf{x_i}=(x_i^1,x_i^2,...,x_i^M)$ is the input feature set of a single input data point $i$ of the $\mathbf{X}=(\mathbf{x_1},\mathbf{x_2},...,\mathbf{x_N})$ dataset consisting of $N$ data points. $\text{P}_\text{data}$ is the true distribution that generates the data but is unknown. In the case of supervised learning, every data point has a label that categorises the event, $\mathbf{y}=(y_1,y_2,...,y_N)$.

\subsection{Performance}

The model performance is usually the decisive measure of a \acrshort{MLlabel} method and, depending on the objective, different metrics are used. In the context of the NN implemented in this thesis, the loss function and the Area Under the ROC Curve (AUC) are presented. The loss function is the quantity optimised during the model training while the AUC is the quantity used to characterise the performance of the model.

\subsection{Loss function}

The loss function $E$ or cost function is optimised during the model training and represents the deviation of a model from the desired behaviour. To be suitable for minimisation, the function has to be differentiable. The choice of the loss function depends on the problem and requires optimisation. For supervised learning, the loss function depends on the true and predicted label values of the data so the worse the prediction is, the higher the loss function.\\

It is standard to express the loss of the full dataset $\mathbf{X}$ as the average of loss of the single data points $\mathbf{x_i}$,

\begin{equation}
    E(\mathbf{X},\boldsymbol{\theta})=\frac{1}{N}\sum_{i=1}^NE(y_i,\text{P}_\text{model}(\mathbf{X},\boldsymbol{\theta}))
\end{equation}

For regression problems the typical expression is the mean square error~\cite{EncyclopediaofML}, 

\begin{equation}
    E_{MSE}(\mathbf{X},\boldsymbol{\theta}) = \frac{1}{N}\sum_{i=1}^N (y_i\text{P}_\text{model})^2,
\end{equation}

which is an average of the deviation from the true labels. For binary classification, the so-called \textit{binary cross-entropy}~\cite{binarycross} is frequently used, 

\begin{equation}
    E_{BCE}(\mathbf{X},\boldsymbol{\theta}) = -\frac{1}{N}\sum_{i=1}^N y_i\cdot\log(\text{P}_\text{model}(\mathbf{X},\boldsymbol{\theta}))+(1-y_i)\cdot\log(1-\text{P}_\text{model}(\mathbf{X},\mathbf{\theta})),
\end{equation}

which is the negative log-likelihood of a Bernouilly distribution. A modified version is used in multi-classification problems. 

\subsection{Area Under the ROC Curve}

In the analyses presented in this thesis, the NN is trained to output a high-level variable with high separation between signal and background. Hence, the loss function is not the only important criteria as both the shape and the separation between the signal and background output distributions are essential to estimate the performance of the analyses. The quantity to evaluate the separation between signal and background for a given variable is the \textit{Area Under the ROC Curve} (AUC), and is the main decisive variable for to choose a training in this thesis.\\

When considering two distinct variable distributions, one originating from a signal sample and the other from a background sample, the \textit{Receiver Operating Characteristic} (ROC) curve is defined as the signal efficiency against the background rejection, and displays the trade-off between signal efficiency and background rejection when defining classification threshold. The shape of the curve already provides a lot of the information regarding the overlap of the distributions, but the characteristic quantity is the integral of the curve, the AUC. Figure~\ref{ML:AUC} illustrates two examples with a different degree of overlap. The AUC minimum value is 0.5, when the overlap between two distributions is total, while the maximum value is 1, when the distributions are totally separated. In the case of AUC=1, a cut in the discriminating variable is able to distinguish completely between signal and background.

\begin{figure}[htbp]
    \RawFloats
    \begin{center}
    \includegraphics[width=0.75\textwidth]{ML/AUCexample.pdf}
    \caption{
        Comparison of Gaussian distributions and corresponding ROC curves to illustrate the behaviour of the ROC curve and the AUC.
    }
    \label{ML:AUC}
    \end{center}
\end{figure}

\subsection{Neural networks}
\label{ML:SectionNN}
Neural Networks (NNs) were introduced in the 1940s~\cite{McCulloch1943} but became feasible in the last decades as large computing power and GPUs are widely available. The concept of a NN consists of nodes (neurons) connected to each other via weights, and the most basic network is referred to as feed-forward NN.\\

An example is illustrated in Figure~\ref{ML:normalNN}, with in one input layer, one hidden layer and one output node. The result of the node has the form of a linear system $b+w\cdot x$, with bias $b$, weight $w$ and the input of the neuron $x$. More technically, the result of every node (except the input layer) is given by the sum of its inputs, the output of the different neurons connected to the given node multiplied by the weights $w_i$, which represent the connection. More generally, a bias $b_i$ is added to the sum and then used as input to an activation function $f_i$, which introduces non-linearity. The final output of the feed-forward NN in Figure~\ref{ML:normalNN} can be expressed as

\begin{figure}[htbp]
    \RawFloats
    \begin{center}
    \includegraphics[width=0.75\textwidth]{ML/nnnormal.pdf}
    \caption{
        Fully connected feed-forward neural network with two input nodes, one hidden layer with two nodes and one output node.
    }
    \label{ML:normalNN}
    \end{center}
\end{figure}

\begin{equation}
    \text{P}_\text{model}(\mathbf{X},\boldsymbol{\theta}) = f_2(\mathbf{b_2}+\mathbf{W_2}f_1(\mathbf{b_1}+\mathbf{W_1}\mathbf{x})),
\end{equation}

with the inputs of a given data-point $\mathbf{x}$; the parameter set $\boldsymbol{\theta}$ including weight matrices $W_i$ and bias terms $b_i$. The index $i=1$ ($i=2$) denotes the hidden layer (output layer) and it is assumed that the two nodes in the hidden layer use the same activation function. Fully expanding in matrix notation,

\begin{equation}
    \text{P}_\text{model}(\mathbf{X},\boldsymbol{\theta}) = f_2\left( \begin{bmatrix} b_{11}^{(2)} \end{bmatrix}+ \begin{bmatrix} w_{11}^{(2)} & w_{12}^{(2)}\end{bmatrix}f_1\left( \begin{bmatrix} b_{11}^{(1)} \\  
                                                            b_{21}^{(1)}  \end{bmatrix}+ \begin{bmatrix} 
    w_{11}^{(1)} & w_{12}^{(1)} \\ 
    w_{21}^{(1)} & w_{22}^{(1)}\end{bmatrix} \begin{bmatrix}
        x_{11}\\
        x_{21}
    \end{bmatrix} \right) \right)
\end{equation}

This simple example has nine free parameters $\boldsymbol{\theta}$ which are optimised during the training; more complex networks easily reach several ten-thousands of free parameters. Due to the non-linearity introduced by $f_i$, a NN can approximate any arbitrary function by giving the network enough freedom (amount of hidden layers and nodes). When a NN consists of multiple hidden layers, it is referred to as a deep learning~\cite{Goodfellow-et-al-2016} algorithm.\\

The main NN structure used in this thesis are feed-forward NN, but there are a vast number of different architectures available developed for very different applications~\cite{livingreview}. Several software packages are available and accessible to the public,  mainly in Python, with popular examples as Pytorch~\cite{NEURIPS2019_9015}, Tensorflow~\cite{tensorflow2015-whitepaper} and Keras~\cite{chollet2015keras}. The main package used in this thesis is the latter, whose models are deployed in \acrshort{ATLASlabel} using the C++ based package lwtnn~\cite{lwtnn}.\\

The training of a NN is the process consisting on the optimisation of the free parameters $\boldsymbol{\theta}$ through a process called gradient descent, which iteratively adjusts the parameters by minimising the loss function. Apart from the free parameters, there are others that are manually set called hyperparameters, and include for example the number of layers and nodes in the NN.\\

Batch training is the typical procedure, where the minimisation of the loss function is done in steps with the training data divided into equally sized data segments. At every step, the loss function is calculated using a segment of data and the internal parameters are updated. A full iteration over the entire dataset is referred to as an epoch. This batch training method makes the minimisation of the loss function faster. If the training would use the full dataset in one step it would end up profiling the very specifics of the training dataset, hence loosing generalisation when evaluating other data not used in the training, this is called overtraining. It could also lead to stop the minimisation process in local minima thus not reaching the true performance.\\

For the batch training approach, the dataset is randomised and then split into an adequate dataset batch size to ensure that every batch is a correct representation of the dataset. Hence, the batch size is a hyperparameter of a NN trainings. If it is chosen too small, the minimisation is faster but the loss function may vary significantly at every step, and may lead to lower performance, as the model could end up being too general.

\subsubsection{Optimiser}

After a first random initialisation of the free parameters $\boldsymbol{\theta}$, they are updated iteratively at each step of the training following,

\begin{equation}
    \theta' = \theta - l \nabla_\theta E(\theta;\mathbf{X})
\end{equation}

where $l$ is the learning rate, also a hyperparameter, that tunes the rate of the update of the weights, presenting the gradient of the loss function with respect to a given free parameter. A large value of $l$ hinders optimal convergence of the loss function as the value might change significantly between steps and may cause the minimisation process to behave erratically before reaching the minimum. On the other side, an excessively low learning rate can slow the optimisation and the minimisation to be stuck in a local minimum. Techniques that vary the learning rate or the batch size at every step can mitigate these extremes and gradually shift to a more precise minimisation of the loss function~\cite{LRBatchSize}. In this thesis, the Adam optimiser~\cite{Kingma2015AdamAM} is used which estimates of the first and second moment of the gradient.

\subsubsection{Backpropagation}

For the gradient descend method, it is necessary to compute the gradient of the loss function with respect to all trainable parameters is needed. However, calculating this gradient analytically for neural networks is infeasible, as it involves nested gradients. Backpropagation~\cite{Rumelhart1986} is used to compute the gradient, which systematically calculates the nested gradients by applying the chain rule to traverse the entire network. To illustrate this, consider the chain rule for the lost function,

\begin{equation}
    \frac{\partial E}{\partial w_{ij}} = \sum_k \frac{\partial E}{\partial y_k} \frac{\partial y_k}{\partial w_{ij}},
\end{equation}

with $y_k$ the output values of the neural network. The backpropagation algorithm calculates these partial derivates efficiently through the reuse intermediate results and the consideration of parameter dependencies. 

\subsubsection{Activation Functions}

The introduction of activation functions $f(z)$ is essential to allow the neural networks to act as a non-linear function. Many candidates of activation functions exist, from simple step functions to monotonically increasing functions (as $\tanh$) or logistic functions (as the \textit{sigmoid} function). Although these activation functions are simple and manage to harmonise the inputs of a node, the resulting NN suffers from vanishing gradient issues which significantly slow the training. The \textit{Rectified Linear Unit} ($\textsc{ReLU}$)~\cite{relu} activation function is widely used and is defined as,

\begin{align}
    f_\textsc{ReLU}(z)= \begin{cases}
            0& \text{for}  z<0 \\
            z& \text{for}  z\geq0\\
            \end{cases}\qquad f'_\textsc{ReLU}(z)= \begin{cases}
                0& \text{for}  z<0 \\
                1& \text{for}  z\geq0\\
                \end{cases}
\end{align}

This function is not affected by vanishing effects and the gradient is fast to compute, as the derivative function $f'(z)$ is very simple. Other popular activation functions are the Leaky $\textsc{ReLU}$~\cite{lrelu} and \textit{Softplus}~\cite{Maas2013RectifierNI}.\\

In general, the output nodes have different activation functions depending on the desired shape of the result. For classification problems, such as the neural networks used in the thesis, the output node can be evaluated with a sigmoid function, allowing the output to be interpreted as a probability,

\begin{equation}
    f_\text{sigmoid}(z)=\frac{1}{1+e^{-z}}
\end{equation}

\subsubsection{Regularisation}

Besides the training performance, the ML model should be resilient to fluctuations in the training data or by the randomness of the training process itself. If performing the training with two equivalent data-sets results in very different models and performance, it might be a strong indication of overfitting or other instabilities, which is difficult to study as a NN has many free parameters.\\

To protect the model's robustness and mantain its generalisability, regularisation techniques are employed during the training. The most popular stochastic methods~\cite{JMLR:v15:srivastava14a,batchnorm,earlystop} are introducing dropout, batch normalisation or early stopping. The dropout method, randomly removes an adjustable percentage (hyperparameter) of weights between neighbouring layers, thus avoiding strong correlations between neurons. Batch renormalisation consists on scaling the input of the layers while the early stopping method halts the training process after a given criterion to avoid overtraining. An early stopping method example could be to stop the training when the loss of the validation set does not improve after a certain amount of epochs. Another popular method, the L2 regularisation, consists in adding a term in the loss function that includes the sum of the squared weight values, thus penalising large weight values.\\

In this thesis, the dropout and early stopping are applied, while no big changes were seen when introducing other methods like the L2 regularisation method. The batch normalisation is not applied, although an equivalent approach is applied to the input layer where the different input variables are scaled to have a mean value of 0 and a variance of 1. This avoids large differences in weights due to the difference of units. One consideration is that distributions like \pT\ are not bounded like $\eta$ and outliers at the tail of the distributions can introduce instabilities to the model.

\subsubsection{Neural network parameterisation}

The analyses presented in this thesis use a NN technique referred to as parameterisation, usually applied to NN with signal or background events generated with different parameters. The parameterised NN~\cite{Baldi_2016} appears as a structure that simplifies training setups as it can replace various classifiers trained at individual values of a parameter. It is trained using the complete statistics produced with the various values of the parameter, thus the training dataset is larger.\\

The NN has as input this parameter that distinguish different classes of interest, like the training label set, a generator parameter or a source of uncertainty~\cite{Ghosh_2021}. Consequently, the response depends on the introduced parameters as depicted in Figure~\ref{ML:PNN}.\\

In the different searches presented in the thesis, the true mass of the targeted new BSM particle is used as a parameter. As the parameter can be used to directly classify the events, the training should be set up appropriately. For signal events the parameter corresponds to the mass of the corresponding sample, while for background it is not well-defined and a random value is assigned to each event, reproducing the same distribution used for the signal events. This makes the NN not to directly use the parameter to perfectly classify the events, while the classification is optimised for each signal.

\begin{figure}[htbp]
    \RawFloats
    \begin{center}
    \includegraphics[width=0.75\textwidth]{ML/nntheta.pdf}
    \caption{
        Simple schematic of a fully connected feed-forward Neural network with input features $x_1, x_2,...$ as well as an input parameter $\theta$, such that the response $f$ depends on the parameter.
    }
    \label{ML:PNN}
    \end{center}
\end{figure}

\subsection{Boosted decision trees}

Boosted decision trees (BDTs) were one of the most commonly used multivariate technique in the last decade of high energy physics, before NNs became more accessible and popular. \\

The unit of a BDT is a decision tree, depicted in Figure~\ref{ML:BDT}. The structure is like a tree, as the name suggests, with branches connected via nodes. A cut on a specific input is made at each node, repeated until a stop criterion is met. The most common criteria are that the minimum events in a leaf is reached or that the maximum amount of cuts is reached (maximal tree depth). The decision tree alone is a weak learner and very sensitive to small changes in the training data, while an ensemble of weak learners leads to a powerful and robust model.


\begin{figure}[htbp]
    \RawFloats
    \begin{center}
    \includegraphics[width=0.75\textwidth]{ML/BDTtree.pdf}
    \caption{
        Schematic representation of a decision tree trained on a dataset composed of signal and background events with example training variables and cuts. The output of the tree is the probability that each event has of being generated by signal. Values above (below) 0.5 correspond to signal-like (background-like) events.
    }
    \label{ML:BDT}
    \end{center}
\end{figure}

Boosting is an ensemble technique for decision trees that combines their individual responses into a single discriminant,

\begin{equation}
    \text{P}_{\text{model}} = \sum_{n=1}^N \alpha_n \text{P}_{\text{tree}}^{(n)}(\mathbf{x_i})
\end{equation}

where $\text{P}_{\text{tree}}^{(n)}(\mathbf{x_i})$ denotes the statistical model of the decision tree $n$, $\mathbf{x_i}$ the input variables and $\alpha_n$ is a weight assigned to each tree's prediction. The boosting algorithm adjusts these weights to minimise the error in the prediction of the ensemble.\\

Different boosting methods are widely available, with the most popular for classification are Gradient Boosting (GradBoost) and Adaptive Boosting (AdaBoost)~\cite{FREUND1997119}. A GradBoost BDT~\cite{Chen_2016} involves training individual trees sequentially by computing the loss function, typically $E_{MSE}$, and adding the contribution of the next tree is added to the ensemble weighted such as the loss function is minimised,

\begin{equation}
    E_n = E\left(\text{P}_{\text{model}}^{(n-1)}(\mathbf{x_i})+\alpha_n \text{P}_{\text{tree}}^{(n)}(\mathbf{x_i})\right)
\end{equation}

AdaBoost is a specific case in which the weights of the events that are misclassified by a given tree are increased to have a greater impact in the loss function minimisation, hence enhancing learning in challenging phase spaces. Nevertheless, this can lead to a model sensitive to statistical deviations of the dataset or outlier events.\\

Implementation of BDTs are in typical \acrshort{ATLASlabel} software, ROOT~\cite{BRUN199781} via the TMVA~\cite{TMVA} package and more widely in python via scikit-learn~\cite{scikit-learn} or xgboost~\cite{Chen_2016}.

\subsection{Input data for training}

The selection of the dataset and its size depends on the problem that the \acrshort{MLlabel} is intending to solve. The datasets used in \acrshort{MLlabel} methods to discriminate between signal and background processes typically consist of simulated events. When the goal is to reconstruct a specific kinematic quantity, for example, only the events from the desired signal process are used instead. The choice of dataset and the number of inputs mostly depend on the problem's complexity and the desired performance, as more intricate scenarios require advanced algorithms with higher number of input variables and large datasets.\\

The input variables used in \acrshort{MLlabel} trainings of this thesis can be categorised as low- or high-level variables. The low-level variables are quantities of individual physics objects that have not been combined or designed to directly help the problem, like the kinematics of the objects of a collision event. High-level variables are referred to those obtained combining low-level variables and designed to offer discrimination, as reconstructed kinematics of a particular signal or the output of other classifiers, like $b$-tagging scores. Although high-level variables offer a lot of discrimination, a complete set of low-level variables have the necessary information to reach or even surpass the same level of discrimination, as correlations between variables can be exploited in advanced setups.\\

It is important to ensure an unbiased training process. For this purpose, the full dataset is split at least in two orthogonal samples, normally called training and validation datasets. The training dataset is used for the actual algorithm training, while the validation dataset is used to evaluate and monitor the final model. While a loss function is used to find the best set of parameters during the training, the performance on the validation set is evaluated in terms of sensitivity, selection efficiency, stability, in order to fine-tune the model such as the choice of input variables or hyperparameters. Ideally, a third dataset referred to as testing set is only used to evaluate the final model and is not involved in the training process or choice of hyperparameters. Some trainings performed in this thesis are not performed with the testing set although no significant bias is introduced, as the difference in performance between the validation and testing set is below statistical effects. In order to evaluate the full dataset, cross-validation (also named $k$-folding~\cite{EncyclopediaofML}) setups are used, where $k$ trainings are performed with the train/validation/test sets labelled accordingly, so every set can be evaluated appropriately.\\

In this thesis and typical \acrshort{ATLASlabel} analyses, every simulated event has an associated unique number not correlated with any physical variable. Hence, it is ideal to split the full dataset into the different sets using this number like splitting the dataset into two with odd or even numbers. Another detail is that every simulated event has an event weight, so the full simulated process has the appropriate cross-section and kinematic distributions, whose value can also be negative in some cases. This is almost exclusive of high energy physics datasets and, although the user level tools accept event weights as input, the absolute value of the event weights has to be used, as negative values are not properly defined in the training.

\section{Profile likelihood fit}
\label{sec:profilelikelihoodfit}

In order to test the compatibility between data and the \acrshort{MClabel} simulations, statistical methods in the context of hypothesis testing are used. The profile likelihood fit is a statistical tool used in this thesis to extract a measurement for the amount of the signal searched for in the analysis. When the presence of signal is not significant, upper limits are extracted based on the asymptotic formulation~\cite{Cowan_2011}. In this section, the profile likelihood fit method is presented with the necessary concepts in the context of a \acrshort{BSMlabel} search.
The technical implementation is provided by the RooStat framework~\cite{10.48550/arxiv.1009.1003}.\\

The fundamental idea behind hypothesis testing is to compare the agreement of the experimental data between two hypotheses and quantify which hypothesis can be discarded with a certain level of confidence. The two hypotheses to be compared are: the null-hypothesis $H_0$, corresponding to the~\acrshort{SMlabel} without new physics; and the alternative hypothesis $H_\mu$, which accounts for \acrshort{BSMlabel} interactions. The $\mu$ refers to the signal strength, commonly referred to as parameter of interest (POI), and is a normalisation factor for the sought signal,

\begin{equation}
    \mu = \frac{\sigma}{\sigma_{ref}}
\end{equation}

where $\sigma$ is arbitrary and $\sigma_{ref}$ is a reference value, typically a benchmark value from a theory or an expected sensitivity, like 1~pb. Hence, $H_\mu$ can be evaluated with a continuous spectrum of signal strengths and will approach the \acrshort{SMlabel} hypothesis ($H_0$) when $\mu\to0$.\\

Given a binned data distribution with $n_i$ events for a bin $i$, the expected value of $n_i$ can be expressed as,

\begin{equation}
    E[n_i(\mu,\mathbf{b},\boldsymbol{\theta})] = \mu\cdot s_i(\boldsymbol{\theta}) + \sum_{k_{\alpha}\in\mathbf{k}}k_\alpha\cdot b_{\alpha,i}(\boldsymbol{\theta})
\end{equation}

with $s_i$ the predicted signal events and $b_{\alpha,i}$ the predicted background events of the process $\alpha$. The normalisation factor $k_\alpha$ affects the background process $\alpha$, analogous to $\mu$. Typically, $k_\alpha$ is introduced only for the most relevant backgrounds. The rest of the processes are normalised to their predicted cross-sections and the corresponding $k_\alpha$ is fixed to one. The nuisance parameters $\boldsymbol{\theta}$ are additional degrees of freedom which correspond to the systematic uncertainties acting both on the shape and normalisation of all processes. Their central value is defined to be zero and the deviation with respect to the original value is referred to as pull, where a deviation of $\pm$1 corresponds to a variation of one standard deviation.\\

The fit procedure allows the reduction of the impact of systematic uncertainties, especially by taking advantage of the highly populated background-dominated bins included in the fit. This requires a good understanding of the background and the systematic effects. To verify the improved background prediction, fits under the background-only hypothesis are performed, and differences between the data and the post-fit background prediction are checked using selections and physical variables other than the ones used in the fit.\\

The binned likelihood function is given as
\begin{equation}
    \mathscr{L}(\mu,\mathbf{k},\boldsymbol{\theta}) = \prod_i^N \frac{ (E[n_i(\mu,\mathbf{b},\boldsymbol{\theta})])^{n_i}}{n_i!}e^{E[n_i(\mu,\mathbf{b},\boldsymbol{\theta})]}\prod_{\theta_j\in\boldsymbol{\theta}}P(\theta_j)
\end{equation}

which corresponds to a product of Poisson probabilities and the penalty terms of all nuisance parameters for all $N$ bins. The form $P(\theta_j)$ are generally Gaussian distributions for each systematic uncertainty. Poisson distributions are used for the statistical uncertainty of each bin, and are introduced in the likelihood to penalise large deviations.\\

The optimal $\mu$, $\mathbf{k}$ and $\boldsymbol{\theta}$ are obtained from the fit to data that maximises the agreement between data and the prediction.

The optimal test statistic to perform the fit is the likelihood ratio, %https://link.springer.com/chapter/10.1007/978-1-4612-0919-5_5

\begin{equation}
    \lambda_\mu = \frac{\mathscr{L}(\mu, \hat{\hat{\mathbf{k}}},\hat{\hat{\boldsymbol{\theta}}})}{\mathscr{L}(\hat{\mu}, \hat{\mathbf{k}},\hat{\boldsymbol{\theta}})}
\end{equation}

with the single-hat parameters being those maximising the likelihood while $\hat{\hat{\mathbf{k}}},\hat{\hat{\boldsymbol{\theta}}}$ the parameters that maximise the likelihood for a given $\mu$. As the likelihoods are products of several terms smaller than one, a more stable test statistic is the negative log-likelihood,

\begin{equation}
    q_\mu = -2\ln\lambda_\mu.
\end{equation}

For the purpose of setting upper limits on the signal production, some special cases are defined depending on $\mu$ and $\hat{\mu}$. If $\hat{\mu}$ is negative, i.e. the fitted signal has a negative normalisation, the modified test statistic assumes signal to be only positive: $\tilde{q}(\mu)=-2\ln\frac{\mathscr{L}(\mu, \hat{\hat{\mathbf{k}}},\hat{\hat{\boldsymbol{\theta}}})}{\mathscr{L}(0, \hat{\hat{\mathbf{k}}},\hat{\hat{\boldsymbol{\theta}}})}$, where the parameters in the denominator optimise the likelihood for $\mu=0$. Another exception is to set the modified test statistic to 0 for $\hat{\mu}>\mu$, as signal below the observed measurement is in complete agreement with the hypothesis.\\

The level of agreement between data and predictions for a given signal strength is quantified by computing the p-value $p_\mu$, which is the probability of the measured data being a deviation from the assumed $H_\mu$,

\begin{equation}
    p_\mu = \int_{q_{\mu,obs}}^\infty f(q_\mu|H_\mu)dq_\mu
\end{equation}

where $f(q_\mu|H_\mu)$ is the probability density function of $q_\mu$ under the assumption of $H_\mu$. The significance $Z=\Phi^{-1}(1-p_\mu)$ (being $\Phi$ the cumulative Gaussian distribution) is often preferred to quantify the level of disagreement in terms of the number of standard deviations. Typically, an alternative hypothesis is rejected at 1.64$\sigma$ ($p_\mu=0.05$) and the background-only at 5$\sigma$ ($p_0=2.87\cdot10^{-7}$).\\

Typically, searches are dedicated to very small signals that are difficult to separate from the background. Rejecting the null hypothesis at a fixed probability may result in excluding signals with low statistics not really searched for in the analysis~\cite{JUNK1999435}. The CL$_{s}$ method addresses this issue by defining,

\begin{equation}
    \text{CL}_{s}=\frac{p_\mu}{1-p_0},
\end{equation}

where $p_0$ is the p-value for the null hypothesis, and $p_\mu$ is the p-value for the signal hypothesis. The ratio normalises the p-value to the confidence level of the background-only hypothesis such that the CL incorporates the information of both hypotheses and by construction is less prone to false discoveries or exclusions. When the measurement is not compatible with the null hypothesis, the denominator is larger and CL$_{s}$ decreases. Therefore, the exclusion limits obtained using the CL$_s$ are conservative and a more cautious interpretation of disagreements with data.

