The study of particle physics is performed at the TeV scale, in the order of $10^{-15}$~m, which requires large and complex machines only possible within international collaborations. \acrshort{CERNlabel} is one of the biggest and most respected scientific collaborations and, since its origins in the 1950's, has hosted many groundbreaking experiments. The \acrlong{LHClabel}~(\acrshort{LHClabel}) is the world's largest particle accelerator, situated underground in the France-Swiss border and in operation since September 2008. The \acrlong{ATLASlabel}~(\acrshort{ATLASlabel}) detector is one of the experiments hosted within the \acrshort{LHClabel} and records the collisions for further data analysis. The work in this thesis is based on the recorded proton-proton collision data at a center-of-mass energy, $\sqrt{\text{s}}$, of 13~TeV between 2015 and 2018.
%https://iopscience.iop.org/article/10.1088/1748-0221/3/08/S08001
%https://iopscience.iop.org/article/10.1088/1748-0221/3/08/S08003
%https://iopscience.iop.org/article/10.1088/1748-0221/3/08/S08004
%https://iopscience.iop.org/article/10.1088/1748-0221/3/08/S08005
%https://iopscience.iop.org/article/10.1088/1748-0221/3/08/S08002
\section{The \acrshort{LHClabel}}

The \acrshort{LHClabel} is a circular particle accelerator with a circumference of 27 km, situated on an average of 100~m underground. The primarily activity is the collisions of protons, however proton-Pb and Pb-Pb collisions are also typically performed for one month a year. Particles are steered, collimated and boosted by different types of superconducting magnets and structures along the accelerator ring.

Proton beams circulate over different accelerators before reaching the \acrshort{LHClabel} and the optimal energy. \textcolor{red}{Figure} shows a schematic view of the \acrshort{CERNlabel} accelerator complex. First, protons are extracted from ionised hydrogen and accelerated up to 50~MeV with LINAC2, a linear accelerator. Then, protons are injected into the \acrlong{PSBlabel}~(\acrshort{PSBlabel}), an accelerator made of four synchrotron rings of 157~m in circumference, increasing the energy up to 1.4~GeV. Similarly, the protons are accelerated in sequence to 26~GeV and 450~GeV by the \acrlong{PSlabel}~(\acrshort{PSlabel}), a circular accelerator of 628~m in circumference, and the \acrlong{SPSlabel}~(\acrshort{SPSlabel}), of 6.9~km in circumference. Finally, the protons are injected to the two pipes of the \acrshort{LHClabel} and boosted up to 6.5~TeV before collision. For the Pb operations, the extraction and accelerators prior to the \acrshort{SPSlabel} are performed instead using LINAC3 and the \acrlong{LEIRlabel}~(\acrshort{LEIRlabel}).

Inside the \acrshort{LHClabel}, two particle beams travel close to the speed of light before they are made to collide. The two separated particle beam pipes are designed to operate at 6.5~TeV in opposite directions and kept at ultrahigh vacuum, down below $10^{-13}$ atmospheres. Surrounding the pipes, superconducting magnets built from niobium-titanium alloy coils generate strong magnetic fields of the order of 8~T through an electric current of 11.8 kA. The magnet coils are surrounded by the magnet yoke, tones of solid steel sheets designed to keep the wiring firmly in place and stabilise the temperature of the magnets. The magnets are cooled down to 1.9~K with superfluid helium provided by a cryogenic system requiring 120 tonnes of helium. The rest of external layers are dedicated to shield the particle radiation, insulate the magnet or maintain the vacuum and the whole structure of up to 28 tonnes. Different varieties and sizes of magnets constitute the accelerator, mainly 1232 dipole magnets 15~m in length which bend the particle beams to follow the circular trajectory, and 392 quadrupole magnets, each 5-7~m long, which focus the beam. Other types of magnets are used to correct the beam shape or to align them for collision. \textcolor{red}{Figure} shows the cross-section of a dipole magnet of the \acrshort{LHClabel}, with the different components.

Particles in each beam pipe are accelerated by 8 superconducting \acrlong{RFlabel}~(\acrshort{RFlabel}) cavities, metallic chambers with alternating electric fields housed in cryogenic chambers, which also space the particles into compact groups named bunches. When protons are accelerated, the bunches contain more than $10^11$ protons spaced every 25~ns (around 7 meters).

The particles are brought to collision at interaction points (IPs) by superconducting multiple magnets focusing the beams. Four detectors are situated at the different IPs: \acrshort{ATLASlabel}, CMS, LHCb and ALICE. The first two are multi-purpose experiments that study a wide range of physics, comprising precision measurements of the \acrshort{SMlabel} as well as searches for beyond the SM such as Supersymmetry, exotic particles or dark matter. Both collaborations are formed by around 3000 scientist each, the two largest at \acrshort{CERNlabel}. The LHCb experiment is dedicated to explore hadrons containing b- or c-quarks, especially investigating CP-violating processes. The ALICE experiment is the only experiment fully focused on heavy-ion collisions and therefore specialised on QCD physics. 

The \acrlong{LEPlabel}~(\acrshort{LEPlabel}) was the previous main experiment at \acrshort{CERNlabel} and its operations finished in 2000 to start the \acrshort{LHClabel} installation in the same tunnel, replacing the predecessor. \acrshort{LEPlabel} was designed to collide $e^+e^-$ beams and operated at a maximum of $\sqrt{\text{s}}=209$~GeV. \acrshort{LHClabel} was designed to accelerate protons or lead ions, which in comparison, are easier to accelerate to higher energies and provide more collision data, although harder to be detected and studied. \acrshort{LEPlabel} explored the \acrshort{EW} scale and provided precision measurements of the \acrshort{SMlabel} and set a lower bound for the mass of the Higgs boson, later discovered during \acrshort{LHClabel} operation in 2012. In September 2008, the first \acrshort{LHClabel} operations started and in November 2009 the first collision was produced. 

\subsection{Performance in Run 2}

The number of events of a certain process is key for its study and can be written as

\begin{equation}
    N = \sigma\mathscr{L} = \sigma \int \mathcal{L} \text{dt}
\end{equation}

where $\sigma$ is the event cross-section for the given process, $\mathscr{L}$ the integrated luminosity and $\mathcal{L}$ the instantaneous luminosity. The cross-section highly depends on the center-of-mass energy, $\sqrt{s}$, one of the main characteristics of a particle collider. As a general rule, the higher the $\sqrt{s}$, the higher is the $\sigma$ for rare \acrshort{SMlabel} processes, interesting for precision measurements, or searches for new massive particles.

The instantaneous luminosity is another of the main characteristics of a particle collider, which for the \acrshort{LHClabel} can be approximated to

\begin{equation}
    \mathcal{L} = f \frac{n_1n_2}{4\pi\sigma_x\sigma_y} F
\end{equation}
%https://cds.cern.ch/record/941318
%https://cds.cern.ch/record/2677054/

with $f$ the revolution frequency, $n_{1,2}$ the total number of protons in each beam and $F$ a reducing factor accounting for the beams are not colliding exactly head-on and other geometric and beam effects\todo{Talk more in detail about the effects?}. The first parameter can be approximated to $f=c/27\text{ km} = 11$~kHz and the total number of protons from the nominal number of bunches, 2808, which can contain up to $10^{11}$~protons. Finally, the denominator is the approximated transverse beam area with transverse beam size $\sigma_{x,y} = 16.6$~$\mu$m. With these assumptions, the instantaneous luminosity is of $\mathcal{O}(10^{34}$~cm$^{-2}$s$^{-1})$.

During 2010 and 2011, the \acrshort{LHClabel} delivered proton collisions at $\sqrt{s}=7$~TeV, while in 2012 were at $\sqrt{s}=8$~TeV. The first proton physics run, namely Run~1, ended in February 2013, which enabled the discovery of the Higgs boson with $\mathscr{L}\sim30$~fb$^{-1}$. The evolution of the integrated luminosity delivered of the Run~2 to the \acrshort{ATLASlabel} experiment is shown in \textcolor{red}{Figure} for a total of $\mathscr{L}=139$~fb$^{-1}$, used in this thesis.

Another parameter of interest is the \textit{pile-up}, which are the additional expected inelastic collisions that occur when crossing bunches of protons. The main source are the collisions that appear within a single bunch crossing, called in-time pile-up. In addition, out-of-time pile-up is referred to interactions from neighboring bunch crossings not resolved fast enough by the detectors. Pile-up effects are a challenge for physics analysis and is inherent to the increase of instantaneous luminosity. \todo{should I include the formula?}The mean interactions per crossing, $\langle\mu\rangle$, is a measure to quantify the pile-up and changes throughout the data taking periods, as shown in \textcolor{red}{Figure}.

The \acrshort{LHClabel} operation parameters have changed every year and some are even above the original designed. \textcolor{red}{Table} summarises some of the parameters of the \acrshort{LHClabel} original design and through different years.

% Parameter 2011 2012 2015 2016 2017 2018
% center-of-mass energy [TeV] 7 8 13 13 13 13
% Av. interactions/crossing 9.1 20.7 13.4 25.1 37.8 36.1
% Bunch spacing [ns] 50 50 25 25 25 25
% Peak luminosity [cm−2
% s
% −1
% ] 3.65 7.73 5.03 13.8 20.9 21.0
% Integrated luminosity 5.5 23.1 4.2 38.5 50.2 63.4

% Parameters Design value 2010 2011 2012
% Beam energy (TeV/c) 7 3.5 3.5 4
% Beta function β
% ∗
% (m) 0.55 2.0/3.5 1.5/1.0 0.6
% Bunch spacing (ns) 25 150 75/50 50
% Max. # bunches/beam 2808 368 1380 1380
% Max. # protons/bunch 1.15 × 1011 1.2 × 1011 1.45 × 1011 1.7 × 1011
% Peak luminosity (cm−2
% s
% −1
% ) 1.00 × 1034 2.1 × 1032 3.70 × 1033 7.7 × 1033
% Emittance (µrad) 3.75 2.0 2.4 2.5
% Max. hµi 19 4 17 37

\section{The \acrshort{ATLASlabel} experiment}

The \acrshort{ATLASlabel} detector is a multi-purpose particle detector, used to study a wide range of physics topics. It is installed 100~m underground at IP-1 of the \acrshort{LHClabel}. Being 25~m in diameter, 44~m in length and 7000~tonnes in weight, \acrshort{ATLASlabel} is the largest particle detector ever built at a collider. As illustrated in \textcolor{red}{Figure}, the detector has a cylindrical shape and is composed of several detector layers built around the collision point of the particles, with an almost full solid angle coverage.

The data recorded by the detector is used by the collaboration in an extensive physics program. One of the main branches is the measurement of \acrshort{SMlabel} processes, as the large luminosity provided by the \acrshort{LHClabel} allows both precision measurements or to target uncovered phenomena. The data collected during Run~2 allows the study of the Higgs boson and its properties, heavily scrutinised since the first observation of the particle. In addition, measurements of the interactions and processes involving the top-quark are particularly interesting to probe the \acrshort{SMlabel} and \acrshort{BSMlabel} theories. A large portion is dedicated to a wide range of \acrshort{BSMlabel} theories that includes supersymmetry, dark matter or additional resonances among others. Finally, the physics program also includes the study of the physics involving $b-$/$c-$quarks and heavy ions.

All detector systems are designed such that they provide optimal performance to fulfill the different physics targets. For that, it is important that the detector can identify particles ranging from few GeV to several TeV with outstanding efficiency and resolution, providing accurate measurements of position and momentum. On the other hand, the conditions of the \acrshort{LHClabel} are extreme and the electronics installed have to be highly resistant to radiation and fast for readout (25~ns between interactions).

\subsection{Coordinate System}

The convention to describe the particles recorded with the \acrshort{ATLASlabel} detector is a right-handed coordinate system as illustrated in \textcolor{red}{Figure}, with the origin at the center of the detector which is also the interaction point. The $z$-axis is defined in the counterclockwise direction along the \acrshort{LHClabel} beamline, the $y$-axis points towards the surface and the $x$-axis towards the center of the ring defined by the accelerator. To describe the physics objects within the detector, spherical coordinates are use instead, with the polar angle, $\theta$, measured from the $z$-axis while the azimuthal angle, $\phi$, measured from the $x$-axis in the $x-y$ plane. The pseudorapidity, $\eta$, is usually used instead of the polar angle, as it transforms easily under relativistic boosts along the $z$-axis. For particles with energy $E$ and forward momentum $p_z$, the expression of $\eta$ can be found from the rapidity, $y$, in the high-energy approximation,

\begin{equation}
   y \equiv \frac{1}{2}\ln\left(\frac{E+p_z}{E-p_z}\right) \xrightarrow[ \frac{m}{E} \to 0]{} -\ln\left(\tan\frac{\theta}{2}\right)\equiv \eta
\end{equation}

\acrshort{ATLASlabel} covers the pseudorapidity region up to $|\eta|<4.9$, although physics analyses
typically consider objects restricted to $|\eta|<2.5$. In addition, the difference in $\eta$ between two points, $\Delta\eta$, is invariant under Lorentz transformation, thus angular distances can be described in the $\eta-\phi$ plane as,

\begin{equation}
    \Delta R = \sqrt{\left(\Delta\eta\right)^2+\left(\Delta\phi\right)^2}
\end{equation}

Another useful expression is the momentum in the $x-y$ plane, 

\begin{equation}
    \vec{p_\text{T}} = \begin{pmatrix} p_x \\ p_y \end{pmatrix}, p_\text{T} = \sqrt{\left(p_x\right)^2+\left(p_y\right)^2}
\end{equation}

as at the time of the collision the particles are made to collide along the $z$-axis, the initial momentum of the transverse plane is known to be zero due to conservation of energy.

\subsection{The Inner Detector}

The \acrlong{IDlabel}~(\acrshort{IDlabel}) [54, 55] is the innermost detector system, which encloses the beam pipe. This detector system provides precise tracking information of charged particles with momentum as low as 100~MeV with a $|\eta|<2.5$ coverage. \textcolor{red}{Figure} shows an overview of the system, which is structured into three sub-detectors: the pixel detector, the \acrlong{SCTlabel}~(\acrshort{SCTlabel}) and the \acrlong{TRTlabel}~(\acrshort{TRTlabel}).

%: Overview of the inner detector of the ATLAS experiment divided into three sub-detectors (Pixels,SCT and TRT)

% https://cds.cern.ch/record/331063
% https://cds.cern.ch/record/331064
% https://arxiv.org/abs/1608.07850

\subsubsection*{Pixel Detector}

The inner part of the \acrshort{IDlabel} is the silicon pixel detector comprising 4 cylindrical layers and 2 end-caps with 3 disc layers each. The layers are located between 33.25~mm to 122.5~mm around the beam pipe with a coverage of $|\eta|$ < 2.5. A single 3D pixel is a radiation-hard silicon detector that produces a small measurable current when a charged particles passes through. The detector is especially important for the reconstruction of tracks, path of charged particles; the primary vertex reconstruction, the position of the main energetic collision; as well as for secondary vertex finding. The \acrlong{IBLlabel}~(\acrshort{IBLlabel}) is the innermost layer, installed in-between Run~1 and Run~2, having the highest granularity with a pixel size of $50\times 250$~$\mu$m (50~$\mu$m in the $\phi$-direction and 250~$\mu$m in $z$-direction) for a total of 12~M pixels. In particular, the \acrshort{IBLlabel} is very efficient to reconstruct secondary vertices, which are key signatures of long-lived particles decays and crucial for the identification of $b$-hadrons. Furthermore, the three remaining layers have a pixel size of $50\times 400$~$\mu$m. Overall, the pixel detector contains 86~M pixels with an expected hit resolution of $8\times 40$~$\mu$m for the \acrshort{IBLlabel} and $10\times 115$~$\mu$m for the rest of pixel layers. In addition, the system makes up around 50\% of all \acrshort{ATLASlabel} readout channels. For the next upgrade, the \acrlong{HLLHClabel}~(\acrshort{HLLHClabel}), a new fully silicon-based \acrlong{ITklabel}~(\acrshort{ITklabel}) will replace the full \acrshort{IDlabel}.

% https://cds.cern.ch/record/1291633
% https://cds.cern.ch/record/2285585

\subsubsection*{Semiconductor tracker}

The \acrlong{SCTlabel}~(\acrshort{SCTlabel}) is a silicon strip detector comprising 4 double layers in the barrel region and nine planar end-cap discs on each side, installed around the pixel detector. The planar strips technology is simpler compared to the silicon pixels, for lower resolution covering a larger area. The strips have a size of $80~\mu$m$\times 12$~cm and cover a region up to $|\eta|<2.5$. The two layers within one layer-module are tilted by 40~mrad. Overall, the \acrshort{SCTlabel} has a resolution of $17\times 580$~$\mu$m with a total of 6.3~M readout channels. In general, the semiconductor-based detectors in \acrshort{ATLASlabel} operate at a temperature between -10 ºC and -5 ºC to suppress different types of electronic noise.


\subsubsection*{Transition Radiation Tracker}

The outermost part of the \acrshort{IDlabel} is the \acrlong{TRTlabel}~(\acrshort{TRTlabel}). In contrast to the others, the \acrshort{TRTlabel} is not based on silicon but is a gaseous detector system. It consists of around 300~k straw tubes with a diameter of 4~mm filled with a gas mixture~\footnote{In Run~2, a mixture of Ar (80\%) and CO$_2$ (20\%) was used instead in modules with gas leaks} of Xe (70\%), CO$_2$ (27 \%) and O$_2$ (3 \%) and with a gold-plated tungsten wire in the tube centre with a potential different to the tube surface of 1.5~kV. When a charged particle hits the tube, the ionisation of the gas is detected as the signal. The straws have a length of 144~cm in the barrel region and 37~cm in the end cap, while the single hit resolution is 120~$\mu$m and 130~$\mu$m, respectively. The \acrshort{TRTlabel} only provides tracking information in the $\phi$ direction, as the tubes are parallel to the beamline. Besides, the \acrshort{TRTlabel} provides particle identification from emitted transition radiation at the material boundaries, since the straws are interleaved with polypropylene. Especially, electrons can be distinguished from charged pions due to larger transition radiation.

% https://arxiv.org/abs/1702.06473

\subsection{The Calorimeter System}

The calorimeter system is responsible for the precise measurement of the energy carried by both charged and neutral particles as well as measuring shower properties to allow for particle identification. Showers are cascades of secondary particles which are formed when a highly energetic particle interacts with dense material. \acrshort{ATLASlabel} uses sampling calorimeters which consist of alternating layers of active material (liquid argon and plastic scintillators) and passive detector material (copper, iron, tungsten and lead). While the active material measures the energy deposit of the particles going through, the passive material is designed to interact and absorb particles, thus induces the showering. The calorimeter system covers the region $|\eta|$ < 4.9 and is places between the central solenoid and the muon spectrometers. \textcolor{red}{Figure} shown an overview of the system, which is composed of two sub-detectors: the electromagnetic [61, 62] and the hadronic calorimeter [63, 64]. 

% https://cds.cern.ch/record/331061
% http://cds.cern.ch/record/2285582
% https://cds.cern.ch/record/331062
% https://iopscience.iop.org/article/10.1088/1748-0221/3/02/P02010

\subsubsection*{Electromagnetic Calorimeter}

The \acrlong{EMlabel}~(\acrshort{EMlabel}) calorimeter encloses the \acrshort{IDlabel} and is a high granularity calorimeter based on \acrlong{LARlabel}~(\acrshort{LARlabel}) technology with absorber plates made out of lead. To provide full coverage in $\phi$, the \acrshort{EMlabel} calorimeter has an accordion-shaped structure where the active material is placed in the gaps between the lead absorber plates and the Kapton electrodes. The detector operates at -183 ºC with a total of 170~k readout channels. The barrel region of the \acrshort{EMlabel} calorimeter covers the region $|\eta|$ < 1.475 and consists of three layers with a 4~mm gap between them and a length of 3.2~m each, with decreasing granularity. The layer closes to the \acrshort{IDlabel} has a granularity of $\Delta\eta\times\Delta\phi= 0.0031\times 0.098$, while the second layer $\Delta\eta\times\Delta\phi= 0.025\times 0.025$ and the outermost $\Delta\eta\times\Delta\phi= 0.05\times 0.025$. In addition, the two end-caps cover the region $|\eta|$ < 3.2 with a slightly coarser granularity. In general, the absorption power at high energies of a calorimeter is quantified by means of the radiation length $X_0$ of its medium. It is defined as the distance over which the particle energy is reduced via radiation losses by a factor $1/e$. The thickness of the barrel region, given in terms of the radiation length, is 22 $X_0$ and 24 $X_0$ for the end-caps. Moreover, the intrinsic energy resolution of the \acrshort{EMlabel} calorimeter is,

\begin{equation}
    \frac{\sigma_E}{E} = \frac{10\%}{\sqrt{E}}\oplus\frac{17\%}{E}\oplus 0.7\%
\end{equation}

% https://hep.uchicago.edu/images/Calorimeters_labels.png

\subsubsection*{Hadronic Calorimeter}

The second calorimeter system is the hadronic calorimeter located around the \acrshort{EMlabel} calorimeter and consists of three components providing around 19000 readout channels. First, the tile calorimeter is made out of alternating layers of steel as absorber material and scintillator plastic tiles as active material, being read out via photomultiplier tubes. The first two layers have the highest granularity with $\Delta\eta\times\Delta\phi= 0.1\times 0.1$. The barrel part of the tile calorimeter covers a region with $|\eta|$ < 1.0 and the two extended barrels a range of 0.8 < $|\eta|$ < 1.7. The resolution of the tile calorimeter is $\sigma$E/E = 50\%/$\sqrt{E}\oplus$ 3\% \todo{check this numbers} [66, p. 3].
Next, the end-cap calorimeters are directly outside the \acrshort{EMlabel} calorimeter and are based on the \acrshort{LARlabel} technology. The end-caps use copper as passive material and cover a region of 1.5 < $|\eta|$ < 3.2 with their highest granularity of also $0.1\times 0.1$ within $|\eta|$ < 2.5. Finally, the forward calorimeter is also \acrshort{LARlabel} based and its first layer uses copper as absorber, which provides information for both electromagnetic and hadronic particles. The other two layers make use of tungsten as absorber which is better suitable for hadronic measurements. In total the forward calorimeter covers a region of 3.2 < $|\eta|$ < 4.9. The overall resolution of the \acrshort{LARlabel} based hadronic calorimeters is,
\begin{equation}
    \frac{\sigma_E}{E} = \frac{50\%}{\sqrt{E}}\oplus\frac{1\%}{E}\oplus 3\%
\end{equation}


$\sigma$E/E = 100\%/$\sqrt{E}\oplus$ 10\% [67, p. 2].

%https://doi.org/10.1140/epjc/s10052-018-6374-z

\subsection{Muon Spectrometer}

Muons mostly traverse the detector without losing energy. Hence they are identified in the muon spectrometer (MS) [68] which is the outermost detector system of ATLAS with a distance to the beam of 5-10 m (see Figure 3.8). It consists of four detector systems grouped into trigger and precision muon tracking chambers. In total the MS has more than one million readout channels and is embedded in three superconducting toroidal magnets (one in the barrel and one at each end-cap), providing a magnetic field in $\phi$-direction (typically perpendicular to the muon trajectory). The muon system is not entirely symmetric in $\phi$ due to some gaps for detector services and support structure (detector feet). The momentum resolution of the MS is around 10\% for 1 TeV muons and around 3\% for 10-200 GeV muons.

% https://cds.cern.ch/record/331068
% https://arxiv.org/abs/0910.2767

\subsubsection*{Muon Trigger Chambers}

The muon trigger chambers are designed for a fast readout to provide trigger information. In the barrel region with $|\eta|$ < 1.05, three layers of resistive plate chambers (RPCs) are used. The RPCs are made out of parallel plates with a high resistivity and a potential difference between them where the gap is filled with a gas mixture (94.7\% C2H2F4, 5\% Iso-C4H10, 0.3\% SF6). Besides the trigger information the RPCs also provide an $\eta-\phi$ measurement with a spatial resolution of 10 mm. In the end-caps (1.05 < $|\eta|$ < 2.4) multi-wire chambers filled with a gas mixture of 55\% CO2 and 45\% n-C5H12 are used. These are called thin gap chambers (TGCs). They use graphite-coated cathodes and the wires are separated by 1.8 mm. Apart from the trigger information, the TGCs provide $\phi$ information with a resolution of 5 mm.

\subsubsection*{Precision Muon Tracking Chambers}

While the muon trigger chambers are always read out, the precision muon tracking chambers are
only read out when a trigger decision was made since the detector technology is slower but provides a high resolution and precision tracking information. The monitored drift tubes (MDTs) are installed in the barrel and end-cap region covering $|\eta|$ < 2.7. MDTs are aluminium drift tubes with a diameter of 3 cm filled with an Ar/CO2 (93/7\%) gas mixture and a wire in their centres. Typically, each chamber contains 3-8 layers of drift tubes resulting in a spatial resolution of 35 µm [46, p. 165]. Cathod strip chambers (CSCs) are installed in the forward region (2.0 < $|\eta|$ < 2.7) and are proportional multi-wire chambers (as the TGCs), providing a radial resolution of 40 µm and a resolution in $\phi$ of 5 mm [46, p. 165].

% https://doi.org/10.1088/1748-0221/3/08/s08003

\subsection{Magnet System}

Besides the detector systems, the magnet system is of major importance to allow momenta and charge measurements. It bends the trajectory of charged particles via the Lorentz force depending on their momentum and charge and consists of two sub-systems. Firstly, the central solenoid magnet located between the ID and the calorimetry generates a constant magnetic field of 2 T. The superconducting magnet made out of NbTi is cooled via liquid helium to a temperature of 1.8 K. Secondly, the vast toroidal magnet system embedded in the MS comprising one barrel toroid and two end-cap toroids with eight coils each. The toroidal magnets deliver an inhomogeneous magnetic field of roughly 0.5 T and 1 T in the central and end-cap regions, respectively.

\subsection{Trigger System and Data Acquisition}

In order to handle the high event rates, which are expected to be 40 MHz for ATLAS corresponding to more than 40 TB/s of data, a trigger system is required to reduce the amount of data to be recorded without losing important information. Since RUN II the trigger system is structured into two parts, the Level-1 (L1) hardware trigger and the software-based high level trigger (HLT) [70, 71] sketched in Figure 3.9. The L1 trigger uses information from the RPCs, TGCs and the calorimeter to identify high pT electrons, muons, photons, jets and high missing transverse momentum. It has a very fast latency of 2.5 µs and reduces the rate to 100 kHz. The L1 trigger identifies regions of interest (RoIs) in $\eta$ and $\phi$ and passes this information to the HLT. The HLT is fully software based and uses the full detector information within the RoIs to reduce the event rate down to approximately 1 kHz with a latency of 200 ms. Afterwards, the data is transferred to a computing centre for further processing and storage.

% https://cds.cern.ch/record/616089
% https://cds.cern.ch/record/2133909