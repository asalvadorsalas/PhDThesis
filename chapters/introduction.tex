\setchapterpreamble[u]{\margintoc}
\chapter*{Introduction}\addcontentsline{toc}{chapter}{Introduction}
\labch{intro}

The discovery of the Higgs boson in 2012 by ATLAS and CMS [] is one of the most recent historic milestones in the field of particle physics. CERN hosts the LHC, which physics' program included the hunt of the Higgs boson. After this achievement, all particles predicted by the Standard Model are discovered, the theory which describes the fundamental particles and their interactions. Nevertheless, the ATLAS experiment continues to scrutinise the Standard Model of particle physics by analysing the ever-increasing amount of particle collisions delivered by the LHC. There are many phenomena not covered by the current theory and any measurement that deviates from the predictions or any hint of a new particle will lay the foundation for a new path in particle physics.

The theory of the Standard Model has successfully guided the experiments with the prediction of particles and precise values of their interactions. However, no other theory has been able to explain the tiny deviations with respect to the theory or filled the gaps that the Standard Model does not even contemplate. Known tensions with respect to the theory predictions are regarding the lepton universality [] or the measurement of the anomalous magnetic dipole momentum of the muon (g-2) []. On the other hand, the theory fails to cover gravity, the neutrino masses, dark matter... although one of the most compelling concerns regarding the Standard Model is known as the hierarchy problem, the seemingly unnatural fact that the Higgs mass, yet not having any constrain in the theory, appears to be at the electroweak scale. One of the possible theoretical solutions consinsts in an expansion of the Standard Model which spawns additional scalar particles. If the Higgs sector is built with one extra doublet (Two Higgs Doublet Models), a total of five scalars are predicted instead, including Higgs bosons with electrical charge.
Another similar feature of the Standard Model is known as the flavour problem, as fermions can be grouped in three families with different mixing patterns, which is also seen as arbitrary choice. This feature can be explained by introducing a broken flavour symmetry spawning a new particle, the flavon, which introduced flavour violating interactions. The presence of flavour-changing neutral current (FCNC) interactions is heavily suppressed in the Standard Model, way below the available sensitivity. These interactions are hence very sensitive to new physics as any observed interaction cannot be explained by the Standard Model.

In this thesis, a direct search for charged Higgs bosons heavier than the top quark and a direct search for neutral scalars lighter than the top quark are presented. The charged Higgs process is searched in the $200-2000$~GeV mass range produced in association with top and bottom quarks and decaying into a top-bottom, while the neutral scalar is searched in the $20-160$~GeV mass range produced in the FCNC decay of a top quark involving a $c$- or a $u$-quark, and finally decaying to a pair of $b$-quarks. Limits on the production of charged Higgs bosons in the same channel have been previously obtained by ATLAS with only the data from 2015 and 2016 for $H^\pm\to tb$ in the $180-600$~GeV mass range [], and more recently by CMS in the $200-600$~GeV mass range using the full Run-2, setting upper limits on the production cross section of X--X pb and X--X pb respectively. On the other hand, both ATLAS and CMS have searched for the top FCNC decay into the SM Higgs, $t\to qH$ with $q$ being either a $c$- or $u$-quark, setting upper limits on the branching fraction of XXX and XXX, with the most sensitive single being with $H\to\gamma\gamma$. However, the generic signature of a new scalar lighter than the top quark presented in this work is uncovered in literature.

Both searches performed in this thesis are performed using the full Run-2 proton-proton collisions collected by the ATLAS experiment from 2015 to 2018 at a center-of-mass energy of 13~TeV. Events are selected to have either one electron or muon and multiple jets, including those originated from the hadronisation of a bottom quark. Results are extracted by means of binned maximum-likelihood fits of the different simulated signal and SM backgrounds to the recorded data. The fit is performed on a discriminant obtained by combining several kinematic variables through the training of parameterised feed-forward neural networks, developed to optimise the sensitivity by separating signal and background events.

The document is structured into three main parts: The first part describes the theoretical and experimental setup, while the second part includes the $H^\pm\to tb$ analysis while the third part includes the $t\to qX$ analysis, both with a detailed description of the strategy and their results. Chapter~\ref{chapter:SM} focuses on the Standard Model and the models that motivate the searches. Chapter~\ref{chapter:ATLASLHC} provides an overview of the LHC and the ATLAS experiment. Chapters~\ref{chapter:EventSim} and~\ref{chapter:EventReco} present the essential aspects of the simulation and reconstruction of simulated proton-proton collisions, while Chapter~\ref{chapter:ML} presents the machine learning techniques and statistical tools used in the different analyses. \todo{tocomplete} %The second part includes the $H^\pm\to tb$ analysis while the third part includes the $t\to qX$ analysis, both with a detailed description of the strategy and their results. For the $H^\pm\to tb$, the limits are interpreted in the context of different benchmark models, including a combination with dark matter models. Regarding $t\to qX$, the measurement of the process involving the Standard Model Higgs is also presented.