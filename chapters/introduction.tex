\setchapterpreamble[u]{\margintoc}
\chapter*{Introduction}
\addcontentsline{toc}{chapter}{Introduction}
\markboth{Introduction}{Introduction}

The discovery of the Higgs boson in 2012 by ATLAS and CMS~\cite{ATLASHiggs2012,CMShiggs2012} is one of the most recent historic milestones in the field of particle physics. CERN hosts the LHC, whose physics programme included the hunt for the Higgs boson. After this achievement, all particles predicted by the Standard Model (SM), the theory which describes the fundamental particles and their interactions, have been discovered. Nevertheless, the ATLAS experiment continues to scrutinise the Standard Model of particle physics by analysing the ever-increasing amount of particle collisions delivered by the LHC. There are many phenomena not covered by the current theory and any measurement that deviates from the predictions or reveals a hint of a new particle could pave the way for new discoveries in particle physics.\\

The theory of the Standard Model has successfully guided the experiments with the prediction of particles and their interactions. However, the theory does not address gravity, the non-zero neutrino masses, the existence of dark matter, or other phenomena. One significant concern with the Standard Model is the hierarchy problem, which refers to the apparent unnaturalness of the Higgs mass being at the electroweak scale despite the inexistence of a constraint on its value within the theory. One possible theoretical solution involves expanding the Standard Model to include additional scalar particles. In Two Higgs Doublet Models (2HDM), the Higgs sector is built with one extra doublet and a total of five particles are predicted instead of a single Higgs boson, and includes Higgs bosons with electrical charge.
Another feature of the Standard Model is the so-called flavour problem, where fermions can be grouped in three families with different mixing patterns, and this is seen as an arbitrary choice. The flavour problem can be addressed by introducing a new particle called the flavon, which arises from a broken flavour symmetry and introduces flavour-violating interactions. Flavour-changing neutral current (FCNC) interactions are heavily suppressed in the Standard Model and fall below the sensitivity of current experiments. These interactions are hence very sensitive to new physics as they can be enhanced with new interactions outside the Standard Model.\\

This thesis presents a direct search for charged Higgs bosons with masses higher than the top quark mass and a direct search for neutral scalars with masses lighter than the top quark. The charged Higgs process is searched in the $200-2000$~GeV mass range, produced in association with top and bottom quarks and decaying into a top-bottom pair. Limits on the production of charged Higgs bosons in the same channel have been previously obtained by ATLAS with only the data from 2015 and 2016 in the same mass range~\cite{ATLASHptb2018}, and more recently by CMS in the $200-3000$~GeV mass range using the full Run-2 data, setting upper limits at 95\% confidence level on the production cross-section of $2.9-0.070$~pb and $9.6-0.01$~pb, respectively. The neutral scalar is searched for in the $20-160$~GeV mass range, produced in a FCNC decay of a top quark involving a $c$- or a $u$-quark, and finally decaying into a pair of b-quarks. This is the first time that either ATLAS or CMS perform this physics search. However, both experiments have searched for the top FCNC decay into the SM Higgs, $t\to qH$ with $q$ being either a $c$- or a $u$-quark. The most recent analysis from ATLAS is in the $H\to\tau\tau$ channel~\cite{ATLAStqHtautau} while the CMS results with 137~fb$^{-1}$ data combines several channels and sets limits to B($t\to uH$)$ < 0.079$ and B($t\to cH$)$ < 0.094$~\cite{CMStqHRun2}.\\

Both searches in this thesis use the full Run-2 proton-proton collisions collected by the ATLAS experiment from 2015 to 2018 at a center-of-mass energy of 13~TeV. Events are required to have either one reconstructed electron or muon and multiple jets, including those originated from the hadronisation of a bottom quark. Results are obtained by performing binned maximum-likelihood fits of the different simulated signal samples and SM backgrounds to the recorded data. The fits are performed using discriminants obtained by combining several kinematic variables through parameterised feed-forward neural networks, developed to optimise the sensitivity to separate signal and background events.\\

This document is structured into three main parts: the first part describes the theoretical and experimental setup, while the second and third parts include the $H^\pm\to tb$ and $t\to qX$ analyses, respectively, both with a detailed description of the strategy and their results. Chapter~\ref{chapter:SM} focuses on the Standard Model and the models that motivate the searches. Chapter~\ref{chapter:ATLASLHC} provides an overview of the LHC and the ATLAS experiment. Chapters~\ref{chapter:EventSim} and~\ref{chapter:EventReco} present the main aspects of the simulation and reconstruction of simulated proton-proton collisions, while chapter~\ref{chapter:MLStat} presents the machine learning techniques and statistical tools used in the different analyses. Chapter~\ref{chapter:Htbanalysis} introduces and explains the details of the $H^\pm\to tb$ analysis, with the results summarised in chapter~\ref{chapter:Htbresults}. Similarly, the details and results of the $t\to qX$ search analysis are discussed in chapters~\ref{chapter:tqXanalysis} and~\ref{chapter:tqXresults}. A summary and conclusions of the work are provided in chapter~\ref{chapter:summary}.