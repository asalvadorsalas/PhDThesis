\setchapterpreamble[u]{\margintoc}
\chapter*{Introduction}\addcontentsline{toc}{chapter}{Introduction}
The discovery of the Higgs boson in 2012 by ATLAS and CMS~\cite{ATLASHiggs2012,CMShiggs2012} is one of the most recent historic milestones in the field of particle physics. CERN hosts the LHC, which physics program included the hunt of the Higgs boson. After this achievement, all particles predicted by the Standard Model (SM) have been discovered, the theory which describes the fundamental particles and their interactions. Nevertheless, the ATLAS experiment continues to scrutinise the Standard Model of particle physics by analysing the ever-increasing amount of particle collisions delivered by the LHC. There are many phenomena not covered by the current theory and any measurement that deviates from the predictions or reveals a hint of a new particle could pave the way for new discoveries in particle physics.\\

The theory of the Standard Model has successfully guided the experiments with the prediction of particles and precise values of their interactions. However, the theory does not address gravity, the neutrino masses, dark matter, and other phenomena. One of the most significant concerns with the Standard Model is the hierarchy problem, which pertains to the apparent unnaturalness of the Higgs mass being at the electroweak scale despite no constraint on its value within the theory. One possible theoretical solution involves expanding the Standard Model to include additional scalar particles. In Two Higgs Doublet Models (2HDM), the Higgs sector is built with one extra doublet and a total of five scalars are predicted instead, including Higgs bosons with electrical charge.
Another similar feature of the Standard Model is known as the flavour problem, as fermions can be grouped in three families with different mixing patterns, which is also seen as an arbitrary choice. The flavour problem can be addressed by introducing a new particle called the flavon, which arises from a broken flavour symmetry and introduces flavour-violating interactions. Flavour-changing neutral current (FCNC) interactions are heavily suppressed in the Standard Model and fall below the sensitivity of current experiments. These interactions are hence very sensitive to new physics as any observed interaction cannot be explained by the Standard Model.\\

This thesis presents a direct search for charged Higgs bosons with masses greater than the top quark and a direct search for neutral scalars with masses lighter than the top quark. The charged Higgs process is searched in the $200-2000$~GeV mass range, produced in association with top and bottom quarks and decaying into a top-bottom pair. The neutral scalar is searched for in the $20-160$~GeV mass range, produced in the FCNC decay of a top quark involving a $c$- or a $u$-quark, and finally decaying into a pair of b-quarks. Limits on the production of charged Higgs bosons in the same channel have been previously obtained by ATLAS with only the data from 2015 and 2016 for $H^\pm\to tb$ in the $200-2000$~GeV mass range~\cite{ATLASHptb2018}, and more recently by CMS in the $200-3000$~GeV mass range using the full Run-2, setting upper limits at 95\% confidence level on the production cross-section of $2.9-0.070$~pb and $9.6-0.01$~pb, respectively. In contrast, both ATLAS and CMS have searched for the top FCNC decay into the SM Higgs, $t\to qH$ with $q$ being either a $c$- or $u$-quark. Both ATLAS and CMS have performed multiple measurements on this process, the most recent from ATLAS being in the $H\to\tau\tau$ channel~\cite{ATLAStqHtautau} while the CMS results with 137~fb$^{-1}$ data combines several channels and sets the limits to $t\to uH < 0.079$ and $t\to cH < 0.094$~\cite{CMStqHRun2}. However, these results are for the SM Higgs and the generic signature of a scalar lighter than the top quark presented in this work is uncovered in literature.\\

Both searches in this thesis use the full Run-2 proton-proton collisions collected by the ATLAS experiment from 2015 to 2018 at a center-of-mass energy of 13~TeV. Events are selected to have either one electron or muon and multiple jets, including those originated from the hadronisation of a bottom quark. Results are obtained by performing binned maximum-likelihood fits of the different simulated signal and SM backgrounds to the recorded data. The fit is performed on a discriminant obtained by combining several kinematic variables through the training of parameterised feed-forward neural networks, developed to optimise the sensitivity by separating signal and background events.\\

The document is structured into three main parts: The first part describes the theoretical and experimental setup, while the second part includes the $H^\pm\to tb$ analysis while the third part includes the $t\to qX$ analysis, both with a detailed description of the strategy and their results. Chapter~\ref{chapter:SM} focuses on the Standard Model and the models that motivate the searches. Chapter~\ref{chapter:ATLASLHC} provides an overview of the LHC and the ATLAS experiment. Chapters~\ref{chapter:EventSim} and~\ref{chapter:EventReco} present the essential aspects of the simulation and reconstruction of simulated proton-proton collisions, while Chapter~\ref{chapter:ML} presents the machine learning techniques and statistical tools used in the different analyses. Chapter~\ref{chapter:Htbanalysis} introduces and explains the details of the $H^\pm\to tb$, with the results collected in Chapter~\ref{chapter:Htbresults}. Similarly, the details and results of the $t\to qX$ search are discussed in Chapters~\ref{chapter:tqXanalysis} and~\ref{chapter:tqXresults}. A summary of the work is provided in Chapter~\ref{chapter:summary}.