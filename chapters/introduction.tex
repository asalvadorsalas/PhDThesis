The discovery of the Higgs boson in 2012 by ATLAS and CMS [] is one of the most recent historic milestones in the field of particle physics. CERN hosts the LHC, which one of the main objectives in the physics program was to prove the existence of the Higgs boson. After this achievement, all particles predicted by the Standard Model are discovered, the theory which describes the fundamental particles and their interactions. Nevertheless, the ATLAS experiment continues to scrutinise the Standard Model by analysing the ever-increasing amount of particle collisions delivered by the LHC. There are many phenomena not covered by the current theory and any measurament that deviates from the predictions or the hint of a new particle will lay the foundation for a new path in particle physics. That is the reason for the precise measurement of the Higgs boson properties and searches related to found deviations in the leptonic sector.

Many theoretical expansions of the Standard Model contain additional scalar particles. For example, if the Higgs sector is built with one extra doublet (Two Higgs Doublet Models), a total of five scalars are predicted instead, including charged scalars. Another interesting possibility involves the presence of flavour-changing neutral current (FCNC) interactions, which are predicted to be heavily suppressed and any increased measurement would also point to new physics. Other theories predict higher contributions, in particular, FCNC interactions involving the top quark and the flavon, a new neutral scalar arising from a symmetry between quark families.
In this thesis, a direct search for charged Higgs bosons heavier than the top quark and a direct search for neutral scalars lighter than the top quark are presented. The charged Higgs process is searched produced in association with top and bottom quarks and decaying into a top-bottom, while the neutral scalar is searched from the FCNC decay of a top quark involving a $c$- or a $u$-quark, and decaying to a pair of $b$-quarks. Both searches are performed using the full Run-2 proton-proton collisions collected by the ATLAS experiment from 2015 to 2018 at a center-of-mass energy of 13~TeV. Limits on the production of charged Higgs bosons have been previously obtained by ATLAS with only the data from 2015 and 2016 for $H^\pm\to tb$ in the $180--600$~GeV mass range [], and more recently by CMS in the $200--600$~GeV mass range using the full Run-2, setting upper limits on the production cross section of X--X pb and X--X pb respectively. On the other hand, both ATLAS and CMS have searched for FCNC $t\to qH$ decays, with $q$ either a $c$- or $u$-quark, setting upper limits on the branching fraction of XXX and XXX. However the generic signature presented involving a new scalar lighter than the top quark is uncovered in literature.

The $H^\pm\to tb$ search presented in this thesis is performed for a mass range from 200 to 2000~GeV produced in association with top and bottom quarks and in final states with a single lepton. The limits on the $H^\pm\to tb$ production and the $t\to qX(bb)$ are set by means of a binned profile likelihood fit of the different simulated signal and Standard Model backgrounds to the collected data. The fit is performed on a discriminant obtained by combining several kinematic variables through different machine learning techniques, which allows to improve the sensitivity of the analysis by improving the separation of signal and background events. The largest part 
