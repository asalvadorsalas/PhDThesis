\chapter*{Abstract}
\addcontentsline{toc}{chapter}{Abstract} % Add the preface to the table of contents as a chapter


This thesis presents two searches for new scalars produced with the 139 fb$^{-1}$ proton-proton collision data at a centre-of-mass energy of 13 TeV, collected by the ATLAS detector at the CERN Large Hadron collider during Run~2.
Both searches are performed in multi-jet final states with one electron or muon and events categorised according to the multiplicity of jets and how likely these are to have originated from the hadronisation of a bottom quark. Parameterised feed-forward neural networks are used to discriminate between signal and background and included in maximum-likelihood fits to the data for the different mass hypotheses.

The first search is dedicated to charged Higgs bosons, predicted by various theories Beyond the Standard Model and motivated by the inadequacy of the Standard Model to explain some experimental phenomena. The work focuses on heavy charged Higgs bosons, heavier than the top quark, and decaying to a pair of top and bottom quarks, $H^\pm\to tb$, while the production is in association with a top and a bottom quark, $pp\to tbH^\pm$.
The search is performed in the mass range between 200 and 2000~GeV. No significant excess of events above the expected Standard Model background is observed, hence upper limits are set for the cross-section of the charged Higgs boson production times the branching fraction of its decay. Results are interpreted in the context of hMSSM, various Mh125 scenarios and 2HDM+a.

The second search targets FCNC decays of top quarks into a new scalar decaying into a pair of bottom quarks, $t \to u/c X(bb)$, in $t\bar{t}$ events. This novel study probes for the scalar on a broad mass range between 20 and 160 GeV and branching ratios below 10$^{-3}$. In the case of the Higgs boson, branching ratios for $t \to u/c H$ are predicted within the Standard Model to be of $\mathcal{O}(10^{-17})/\mathcal{O}(10^{-15})$. Several Beyond the Standard Model theoretical models predict new particles and enhanced branching ratios. In particular, simple extensions involve the Froggatt-Nielsen mechanism, which introduces a scalar field with flavour charge, the so-called flavon, featuring flavour violating interactions. As no significant excess is observed, upper limits for both FCNC decays $t\to uX$ and $t\to cX$ are computed. In addition, limits are set for the process involving the Standard Model Higgs.