\chapter{$H^+\to tb$ analysis results}
In order to test for the presence of an $H^+\to tb$ signal, a binned maximum-likelihood fit to the data is performed as described in Section. In total 18 fits are performed, one for each mass hypothesis, fitting the NN output evaluated at the corresponding mass as explained in Section. Four regions are used in the fit: 5j3b, 5j$\geq$4b, $\geq$6j3b and $\geq$6j$\geq$4b. The distributions have ten bins, except 5j$\geq$4b which have eight, and the irregular bin width of each region is the result of an sation that obtain the best sensitivity. Two initially unconstrained fit parameters are used to model the normalisation of the \ttb\ and \ttc\ backgrounds. The parameter of interest is the product of the production cross-section $\sigma(pp\to tbH^+)$ and the branching ratio BR($H^+\to tb$).

A total of XXX nuisance parameters are introduced in the fit. To speed up the fit and ease the convergence, the shape or normalisation compontents of a given systematic uncertainty is pruned if below a threshold of 1\%. In addition, smoothing techniques are applied to reduce the impact of statistical fluctuations when computing the templates of systematic uncertainties.

This section provides the expected and observed results on the fitted signal strength, CL$_{\text{s}}$ exclusion limits,  a brief review of the performance and the interpretation and combination of the results in the 2HDM+a model.

\section{Fit results}

The analysis optimisation is performed on \acrshort{MClabel} simulation and the performance is evaluated via \textit{Asimov} data instead of experimental data. The dataset is built from the nominal background and the chosen signal, thus the normalisation factors and nuisance parameters extracted from the fit are the default ones by construction. Nevertheless, the profile likelihood fit provides uncertainties on the signal strength and the expected upper limits. It is the standard procedure to optimise the analysis without looking at the experimental data to avoid introducing any bias, specially in sensitive regions. Once the desired expected sensitivity is obtained and the background modelling reproduces the experimental data in signal-depleted regions, more experimental data is added to the fit.

Pre-ft and post-ft distributions of the BDT output in the signal regions are shown from
Figure 6.9 to Figure 6.12. The agreement between the observed and expected yields is
already good at pre-ft level, but an improvement can be observed after the ft

The agreement between
the observed and expected event yields in all regions, after performing the ft, is shown in
Figure 6.8. The total uncertainty (systematic and statistical) is shown as a light-blue shadedband. The results are reported for the 200 and 800 GeV signal hypotheses. With respect
to the pre-ft comparison (Figure 6.3), data and MC show an improved compatibility. The
signal does not appear because the signal strength is ftted to slightly negative values.
Table 4 shows the event yields after the background-plus-signal fit under the 200 GeV
and 800 GeV H+ mass hypotheses

The nuisance
parameters and normalisation factors are moved from their original values, accommodating for normalisation and
shape diferences between the observed and predicted distributions. These normalisation factors range from 1.2 to 1.6
(0.2 to 1.8) with a typical uncertainty of 0.2 (0.6) for the tt¯ + ≥1b (tt¯ + ≥1c) background,
depending on the H+ mass hypothesis used in the fit. The fitted tt¯ + heavy-flavour jets
backgrounds in the different fits are consistent within two standard deviations
No significant excess above the expected SM background
is observed in all regions and mass intervals and upper limits on the cross-section times
branching ratio are derived as function of the H+ mass.

The 95\% confidence level (CL) upper limits on $\sigma(pp\to tbH^+)\times$B$(H^+ \to tb)$ obtained using the CLS method are presented in figure 6. Uncertainties in the predicted H+ cros sections or branching ratios are not included. The observed (expected) limits range from $\sigma$ $\times$ B = 3.6 (2.6) pb at mH+ = 200 GeV to $\sigma\times$B = 0.036 (0.019) pb at mH+ = 2 TeV.  Compared to the previous ATLAS search for tbH+ production followed by H+ → tb decays with
36~fb$^{-1}$, the observed $\sigma\times$B limits improved by 5\% to 70\%, depending on the H+ mass,
apart from the lowest one. Figure 7 shows 95\% CL exclusion limits set on tan β as a function of mH+ for various
benchmark scenarios in the MSSM. It is the first time that they are shown for all M125h
available scenarios using the H+ → tb channel. In the hMSSM framework, effective couplings of the lighter Higgs boson to the top quark, bottom quark and vector bosons are
derived from fits to LHC data on the production and decay rates of the observed Higgs
boson, including the limits from the search for heavier neutral and charged Higgs boson
states. The M125h, M125h ( tildechi), M125h(tildetau), M125h (alignment) and M125 h1
(CPV) scenarios also feature a scalar particle with mass and couplings compatible with those of the observed Higgs
boson, and force a significant portion of their parameter space to be compatible with the
limits from searches for supersymmetric particles. In the M125h scenario, all supersymmetric particles are relatively heavy and the decays of the MSSM Higgs bosons are essentially unaffected, whereas the M125h (ch) and M125h (tau) models include either light charginos and neutralinos (M125h( chi)) or light staus (M125h(tau )). In both cases a charged Higgs boson of sufficiently high mass is allowed to decay into the supersymmetric particles. Finally, the
value of tan beta in both the M125h (alignment) scenario, characterised by one of the two neutral CP-even scalars having ouplings like those of the SM Higgs boson, and the M125h1(CPV) scenario, which includes CP violation in the Higgs sector, is already constrained to be in the range 1-20 by previous searches at the LHC %https://arxiv.org/abs/1409.5615
. Uncertainties in the predicted H+
cross-sections or branching ratios are not included in the limits. For all scenarios except
the hMSSM, Higgs boson masses and mixing (and effective Yukawa couplings) have been
calculated with the code FeynHiggs %https://arxiv.org/abs/hep-ph/9812320 https://arxiv.org/abs/hep-ph/9812472 https://arxiv.org/abs/hep-ph/0212020 https://arxiv.org/abs/1312.4937 https://arxiv.org/abs/hep-ph/0611326 https://arxiv.org/abs/1608.01880 https://arxiv.org/abs/1706.00346
. Whereas in the hMSSM the branching ratios are computed solely with HDECAY %https://arxiv.org/abs/1801.09506 https://arxiv.org/abs/hep-ph/9704448
, all other scenarios combine the most
precise results of FeynHiggs, HDECAY and PROPHECY4f %https://arxiv.org/abs/hep-ph/0604011 https://arxiv.org/abs/hep-ph/0611234
. In the context of these scenarios, tan beta values below 1 are observed to be excluded at 95\% CL for H+ masses between 00 and $\sim$790 GeV. High values of $\tan\beta$ between 34 and 60 are excluded in a similar mass range in the hMSSM and 125h(chi) models. The most stringent limit, $\tan\beta$ < 2.1 excluded at 95\% CL, is set for the H+ mass hypothesis of 225 GeV in the hMSSM and for the 250 GeV H+ mass hypothesis in the M125h, M125h(chi), M125h(tau), M125h(alignment) and M125h1 (CPV) models. The low $\tan\beta$ and high H+ mass parameter space was not excluded by any other analysis before, while the high $\tan\beta$ was already excluded by the H+ → τν search. Compared to previous results of the same search channel, this analysis excludes a broader region of large $\tan\beta$. Additionally, an extended region of low $\tan\beta$ and low and high H+ masses is also excluded.

%blinding strategy? 
%correlations?
%tests?

\subsection{Dominant uncertainties}

The uncertainty associated to the fit result is mainly driven by systematic uncertainties which are
described in Section 13.6. The impact of the different nuisance parameters and hence the systematic
uncertainty sources are evaluated and ranked by their impact on the signal strength $\Delta\mu$ which is the
shift in µincl. evaluated in a separate fit, with fixed nuisance parameter $\hat{\theta} \pm \Delta\hat{\theta}$ , with respect to the nominal fit. The parameters with a hat (e.g. $\hat{\theta}$) correspond to the best-fit values and those without a hat are the corresponding pre-fit values. \Delta\hat{\theta} is the shift on $\mu$. when shifting a nuisance parameter from its fitted value $\hat{\theta}$ by one standard deviation $\Delta\hat{\theta}$ upwards and downwards. The 20 highest ranked nuisance parameters
according to their post-fit impact are shown in Figure 14.11. The upper axis represents the scale for
the pre-fit and post-fit impact on $\mu$. The pre-fit (post-fit) impact is given as $\hat{\theta} \pm \Delta\theta (\hat{\theta} \pm \Delta\hat{\theta})$, with $\Delta\theta$
(\Delta\hat{\theta}) the pre-fit (post-fit) uncertainties. While $\Delta\theta$ is set to one which is the pre-fit prior corresponding to one standard deviation, the post-fit value of $\Delta\hat{\theta}$ is typically smaller due to constraints from the fit. Both the pre-fit and post-fit impacts are shown as empty and filled rectangles, respectively. The lower axis indicates the scale of the pull of a nuisance parameter defined as $\hat{\theta} -\theta_0 / \Delta\theta$ with $\theta_0$ the nominal pre-fit value. The pulls are indicated as black points with their respective error bar. The background normalisation k(tt¯ + >1b) is drawn with its actual value and since its pre-fit impact is not properly defined,
it is not shown. The six highest-ranked nuisance parameters are all associated to the tt¯ + >1b modelling where the two dominant systematic uncertainties are coming from the NLO matching which
are retrieved from the comparison of the two generators MADGRAPH5_aMC@NLO+PYTHIA8 and
POWHEGBOX+PYTHIA8. Besides the uncertainties from the tt¯ + >1b modelling, also tW and
signal modelling related nuisance parameters are showing up in the ranking. However, their impact is small compared to the tt¯ + >1b nuisance parameters. In addition, the impact on the signal
strength is evaluated in groups of systematic uncertainty sources listed in Table 14.1. A consistent
picture is drawn, the tt¯ + >1b modelling dominates the systematic uncertainties followed by the
signal modelling and the tW modelling. The largest instrumental uncertainty is originating from the
flavour-tagging calibration. Moreover, the available MC statistics for the background (Backgroundmodel statistical uncertainty) is of similar size as the flavour-tagging uncertainties, which can be
reduced by generating more events. The largest pulls are coming from the tt¯ + >1b ISR and the p
bb
T
shape uncertainty. The tt¯ + >1b
ISR nuisance parameter is pulled by about 1.4$\sigma$, mainly correcting for the mismodelling of extra radiation in tt¯ + >1b events. Thus a softer renormalisation and factorisation scale is favoured by data in
the ME calculation and should be taken into account in the MC production for a future analysis. Extensive studies were performed understanding this pull in detail. In particular, the distribution of the
number of jets, which are used to categorise events, is corrected as shown pre-fit in Figure 13.5 (b)
and post-fit in Figure 14.9 (a) due to this pull. The shape of the BDT distributions used as input
for the fit are not found to be affected by this pull. In addition, it was checked if decorrelating the
tt¯ + >1b ISR nuisance parameter across all analysis regions would have an impact, but no real differences were spotted. The largest pull was seen in the dilepton CR>4j
3b hi while all other pulls not

Table impact < 
rankings

Upper limits <? 
exclusions <

2HDM+a combination
---------------
Posfit NN <
Summary posfit
postift yields < 