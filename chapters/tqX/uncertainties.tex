\section{Systematic uncertainties}
The presented analysis is heavily affected by systematic uncertainties from different sources, explained in this Section. There are two main categories of systematic uncertainties: experimental uncertainties, mainly associated to the reconstruction of the various physics objects, and the modelling uncertainties related to the signal and background process modelling in MC. In total, X nuisance parameters are used and summarised in Table~\ref{tqX:tablesys}, corresponding to the systematic components, and the \ttb\ and \ttc\ normalisation factors are included in the fit. The systematic uncertainties can either affect both the shape and the normalisation (SN), only the normalisation (N) or only the shape (S) of a process. Some uncertainty sources might consist of several independent components, e.g. the $b$-jet efficiency calibrations, and one nuisance parameter is associated to each component. In addition, for every bin considered in the analysis two nuisance parameters are assigned, one for signal and one for background, to take into account the uncertainties coming from the finite statistics of the \acrshort{MClabel} samples.

\begin{table}[htbp]
  \centering
  \begin{tabular}{lcc}
  \toprule \toprule
  \textbf{Systematic uncertainty}  & \textbf{Type} & \textbf{Components} \\
  \midrule
  \multicolumn{3}{l}{\textbf{Experimental uncertainties}} \\
  \midrule
  Luminosity             			  	        &   N  &  1  \\
  Pileup modelling       				        &  SN  &  1  \\
  \midrule
  \multicolumn{3}{l}{\textit{Physics objects}} \\
  \hspace{2ex}Electrons 				        &  SN  &  4 \\
  \hspace{2ex}Muons		  		  	            &  SN  & 10  \\
  \hspace{2ex}Jet energy scale                  &  SN  & 29 \\
  \hspace{2ex}Jet energy resolution             &  SN  &  9 \\
  \hspace{2ex}Jet vertex tagger                 &  SN  &  1 \\
  \hspace{2ex}$E_\textrm{T}^{\textrm{miss}}$    &  SN  &  3 \\
  \midrule
  \multicolumn{3}{l}{\textit{$b$-tagging}} \\
  \hspace{2ex}Efficiency         			    &  SN  & 45  \\
  \hspace{2ex}Mis-tag rate ($c$)			    &  SN  & 15  \\
  \hspace{2ex}Mis-tag rate (light)			    &  SN  & 20  \\
  \midrule
  \multicolumn{3}{l}{\textbf{Signal modelling}} \\
  \midrule
  \multicolumn{3}{l}{\textit{Signal}} \\
  \hspace{2ex} \ttl\ PS \& hadronisation                  &  SN  & 3   \\
  \hspace{2ex} \ttl\ NLO                  &  SN  &  3 \\
  \hspace{2ex} \ttl\ radiation            &  SN  &  5 \\
  \hspace{2ex} \ttl\ reweighting          &  SN  &  4 x 3 \\
  \\
  \midrule
  \multicolumn{3}{l}{\textit{\ttbar\ background}} \\
  \hspace{2ex}$t\bar t +\ge 1c$ normalisation   &   N & 1 \\
  \hspace{2ex}$t\bar t +\ge 1b$ normalisation   &   N & 1 \\
  \hspace{2ex}$t\bar t$ reweighting             &   SN   & 4 x 3  \\
  \hspace{2ex}$t\bar t +$ light modelling       &  SN  &  7 x 3 \\
  \hspace{2ex}$t\bar t +\ge 1c$  modelling      &  SN  &  7 x 3 \\
  \hspace{2ex}$t\bar t +\ge 1b$  modelling      &  SN  &  8 x 3 \\
  \hspace{2ex}$t\bar t$: $W\to cb$ normalisation & N & 1 \\
  \hspace{2ex}$t\bar t$: $W\to cb$ modelling & SN & 2 x 3 \\
  \midrule
  \multicolumn{3}{l}{\textit{Other backgrounds}} \\
  \hspace{2ex}Single top cross-section		       &   N  &  1  \\
  \hspace{2ex}Single top modelling			        &  SN  &  22  \\
  \hspace{2ex}Diboson normalisation			        &   N  &  3  \\
  \hspace{2ex}$V$+jets			                    &   N  &  3  \\
  \hspace{2ex}$t\bar{t}V$ cross-section			    &  N  &  1  \\
  \hspace{2ex}$t\bar{t}H$ cross-section		    &   N  &  1  \\
  \hspace{2ex}$t\bar{t}H$ modelling		         &   SN  &  2  \\
  \hspace{2ex}$tH$ cross-section		            &   N  &  1  \\
  \bottomrule
  \bottomrule
  \end{tabular}
  \caption{
  Overview of the systematic uncertainties included in the analysis.
  An "N" means that the uncertainty is taken as normalisation-only for all processes and regions affected,  whereas "SN" means that the uncertainty is taken on both shape and normalisation. Some of the systematic uncertainties are split into several components for a more accurate treatment: the number of such components is indicated in the rightmost column. There is one component for each jet multiplicity for the \ttbar\ modelling uncertainties, explicitely indicated as "x 3".
  }
  \label{tqX:tablesys}
\end{table}

\subsection{Experimental uncertainties}
The experimental uncertainties have in general a low impact on the final fit. Only the uncertainties associated to jets and $b$-tagging have influence. All experimental nuisance parameters are correlated across all analysis regions and processes.

The total uncertainty on the integrated luminosity of the full Run~2 is the only experimental uncertainty that only affects the normalisation, and is measured to be 1.7\% %https://cds.cern.ch/record/2677054
The pile-up modelling uncertainty, that account for the related differences between data and simulation, is also considered.%https://journals.aps.org/prl/abstract/10.1103/PhysRevLett.117.182002

\subsubsection{Jets and heavy-flavour tagging}
The uncertainties associated to jets are the most relevant experimental sources in this analysis. Although the single components are in the range of 1\%-5\%, the analysis is most sensitive on events with large number of jets, thus enhancing their effect. The uncertainties on the jet energy scale and resolution add up to 29 and 9 nuisance parameters, respectively [120] %https://arxiv.org/abs/2007.02645, https://arxiv.org/abs/1910.04482 .

The uncertainties for the jet energy scale are extracted from test-beam and LHC data as well as from simulation. Further uncertainty sources are also considered assuming a conservative uncertainty of $\pm$50\% on the quark-gluon fraction for the simulation of jets with different flavours. Moreover, pile-up corrections are taken into account as well as uncertainties from jet kinematics as well as differences between the two detector simulations used in the analysis. The jet energy resolution uncertainties are extracted from measured di-jet events using Run~2 data and MC simulation. Finally, the jet vertex tagger uncertainty is extracted from data-\acrshort{MClabel} calibrations measured %https://link.springer.com/article/10.1140/epjc/s10052-016-4430-0
in $Z\to\mu^+\mu^-$ events.

Uncertainties related to $b$-tagging are relevant in this analysis due the use of the $b$-jets for the selection and the use of their kinematics and the PCBT score in the NN. The $b$-tagging calibrations are derived separately for jets containing $b$-hadrons, $c$-hadrons or neither as a function of \pT\ and the different $b$-tagging working points in dedicated calibration analysis targetting different topologies. The different uncorrelated sources are obtained from a principal component analysis (eigenvalue decomposition) and are in the range of 2\%-10\% for the $b$-jet efficiency calibration and between 10\% to 25\% and
15\% to 50\% for the c-jets and light-flavour jets mis-tag rate calibration, respectively. In total, the flavour-tagging uncertainties are decomposed into 85 components.

\subsubsection{Leptons}
Even though the systematic uncertainties related to leptons have a small effect, 14 different uncertainty sources are taken into account [127, 128]. % https://arxiv.org/abs/1908.00005 https://arxiv.org/abs/1603.05598 
The components are related to the trigger, reconstruction, identification and isolation efficiencies for electrons (four components) and muons (ten components).

\subsubsection{Missing transverse momentum}
The systematic uncertainties associated to the \MET\ have minor impact as it is only used in the selection to reduce QCD background. Since the \MET\ is calculated from the reconstructed physics objects and a soft term, the energy scale and resolution uncertainties from the physics objects are propagated to the \MET\ together with an additional component for the soft term.

\subsection{Modelling Uncertainties}
In contrast to the experimental uncertainties, the modelling uncertainties are not correlated across all background and signal processes, but typically they are still correlated across analysis regions with some exceptions. The uncertainties are split into several components depending on the signal and background processes as well as into different physics effects in \acrshort{MClabel} generators. While the cross-section, branching fraction and normalisation uncertainties only affect the normalisation of the physics processes, all other modelling uncertainties are also sensitive to shape effects.


\subsubsection{Signal modelling}

Several normalisation and shape uncertainties are taken into account for the $t\to qX$. Since no alternative signal samples are used in the analysis, the uncertainties from the choice of NLO generator, PS and hadronisation, and reweighting of the \ttl\ background are assigned to the signal. These uncertainties are chosen to be correlated with the \ttl\ background, motivated by the fact that the \ttbar\ pair generated in the signal sample should modeled as the \ttl\ background process.

\subsubsection{\ttbar\ modelling}

The \ttbar\ process is the most important background in the analysis and a large number of uncertainties are considered for an apropiate modelling. Since the composition of the \ttbar\ subcategories are different in the analyisis' regions, different effects are expected. The \ttb\ and \ttc\ processes are fairly sensitive to differences in the precision of the \acrshort{MElabel} calculation or the flavour scheme. The \ttl\ process profit from precise measurements [ref]. Hence, all systematic uncertainties associated to \ttbar\ are uncorrelated across the \ttb, \ttc\ and \ttl\ categories, having separate nuisance parameters. Furthermore, the components are also uncorrelated among jet multiplicities. Unless stated otherwise, the nuisance parameters are correlated among bins and among regions with the same jet multiplicity. Table~\ref{tqX:tablesysttbar} summarises the uncertainties applied to the \ttbar\ background. 

\begin{table}[htbp]
  \centering
  \small
  \addtolength{\leftskip} {-2cm} % menja marges
  \addtolength{\rightskip}{-2cm}
  \begin{tabular}{llll}
  \toprule
  \toprule
  Uncertainty source      & \multicolumn{2}{l}{Description} & Components \\
  \midrule
  \ttb\ normalisation     & \multicolumn{2}{l}{$\pm$50\%} & \ttb \\
  \ttc\ normalisation     & \multicolumn{2}{l}{$\pm$50\%} & \ttc \\
  \ttbar\ reweighting        & \multicolumn{2}{l}{Uncertainties of fitted function parameters} & \ttbar\ and $Wt$ \\
  \midrule
  $\mu_R$             	  &   Scaling by 0.5 (2.0) & in \POWHEGBOX+\PYTHIA &  \ttbar  \\
  $\mu_F$             	  &   Scaling by 0.5 (2.0) & in \POWHEGBOX+\PYTHIA &  \ttbar  \\
  $\mu_R\times\mu_F$  	  &   Scaling both by 0.5 (2.0) & in \POWHEGBOX+\PYTHIA &  \ttbar  \\
  ISR                     &   Varying $\alpha_{S}^{ISR}$    & in  \POWHEGBOX+\PYTHIA    &  \ttbar   \\
  FSR                     &   Varying $\alpha_{S}^{FSR}$    & in  \POWHEGBOX+\PYTHIA    &  \ttbar   \\
  NLO matching            & \MGMCatNLO+\PYTHIA              & vs.  \POWHEGBOX+\PYTHIA    &  \ttbar   \\
  PS \& hadronisation     & \POWHEGBOX+\HERWIG              & vs.  \POWHEGBOX+\PYTHIA    &  \ttbar   \\
  \ttb\ modelling       & \multicolumn{2}{l}{4FS vs. 5FS}  &   \ttb       \\
  \bottomrule\bottomrule
  \end{tabular}
  \caption{
    Summary of the sources of systematic uncertainty for the \ttbar\ background modelling.
    The last column of the table indicates the sub-components for the corresponding systematic uncertainty.
    All systematic uncertainty sources,
    except those associated to the \ttbar\ reweighting,
    are treated as uncorrelated across the three components.
  }
  \label{tqX:tablesysttbar}
\end{table}

A normalisation uncertainty of 50\% is assumed separately for \ttb\ and \ttc. The choice is motivated by the level of agreement between data and prediction in the control regions for this background before the fit. The \ttb\ normalisation uncertainty is constrained from the fit, while the \ttc\ normalisation uncertainty is not. The background originating from \ttbar\ a $W$ decaying as $W\to cb$ is modelled with dedicated samples of simulated events.
For the uncertainties related to ISR, $\mu_R$ and $\mu_F$ are varied independently by a factor 0.5 (2.0) for the up (down) variation, with an extra nuisance parameter for the simultenous variation of $\mu_R$ and $\mu_F$. The ISR and FSR components are obtained setting accordingly $\alpha_S^{ISR}$ to 0.140 (0.115) and $\alpha_S^{FSR}$ to 0.140 (0.115), where the nominal values are both 0.127.

Two-point systematics are derived for the rest of modelling uncertainties. For a given distribution, this type of uncertainties are obtained from the difference in the prediction when comparing two different samples generated with different \acrshort{MClabel} setups. The systematic uncertainty related to \acrshort{PSlabel} are retrieved by comparing the nominal setup, \POWHEGBOX+\PYTHIA~8 to the prediction of the sample generated with \POWHEGBOX+\HERWIG~7, where the \acrshort{PSlabel} has been modelled with a different generator. Similarly, the uncertainty related to the NLO matching is retrieved from a sample generated with \MGMCatNLO+\PYTHIA~8. 
To cover the differences between 4FS and 5FS, the nominal is compared to a \textsc{PowhegBoxRes}+\PYTHIA 8 (4FS) \ttb\ sample. This uncertainty is only applied to the \ttb\ component.

The weights derived in Section~\ref{tqX:secRW} that are applied to improve the modelling of the \ttbar\ background are also subject to uncertainties. The associated statistical uncertainties are varied obtaining 12 nuisance parameters, which are correlated between the different \ttbar\ components\footnote{as mentioned above, the \ttl\ components are correlated with the signal.}. Aside, the \ttbar\ samples with alternative \acrshort{MClabel} setups do not have the same composition of \ttbar\ subcomponents as the nominal sample, especially \POWHEGBOX+\HERWIG~7. This difference can change significantly the fractions in the fit however, the normalisation of the sub-processes in the analysis regions are measured with the fit. To avoid the propagation of the normalisation effect from the comparison, the alternative samples are scaled to ensure the same flavour composition as the nominal sample in the analysis regions. In addition, the same modelling issues that motivate the \ttbar\ weights can be found in the alternative samples, hence different sets of uncertainties are also scaled by the yield ratio between nominal and alternative \ttbar. The 50\% normalisation effects of \ttb\ and \ttc\ have an effect on the reweigthing, which is estimated by applying a dedicated reweighting computed with the normalisation effects. These propagation effects are correlated with the corresponding 50\% prior and assigned to the rest of individual \ttbar\ components.

\subsubsection{Other background modelling}
The systematic uncertainties associated to background processes other than \ttbar+jets play a subordinate role.

A +5\%/-4\% uncertainty is considered for the cross-sections of the three single-top production modes, estimated from averaging the theoretical uncertainties in $t$-, $s$- and $Wtt$ productions %https://arxiv.org/abs/1005.4451  https://arxiv.org/abs/1103.2792 https://arxiv.org/abs/1001.5034
. Uncertainties associated with the \acrshort{PSlabel} model, and with the NLO matching scheme are evaluated by comparing, for each process individually, the nominal \POWHEGBOX+\PYTHIA~8 sample with a sample produced using \POWHEGBOX+\HERWIG~7 and \MGMCatNLO+\PYTHIA~8, respectively. Similarly to the \ttbar\ background, the multiple ISR and FSR modelling uncertainties included and all the modelling uncertainties are decorrelated among jet multiplicity. The uncertainty associated to the interference between $Wt$ and \ttbar\ production at NLO [63] is assessed by comparing the nominal \POWHEGBOX+\PYTHIA~8 sample produced using the “diagram removal” scheme with an alternative sample produced with the same generator but using the “diagram subtraction” scheme.

Uncertainties affecting the normalisation of the $V$+jets background are estimated for the sum of $W$+jets and $Z$+jets. The agreement between data and the total background prediction is found to be within approximately 40\%, taken to be the total normalisation uncertainty correlated across all $V$+jets processes. An additional 25\% uncertainty is added in quadrature for each additional jet multiplicity beyond four, resulting in 47\% and 52\% in regions with five and six jets, respectively %https://arxiv.org/abs/0706.2569
.

Uncertainties in the diboson background normalisation include 5\% from the NLO theory cross-sections %https://arxiv.org/abs/hep-ph/9905386
, as well as an additional 24\% normalisation uncertainty added in quadrature for each additional inclusive jet-multiplicity bin, based on a comparison among different algorithms for merging LO matrix elements and parton showers %%https://arxiv.org/abs/0706.2569
. Therefore, the total uncertainty is 34\%, 42\% and 48\% for events with four, five,
and six jets, respectively. Recent comparisons between data and \SHERPA~2.1.1 for $WZ\to\ell'\nu\ell\ell+\geq$4 jets show agreement within the experimental uncertainty of approximately 40\% %https://arxiv.org/abs/1606.04017
, which further motivates the uncertainties above. Uncertainties in the $t\bar{t}V$, $tZ$, $t\bar{t}H$ and $tH$ cross-sections are 60\%, 60\%, +9/-12\% and 50\%, respectively, arising from the uncertainties in their respective NLO theoretical cross-sections %https://arxiv.org/abs/1610.07922  https://arxiv.org/abs/1901.03584
In addition, $t\bar{t}H$ uncertainties related to the \acrshort{PSlabel} and the NLO matching is similarly implemented comparing the nominal with the same type of alternative \acrshort{MClabel}.